% -*- coding:utf-8; mode:latex; -*-

% Copyright (C) 2018 Moritz Klammler <moritz.klammler@alumni.kit.edu>
%
% Copying and distribution of this file, with or without modification, are permitted in any medium without royalty
% provided the copyright notice and this notice are preserved.  This file is offered as-is, without any warranty.

\makeatletter

\usetheme{KIT}

\usepackage{ifluatex}
\usepackage{fontspec}
\usepackage{polyglossia}
\usepackage{amsmath,amsfonts}
\usepackage{csquotes}
\usepackage{etoolbox}
\usepackage{tabularx}
\usepackage{booktabs}
\usepackage{ragged2e}
\usepackage{multicol}
\usepackage{gnuplot-lua-tikz}
\usepackage[ttscale=0.85]{libertine}

\usepackage[%
  backend = biber,
  citestyle = authoryear,
  bibstyle = chem-acs,
  giveninits = true,
]{biblatex}

\usetikzlibrary{calc}
\usetikzlibrary{decorations.pathmorphing}
\usetikzlibrary{fit}
\usetikzlibrary{matrix}
\usetikzlibrary{positioning}

\setmainlanguage{english}

\usenavigationsymbols
\setbeamercovered{transparent}
\setcounter{tocdepth}{1}

\setbeamercolor*{bibliography entry author}{fg=KITgreen}
\setbeamercolor*{bibliography entry title}{fg=KITgreen}
\setbeamercolor*{bibliography entry journal}{fg=KITgreen}
\setbeamercolor*{bibliography entry note}{fg=KITgreen}

\newcommand*{\ResetBibliographyStyle}{%
  \setbeamercolor*{bibliography entry author}{fg=black}
  \setbeamercolor*{bibliography entry title}{fg=black}
  \setbeamercolor*{bibliography entry journal}{fg=black}
  \setbeamercolor*{bibliography entry note}{fg=black}
  \setbeamertemplate{bibliography item}{\KIT@mark}
  \renewcommand*{\bibfont}{\small}
}

\setbeamercolor{structure}{fg=KITgreen}
%\setbeamercolor*{palette secondary}{fg=KITgreen}
%\setbeamercolor*{palette tertiary}{fg=KITgreen}

\AtBeginSection[\relax]{\frame{\frametitle{Agenda}\tableofcontents[currentsection]}}

\setbeamerfont*{section in toc}{size=\large, series=\bfseries}
\setbeamerfont*{section in toc shaded}{size=\large}
\setbeamerfont*{subsection in toc}{size=\normalsize, series=\bfseries}
\setbeamerfont*{subsection in toc shaded}{size=\normalsize}
\setbeamerfont*{subsubsection in toc}{size=\small, series=\bfseries}
\setbeamerfont*{subsubsection in toc shaded}{size=\small}

\setbeamercolor*{section in toc}{fg=KITgreen}
\setbeamercolor*{section in toc shaded}{}
\setbeamercolor*{subsection in toc}{fg=KITgreen}
\setbeamercolor*{subsection in toc shaded}{}
\setbeamercolor*{subsubsection in toc}{fg=KITgreen}
\setbeamercolor*{subsubsection in toc shaded}{}

\setbeamertemplate{section in toc shaded}[default][100]
\setbeamertemplate{subsection in toc shaded}[default][100]
\setbeamertemplate{subsubsection in toc shaded}[default][100]

\csundef{beamer@@tmpop@section in toc@KIT}
\csundef{beamer@@tmpop@subsection in toc@KIT}
\csundef{beamer@@tmpop@subsubsection in toc@KIT}

\defbeamertemplate{section in toc}{KIT}{
  \noindent\leavevmode\inserttocsection\par
}
\defbeamertemplate{subsection in toc}{KIT}{
  \noindent\leavevmode\inserttocsubsection\par
}
\defbeamertemplate{subsubsection in toc}{KIT}{
  \noindent\leavevmode\inserttocsubsubsection\par
}

\newcommand*{\slidecite}[1]{%
  \newcommand*{\slide@cite}[1]{\noindent\color{red}\fullcite{##1}\par\smallskip}%
  \begin{tikzpicture}[remember picture, overlay]%
    \node[anchor = south west, xshift = 1em, yshift = 5ex] at (current page.south west) {%
      \parbox{0.9\paperwidth}{%
        \scriptsize\RaggedRight%
        \forcsvlist\slide@cite{#1}%
      }%
    };%
  \end{tikzpicture}%
}

\newcommand*{\textand}{\kern0.2em\textperiodcentered\kern0.2em}

\newcommand*{\GraphE}{E}
\newcommand*{\GraphGVE}{\Graph=(\GraphV,\GraphE)}
\newcommand*{\GraphV}{V}
\newcommand*{\Graph}{G}
\newcommand*{\IntsN}{\mathbb{N}}
\newcommand*{\Layout}{\Gamma}
\newcommand*{\Reals}[1][]{\ifblank{#1}{\mathbb{R}}{\mathbb{R}_{#1}}}
\newcommand*{\abs}[1]{\left\lvert#1\right\rvert}
\newcommand*{\enum}[1]{\mbox{\texttt{#1}}}
\newcommand*{\menum}[1]{\mbox{\text{\upshape\ttfamily#1}}}
\newcommand*{\percent}{\,\%}
\newcommand*{\vecnorm}[1]{\left\lVert#1\right\rVert}

\DeclareMathOperator{\dist}{dist}
\DeclareMathOperator{\length}{length}
\DeclareMathOperator{\mean}{mean}
\DeclareMathOperator{\stdev}{stdev}
\DeclareMathOperator{\stress}{\menum{STRESS}}

\renewcommand<>{\alert}[1]{{\only#2{\color{KITgreen}}{#1}}}

\newcommand*{\InputTikzGraph@Normal}[3][]{\tikz[x=#2, y=-#2, #1]{\input{#3}}}

\newcommand*{\InputTikzGraph@Dummy}[3][]{%
  \urldef\inputtikzgraph@url\url{#3}%
  \begin{tikzpicture}[x=#2, y=#2, #1]
    \booltrue{tikzgraphpreamble}
    \input{#3}
    \begin{scope}[yscale=\aspectratio]
      \draw[use as bounding box] (-0.5, -0.5) rectangle (0.5, 0.5);
      \node[font=\tiny] at (0, 0) {\inputtikzgraph@url};
    \end{scope}
  \end{tikzpicture}
}

\let\InputTikzGraph\InputTikzGraph@Normal
%\let\InputTikzGraph\InputTikzGraph@Dummy

\newcommand*{\InputLuatikzPlot}[2][]{\input{#2}}

\newcommand*{\PropertyDemo}[3][PROPERTY]{%
  \begin{frame}
    \frametitle{#3 (\enum{#1})}
    \begin{tabular}{ccc}
      \InputTikzGraph[graphcolor = KITpalegreen]{30mm}{pics/demograph-a.tikz}&
      \InputTikzGraph[graphcolor = KITpalegreen]{30mm}{pics/demograph-b.tikz}&
      \InputTikzGraph[graphcolor = KITblue]{30mm}{pics/demograph-c.tikz}\\[2ex]
      % It would be too simple if Gnuplot could just /use/ our TeX color names ...
      \InputLuatikzPlot{pics/#2-a.pgf}&
      \InputLuatikzPlot{pics/#2-b.pgf}&
      \InputLuatikzPlot{pics/#2-c.pgf}
    \end{tabular}
  \end{frame}
}

\newcommand*{\RandomThumbnailDemo}[2][x = 0.6\paperwidth, y = 0.3\paperheight]{%
  \begin{center}
    \begin{tikzpicture}[remember picture, overlay, #1]
      \ifdef{\RandomSeed}{\pgfmathsetseed{\RandomSeed}}{}
      \coordinate (origin) at (current page.center);
      \begin{scope}[shift = (origin)]
        \foreach \what in {#2} {
          \node<+->[thumbnail] at ({0.5-rnd}, {0.5-rnd}) {%
            \begin{tabular}{c}
              \InputTikzGraph{0.25\textwidth}{pics/\what.tikz}\\[1ex]
              \enum{\MakeUppercase{\what}}
            \end{tabular}
          };
        }
      \end{scope}
    \end{tikzpicture}
  \end{center}
}

% #1 ... expectation: (?) dunno, (-) negative, (+) positive
% #2 ... 8-digit graph id (currently not used)
% #3 ... 8-digit lhs layout id
% #4 ... 8-digit rhs layout id
% #5 ... discriminator (discmod, comb, stress)
\newcommand*{\GetDiscriminatorResult}[5]{%
  \ifcsvoid{msc@#3-#4@#5}{}{%
    \msc@assert@result{#1}{\csuse{msc@#3-#4@#5}}{#5}{#3}{#4}%
  }%
  \csuse{msc@#3-#4@#5}%
}

\ifluatex
  \newcommand*{\msc@assert@result}[5]{%
    \directlua{%
      signspec = "\luatexluaescapestring{#1}"
      p = tonumber("\luatexluaescapestring{#2}")
      discriminator = "\luatexluaescapestring{#3}"
      lhsid = "\luatexluaescapestring{#4}"
      rhsid = "\luatexluaescapestring{#5}"
      doit = tonumber(kpse.var_value("MSC_TEX_ASSERT") or "0")
      if doit then
          formatstring = string.gsub(
              "Checking assertion !s(!s, !s) !s 0 ...  !s (p = !+.3f !!)\@backslashchar n",
              "[!]", "\@percentchar\@percentchar"
          )
          if signspec == "+" then
              result = (p > 0) and "passed" or "failed"
              texio.write_nl(string.format(formatstring, discriminator, lhsid, rhsid, ">", result, p))
              if not (p > 0) then
                  tex.error("assertion failed")
              end
          end
          if signspec == "-" then
              result = (p < 0) and "passed" or "failed"
              texio.write_nl(string.format(formatstring, discriminator, lhsid, rhsid, "<", result, p))
              if not (p < 0) then
                  tex.error("assertion failed")
              end
          end
      else
          texio.write_nl("Assertions are not enables; set MSC_TEX_ASSERT=1 to enable\@backslashchar n")
      end
    }%
  }
\else
  \newcommand*{\msc@assert@result}[5]{}
\fi

\newcommand*{\ShowDemoScale}{\let\democolor@scale\democolor@}
\newcommand*{\ShowDemoResult}{\let\democolor@pin\democolor@}
\newcommand*{\ShowDemo}{\ShowDemoScale\ShowDemoResult}

\newcommand*{\democolor@}{KITred}
\newcommand*{\democolor@scale}{white}
\newcommand*{\democolor@pin}{white}

\pgfkeys{
  /msc/graphstudy/.cd,
  comb/.estore in = \msc@graphstudy@comb,
  discmod/.estore in = \msc@graphstudy@discmod,
  stress/.estore in = \msc@graphstudy@stress,
}

\newcommand*{\DiscriminatorDemo}[3][]{%
  \begin{center}
    \pgfkeys{/msc/graphstudy/.cd, #1}
    \begin{tikzpicture}[x = 0.01\paperwidth, y = 0.01\paperheight]
      \begin{scope}[scale = 0.4]
        \node at (-50, 0) {\InputTikzGraph{0.3\textwidth}{#2}};
        \node at (+50, 0) {\InputTikzGraph{0.3\textwidth}{#3}};
        \coordinate (atscale) at (0, -75);
        \begin{scope}[shift = (atscale)]
          \path[draw = \democolor@scale, thick, <->] (-110, 0) -- (+110, 0);
          \path[draw = \democolor@scale, thick] (0, -3) -- (0, +3);
          \node[text = \democolor@scale, font = \footnotesize, anchor = north, yshift = -0.5ex] at (-100, 0) {$-1$};
          \node[text = \democolor@scale, font = \footnotesize, anchor = north, yshift = -0.5ex] at (   0, 0) {$0$};
          \node[text = \democolor@scale, font = \footnotesize, anchor = north, yshift = -0.5ex] at (+100, 0) {$+1$};
          \foreach \what in {comb, stress} {
            \ifcsvoid{msc@graphstudy@\what}{}{%
              \edef\msc@graphstudy@what{\csuse{msc@graphstudy@\what}}%
              \path[draw = \democolor@pin] (\msc@graphstudy@what, 3pt) -- +(-2, +8) -- +(+2, +8) -- cycle;
              \node[anchor = south, text = \democolor@pin, font = \scriptsize] at (\msc@graphstudy@what, 10) {%
                \enum{\MakeUppercase{\what}}
              };
            }
          }
          \ifdefvoid{\msc@graphstudy@discmod}{}{%
            \path[fill = \democolor@pin] (\msc@graphstudy@discmod, 3pt) -- +(-2, +8) -- +(+2, +8) -- cycle;
          }
        \end{scope}
      \end{scope}
    \end{tikzpicture}
  \end{center}
}

\newcommand*{\FormatPercentageWithError}[3][2.5em]{%
  \ifblank{#2}{}{\ensuremath{(\makebox[#1][r]{\ensuremath{#2}}\,\pm\makebox[2em][r]{\ensuremath{#3}}\,)\percent}}%
}

\newcommand*{\InputCompetingMetrics}[1]{%
  \begingroup%
  \newcommand*{\CompetingMetricResult}[5][\@undefined]{%
    ##1 & \FormatPercentageWithError{##2}{##3} & \FormatPercentageWithError[2em]{##4}{##5}\\
  }%
  \begin{tabularx}{\linewidth}{Xr@{\qquad}r}%
    \toprule%
    \textit{Metric} & \textit{Success Rate} & \textit{Advantage}\\%
    \midrule%
    \input{#1}%
    \bottomrule%
  \end{tabularx}%
  \endgroup%
}

\newcommand*{\InputPunctureResult}[2][\relax]{%
  \begingroup%
  \newcommand*{\PunctureResult}[5][\@undefined]{%
    ##1 & \FormatPercentageWithError{##2}{##3} & \FormatPercentageWithError{##4}{##5}\\
  }%
  \input{#1}%
  \begin{tabularx}{\linewidth}{Xr@{\qquad}r}%
    \toprule%
    \textit{Property} & \textit{Sole Exclusion} & \textit{Sole Inclusion}\\%
    \midrule%
    \input{#2}%
    \midrule%
    \PunctureResult[\textit{Baseline Using All Properties}]{\XValSuccessMean}{\XValSuccessStdev}{}{}%
    \bottomrule%
  \end{tabularx}%
  \endgroup%
}

\newbool{tikzgraphpreamble}

\newlength{\vertexsize}
\setlength{\vertexsize}{2pt}

\newcommand*{\graphcolor}{KITgreen}

\pgfkeys{
  /tikz/.cd,
  graphcolor/.store in = \graphcolor,
}

\tikzset{%
  vertex/.style = {
    shape = circle,
    fill = \graphcolor,
    inner sep = 0pt,
    minimum size = \vertexsize,
  },
  edge/.style = {
    draw = \graphcolor,
    thin,
  },
  smartbox/.style = {
    fill = KITyellow!10,
    draw = KITyellow,
    inner sep = 1ex,
    preaction = {draw, line width = 5pt, white},
  },
  thumbnail/.style = {
    font = \small,
    fill = white,
    draw = gray,
    inner sep = 1em,
    preaction = { draw, line width = 2mm, white },
  },
}

\makeatother
