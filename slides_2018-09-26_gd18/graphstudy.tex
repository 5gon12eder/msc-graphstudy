% -*- coding:utf-8; mode:latex; -*-

% Copyright (C) 2018 Moritz Klammler <moritz.klammler@alumni.kit.edu>
% Copyright (C) 2018 Tamara Mchedlidze <mched@iti.uka.de>
% Copyright (C) 2018 Alexey Pak <alexey.pak@iosb.fraunhofer.de>
%
% This work is licensed under a Creative Commons Attribution-NonCommercial-NoDerivatives 4.0 International License
% (https://creativecommons.org/licenses/by-nc-nd/4.0/).

\documentclass{beamer}

% -*- coding:utf-8; mode:latex; -*- %

% Copyright (C) 2018 Moritz Klammler <moritz.klammler@student.kit.edu>
%
% Copying and distribution of this file, with or without modification, are permitted in any medium without royalty
% provided the copyright notice and this notice are preserved.  This file is offered as-is, without any warranty.

\makeatletter

\usepackage{booktabs}
\usepackage{tabularx}
\usepackage{gnuplot-lua-tikz}

\newcommand*{\email}[1]{\mbox{\texttt{\href{mailto:#1}{#1}}}}

\newcommand*{\Reals}[1][]{\ifblank{#1}{\mathbb{R}}{\mathbb{R}_{#1}}}
\newcommand*{\RealsPos}{\Reals[>0]}
\newcommand*{\RealsNN}{\Reals[\geq0]}
\newcommand*{\IntsZ}{\mathbb{Z}}
\newcommand*{\IntsN}{\mathbb{N}}
\newcommand*{\IntsNz}{\mathbb{N}_0}
\newcommand*{\BitZero}{\text{\texttt{0}}}
\newcommand*{\BitOne}{\text{\texttt{1}}}
\newcommand*{\Bits}{\{\BitZero,\BitOne\}}

\let\vec\relax
\let\Vec\relax
\let\implies\Rightarrow
\let\impliedby\Leftarrow
\let\equivalent\Leftrightarrow
\newcommand*{\suchthat}{:}
\newcommand*{\degree}{^\circ}
\let\conjunction\cap
\let\disjunction\cup
\newcommand*{\diff}[1]{\mathrm{d}#1\kern0.3em}

\newcommand*{\Layout}{\Gamma}
\newcommand*{\Graph}{G}
\newcommand*{\GraphV}{V}
\newcommand*{\GraphE}{E}
\newcommand*{\vecz}{0}

\newcommand*{\GraphGVE}{\Graph=(\GraphV,\GraphE)}

\newcommand*{\CMake}{\mbox{CMake}}
\newcommand*{\CXX}{\mbox{C++}}
\newcommand*{\JavaScript}{\mbox{JavaScript}}
\newcommand*{\Keras}{\mbox{Keras}}
\newcommand*{\Python}{\mbox{Python}}
\newcommand*{\SQLite}{\mbox{SQLite}}
\newcommand*{\TensorFlow}{\mbox{TensorFlow}}

\newcommand*{\Entropy}{\@ifstar\EntropyDifferential\EntropyDiscrete}
\newcommand*{\EntropyDiscrete}{S}
\newcommand*{\EntropyDifferential}{\bar{S}}

\DeclareMathOperator{\BigO}{\mathcal{O}}
\DeclareMathOperator{\dist}{dist}
\DeclareMathOperator{\edgelen}{length}
\DeclareMathOperator{\mean}{mean}
\DeclareMathOperator{\rms}{rms}
\DeclareMathOperator{\sign}{sign}
\DeclareMathOperator{\stdevp}{stdevp}
\DeclareMathOperator{\stdev}{stdev}
\DeclareMathOperator{\stress}{stress}

\DeclareMathOperator*{\Conjunction}{\bigcap}
\DeclareMathOperator*{\Disjunction}{\bigcup}
\DeclareMathOperator*{\argmax}{arg\,max}
\DeclareMathOperator*{\argmin}{arg\,min}

\newcommand*{\abs}[1]{\left\lvert#1\right\rvert}
\newcommand*{\card}[1]{\left\lvert#1\right\rvert}
\newcommand*{\vecnorm}[1]{\left\lVert#1\right\rVert}
\newcommand*{\floor}[1]{\left\lfloor#1\right\rfloor}
\newcommand*{\ceil}[1]{\left\lceil#1\right\rceil}
\newcommand*{\nint}[1]{\left\lfloor#1\right\rceil}
\newcommand*{\multiset}[1]{\left[#1\right]}

\newcommand{\bra}[2][\relax]{%
  \ifx\relax#1\relax%
    \left\langle#2\right|%
  \else%
    \left\langle#2\middle|#1\right|%
  \fi%
}

\newcommand{\ket}[2][\relax]{%
  \ifx\relax#1\relax%
    \left|#2\right\rangle%
  \else%
    \left|#1\middle|#2\right\rangle%
  \fi%
}

\newcommand{\braket}[3][\relax]{%
  \ifx\relax#1\relax%
    \left\langle#2\middle|#3\right\rangle%
  \else%
    \left\langle#2\middle|#1\middle|#3\right\rangle%
  \fi%
}

\newcommand*{\enum}[1]{\mbox{\upshape\ttfamily#1}}
\newcommand*{\menum}[1]{\mbox{\text{\upshape\ttfamily#1}}}
\newcommand*{\QuasiNd}{\enum{QUASI\(\langle{n}\rangle\)D}}
\newcommand*{\TorusN}{\enum{TORUS\(\langle{n}\rangle\)}}
\newcommand*{\code}[1]{\mbox{\texttt{#1}}}

\renewcommand*{\acroenparen}[1]{(\emph{#1})}

\newcommand*{\latintext}[1]{\emph{\textlatin{#1}}}

\newcommand*{\InputTikz}[2][]{\tikz[#1]{\input{#2}}}

\ifdraft{%
  \newcommand*{\InputTikzGraph}{\@ifstar\InputTikzGraph@Dummy\InputTikzGraph@Dummy}
}{%
  \newcommand*{\InputTikzGraph}{\@ifstar\InputTikzGraph@Big\InputTikzGraph@NotBig}
}

\newcommand*{\InputTikzGraph@NotBig}[3][]{\tikz[x=#2, y=#2, #1]{\input{#3}}}
\newcommand*{\InputTikzGraph@Big}[3][]{\tikz[x=#2, y=#2, #1]{\setlength{\vertexsize}{2pt}\input{#3}}}

\newcommand*{\InputTikzGraph@Dummy}[3][]{%
  \urldef\inputtikzgraph@url\url{#3}%
  \begin{tikzpicture}[x=#2, y=#2, #1]
    \booltrue{tikzgraphpreamble}
    \input{#3}
    \begin{scope}[yscale=\aspectratio]
      \draw[use as bounding box] (-0.5, -0.5) rectangle (0.5, 0.5);
      \node[font=\tiny] at (0, 0) {\inputtikzgraph@url};
    \end{scope}
  \end{tikzpicture}
}

\newlength{\samplelayoutwidth}
\newlength{\samplelayoutheight}
\newbool{tikzgraphpreamble}

\newcommand*{\UpdateSampleLayoutHeight}[1]{%
  \booltrue{tikzgraphpreamble}
  \input{#1}
  \setlength{\samplelayoutheight}{\aspectratio\samplelayoutwidth}
  \boolfalse{tikzgraphpreamble}
}

\newcommand*{\AtBeginSampleLayout}{%
  \path[tikzgraphbbox, use as bounding box]
       (-0.5\samplelayoutwidth, -0.5\samplelayoutheight) rectangle (+0.5\samplelayoutwidth, +0.5\samplelayoutheight);
}

\newcommand*{\WorseLayoutDemo}[2][WORSE]{%
  \begin{figure}
    \begin{center}
      \setlength{\samplelayoutwidth}{0.45\textwidth}
      \UpdateSampleLayoutHeight{pics/#2-00000.tikz}
      \begin{tabular}{c@{\qquad}c}
        \InputTikzGraph[execute at begin picture = \AtBeginSampleLayout]{\samplelayoutwidth}{pics/#2-00000.tikz}&
        \InputTikzGraph[execute at begin picture = \AtBeginSampleLayout]{\samplelayoutwidth}{pics/#2-00500.tikz}\\[1ex]
        \(r=0\,\%\) & \(r=5\,\%\)\\[3ex]
        \InputTikzGraph[execute at begin picture = \AtBeginSampleLayout]{\samplelayoutwidth}{pics/#2-01000.tikz}&
        \InputTikzGraph[execute at begin picture = \AtBeginSampleLayout]{\samplelayoutwidth}{pics/#2-02000.tikz}\\[1ex]
        \(r=10\,\%\) & \(r=20\,\%\)\\[3ex]
        \InputTikzGraph[execute at begin picture = \AtBeginSampleLayout]{\samplelayoutwidth}{pics/#2-05000.tikz}&
        \InputTikzGraph[execute at begin picture = \AtBeginSampleLayout]{\samplelayoutwidth}{pics/#2-10000.tikz}\\[1ex]
        \(r=50\,\%\) & \(r=100\,\%\)
      \end{tabular}
    \end{center}
    \caption[Example \enum{#1} application]{%
      \enum{#1} application illustrated on a regular grid for increasing rates of degradation.
    }
    \label{fig:worsening-#2}
  \end{figure}
}

\newcommand*{\InterLayoutDemoPics}[2][INTER]{%
  \begin{center}
    \setlength{\samplelayoutwidth}{0.45\textwidth}
    \begin{tabular}{c@{\qquad}c}
      \InputTikzGraph{\samplelayoutwidth}{pics/#2-00000.tikz}&
      \InputTikzGraph{\samplelayoutwidth}{pics/#2-02000.tikz}\\[1ex]
      \(r=0\,\%\) & \(r=20\,\%\)\\[3ex]
      \InputTikzGraph{\samplelayoutwidth}{pics/#2-04000.tikz}&
      \InputTikzGraph{\samplelayoutwidth}{pics/#2-06000.tikz}\\[1ex]
      \(r=40\,\%\) & \(r=60\,\%\)\\[3ex]
      \InputTikzGraph{\samplelayoutwidth}{pics/#2-08000.tikz}&
      \InputTikzGraph{\samplelayoutwidth}{pics/#2-10000.tikz}\\[1ex]
      \(r=80\,\%\) & \(r=100\,\%\)
    \end{tabular}
  \end{center}
}

\newcommand*{\PropertyDemo}[2][]{%
  \begin{center}
    \setlength{\samplelayoutwidth}{0.25\textwidth}
    \begin{tabular}{c@{\qquad}c@{\qquad}c}
      \InputTikzGraph*{\samplelayoutwidth}{pics/demograph-a.tikz}&
      \InputTikzGraph*{\samplelayoutwidth}{pics/demograph-b.tikz}&
      \InputTikzGraph*{\samplelayoutwidth}{pics/demograph-c.tikz}\\[1ex]
      \InputLuatikzPlot[width=\samplelayoutwidth, #1]{pics/#2-a.pgf}&
      \InputLuatikzPlot[width=\samplelayoutwidth, #1]{pics/#2-b.pgf}&
      \InputLuatikzPlot[width=\samplelayoutwidth, #1]{pics/#2-c.pgf}
    \end{tabular}
  \end{center}
}

\newlength{\vertexsize}
\setlength{\vertexsize}{3pt}
\tikzset{
  Vertex/.style = {
    font = \footnotesize,
  },
  vertex/.style = {
    shape = circle,
    fill,
    inner sep = 0pt,
    minimum size = \vertexsize,
  },
  edge/.style = {
    draw,
    thin,
  },
  tikzgraphbbox/.style = {
    fill = \ifdraft{yellow!10}{none},
  },
  princomp/.style = {
    draw = white,
    double = black,
    line width = 1.0pt,
    double distance = 2.0pt,
    line cap = round,
    line join = round,
  },
  princomp1st/.style = { princomp },
  princomp2nd/.style = { princomp },
}

\SetKwBlock{Routine}{Routine}{End}
\SetKwComment{comment}{}{}
\SetKwInput{KwConstant}{Constants}
\SetKwProg{SubRoutine}{SubRoutine}{}{End}
\SetKw{KwAnd}{and}
\SetKw{KwBreak}{break}
\SetKw{KwOr}{or}
\SetKw{KwYield}{yield}

% https://tex.stackexchange.com/a/251936
\DefineBibliographyStrings{english}{%
  andothers = {\em et\addabbrvspace al\adddot}
}

\newlength{\xval@hunderter}
\newlength{\xval@hunderter@pm}
\newcommand*{\InputConfusionMatrix}[1]{%
  \begingroup%
    \setlength{\parindent}{\z@}%
    \settowidth{\xval@hunderter}{\ensuremath{100.00\,\%}}%
    \settowidth{\xval@hunderter@pm}{\ensuremath{\pm100.00\,\%}}%
    \newcommand*{\FormatMeanStdev}[2]}%
      \makebox[\xval@hunderter@pm][r]{\ensuremath{\pm\kern0.2em##2\,\%}}%
    }%
    \newcommand*{\FormatCell}[1]{\FormatMeanStdev{\csuse{XVal##1Mean}}{\csuse{XVal##1Stdev}}}%
    \newcommand*{\AltLine}[2]{\multicolumn{4}{l}{\textit{##1:\hfill##2}}}%
    \input{#1}%
    \begin{tabularx}{\linewidth}{X|rr|r}%
      \toprule%
                            & \textit{Cond.~Neg.}   & \textit{Cond.~Pos.}   & \textit{$\Sigma$}\\%
      \midrule%
      \textit{Pred.~Neg.}   & \FormatCell{TrueNeg}  & \FormatCell{FalseNeg} & \FormatCell{PredNeg}\\%
      \textit{Pred.~Pos.}   & \FormatCell{FalsePos} & \FormatCell{TruePos}  & \FormatCell{PredPos}\\%
      \midrule%
      \textit{$\Sigma$}     & \FormatCell{CondNeg}  & \FormatCell{CondPos}  & \FormatMeanStdev{100.00}{0.00}\\%
      \bottomrule
      \multicolumn{4}{l}{}\\
      \AltLine{Success Rate}{\FormatCell{Success}}\\%
      \AltLine{Failure Rate}{\FormatCell{Failure}}\\[1ex]%
      \AltLine{Average Number of Tests}{\ensuremath{\approx\XValCountApprox}}\\%
      \AltLine{Number of Repetitions}{\ensuremath{\XValTestRuns}}\\%
    \end{tabularx}%
  \endgroup%
}

\newcommand*{\InputPunctureResult}[2][\relax]{%
  \begingroup%
  \newcommand*{\PunctureResult}[5][\@undefined]&\ensuremath{##3\,\%}}&
    \ifblank{##4}{\multicolumn{2}{c}{}}{\ensuremath{##4\,\%}&\ensuremath{##5\,\%}}\\
  }%
  \input{#1}%
  \begin{tabularx}{\linewidth}{X@{\qquad}r@{\makebox[1.5em][r]{$\pm$}}rr@{\makebox[1.5em][r]{$\pm$}}r}%
    \toprule%
    \textit{Property} & \multicolumn{2}{r}{\textit{Sole Exclusion}} & \multicolumn{2}{r}{\textit{Sole Inclusion}}\\%
    \midrule%
    \input{#2}%
    \midrule%
    \PunctureResult[\textit{Baseline Using All Properties}]{\XValSuccessMean}{\XValSuccessStdev}{}{}%
    \bottomrule%
  \end{tabularx}%
  \endgroup%
}

\newcommand*{\InputLuatikzPlot}[2][]{\input{#2}}

\newcommand*{\regression}[3][f]{\ensuremath{#1(x)=#2+#3x}}

\makeatother


\providecommand*{\BonusSlide}{beamer}

\TitleImage[viewport = {303 303 703 703}]{pics/titlepicture.pdf}

% There is s not-so-smart but very restrictive "safety check" built into the KIT's beamer theme that should prevent you
% from making the secondary logo too large.  Well, I can see your point alright ...
\logo{\tikz[remember picture]{ \coordinate(secondarylogo); }}

\addbibresource{literature.bib}

\AtBeginDocument{
  \input{eval-cross-valid.tex}
  \input{disco.tex}
  \input{nn-info.tex}
}

\hypersetup{
  pdfauthor = {M. Klammler, T. Mchedlidze and A. Pak},
  pdftitle = {Aesthetic Discrimination of Graph Layouts},
  pdfsubject = {26th International Symposium on Graph Drawing and Network Visualization, Barcelona (2018)},
  pdfkeywords = {}
}

\title{Aesthetic Discrimination of Graph Layouts}

\subtitle{Moritz Klammler {\textand} Tamara Mchedlidze {\textand} Alexey Pak}

\institute[M.~Klammler {\textand} T.~Mchedlidze {\textand} A.~Pak]{%
  26\textsuperscript{th} International Symposium on Graph Drawing and Network Visualization, Barcelona (2018)
}

\date{September 2018}

\begin{document}

\begin{frame}
  \maketitle
  \begin{tikzpicture}[remember picture, overlay]
    \node[anchor = north east] at (secondarylogo) {\pgfimage[height = 7mm]{pics/iosb-logo.pdf}};
  \end{tikzpicture}
\end{frame}

\begin{frame}<\BonusSlide>
  \frametitle{\abstractname}
  \noindent\parbox{\textwidth}{%
    \footnotesize
    This paper addresses the following basic question: given two layouts of the same graph, which one is more
    aesthetically pleasing? We propose a neural network-based discriminator model trained on a labeled dataset that
    decides which of two layouts has a higher aesthetic quality. The feature vectors used as inputs to the model are
    based on known graph drawing quality metrics, classical statistics, information-theoretical quantities, and
    two-point statistics inspired by methods of condensed matter physics. The large corpus of layout pairs used for
    training and testing is constructed using force-directed drawing algorithms and the layouts that naturally stem from
    the process of graph generation. It is further extended using data augmentation techniques. Our model demonstrates a
    mean prediction accuracy of $\XValSuccessMean\percent$, outperforming discriminators based on stress and on the
    linear combination of popular quality metrics by a small but statistically significant margin.
  }
  \vfill
  \slidecite{Arxiv,MScThesis,GitHubRepo}
\end{frame}

\section{Problem Statement}

\begin{frame}
  \frametitle{Problem Statement}
  Given two \highlight{2}{vertex layouts} \(\Layout_a\) and \(\Layout_b\) for the same \highlight{3}{simple graph}
  \(\GraphGVE\).\newline
  Is \(\Layout_a\) or \(\Layout_b\) more aesthetically pleasing?
  \par\vspace{1pc}
  \only<4>{\ShowDemo}
  \DiscriminatorDemo[%
    discmod = \GetDiscriminatorResult{?}{2c0ab2fa}{dab7fdeb}{fb8d2845}{discmod},
  ]{pics/example-a.tikz}{pics/example-b.tikz}
  \begin{tikzpicture}[remember picture, overlay]
    \node<2>[smartbox, anchor = center, yshift = 2cm] at (current page.south) {%
      \begin{tabular}{lrcl}
        $\Layout:$ & $\GraphV$ & $\to$     & $\Reals^2$\\
                   & $v$       & $\mapsto$ & $(x_v,y_v)$\\
      \end{tabular}
    };
    \node<3>[smartbox, anchor = center, yshift = 2cm] at (current page.south) {%
      \begin{tabular}{@{\makebox[1.5em][l]{$\to$}}l}
        undirected\\
        no loops\\
        no multiple edges\\
      \end{tabular}
    };
  \end{tikzpicture}
\end{frame}

% Show agenda before each but the first section.
\AtBeginSection[\relax]{\frame{\frametitle{\contentsname}\tableofcontents[currentsection]}}

\section{Related Work}

\begin{frame}<1,2>[label = relatedwork]
  \frametitle{Related Work}
  \begin{itemize}
  \item<+-> Simple Metrics
    \begin{itemize}
    \item number of edge crossings
    \item minimum crossing angle (cross resolution)
    \item minimum angle between incident edges (angular resolution)
    \item standard deviation of edge lengths
    \item\dots
    \end{itemize}
    \bigskip
  \item<+-> % Combined Metric
    \( \menum{COMB}(\Gamma_i) = \sum_{M} w_M \: z_{M}(\Gamma_i) \)
    \qquad with \qquad
    \( z_M = ( M(\Gamma_i) - \mu_M ) \: / \: \sigma_M \)
    \bigskip
  \item<+-> % Stress
    \(
      \stress(\Gamma) = \sum_{i=1}^{n-1} \sum_{j=i+1}^{n} k_{ij}
      \Big( \dist_\Gamma(v_i,v_j) - L \cdot \dist_G(v_i,v_j) \Big)^2
    \)
  \end{itemize}
  \slidecite{Huang2013,Kamada1989}
\end{frame}

\begin{frame}
  \frametitle{Related Work}
  \framesubtitle{Combined Metric (\enum{COMB})}
  \ShowDemoScale\only<2>{\ShowDemoResult}
  \DiscriminatorDemo[%
    comb       = \GetDiscriminatorResult{+}{5b3b66d2}{a07125fe}{101708ec}{comb},
    no-discmod = \GetDiscriminatorResult{-}{5b3b66d2}{a07125fe}{101708ec}{discmod},
  ]{pics/5b3b66d2-a07125fe-more.tikz}{pics/5b3b66d2-101708ec-less.tikz}
\end{frame}

\againframe<2,3>{relatedwork}

\begin{frame}
  \frametitle{Related Work}
  \frametitle{Stress (\enum{STRESS})}
  \ShowDemoScale\only<2>{\ShowDemoResult}
  \DiscriminatorDemo[%
    stress     = \GetDiscriminatorResult{+}{8314f2c1}{0b878c80}{c7b63d0b}{stress},
    no-discmod = \GetDiscriminatorResult{-}{8314f2c1}{0b878c80}{c7b63d0b}{discmod},
  ]{pics/8314f2c1-0b878c80-more.tikz}{pics/8314f2c1-c7b63d0b-less.tikz}
\end{frame}

\section{Methodology}

\begin{frame}<-6>[label = outermethod]
  \frametitle{Methodology Overview}
  \only<6>{\newcommand*{\overspecial}{data}}
  \only<7>{\newcommand*{\overspecial}{featex}}
  \only<8>{\newcommand*{\overspecial}{discmod}}
  \ifdef{\overspecial}{%
    \ifdefstring{\overspecial}{data}{%
      \newcommand*{\ColorData}{KITyellow}
      \newcommand*{\ColorDataA}{KITyellow}
      \newcommand*{\ColorDataAs}{KITyellow}
      \newcommand*{\ColorDataB}{KITyellow}
      \newcommand*{\ColorDataBs}{KITyellow}
      \newcommand*{\TextColorData}{KITorange}
      \newcommand*{\TextColorDataA}{\ColorDataA}
      \newcommand*{\TextColorDataB}{\ColorDataB}
    }{%
      \newcommand*{\ColorData}{nospecialcolor}
      \newcommand*{\ColorDataA}{nospecialcolor}
      \newcommand*{\ColorDataAs}{nospecialcolor}
      \newcommand*{\ColorDataB}{nospecialcolor}
      \newcommand*{\ColorDataBs}{nospecialcolor}
      \newcommand*{\TextColorData}{nospecialcolor}
      \newcommand*{\TextColorDataA}{\ColorDataA}
      \newcommand*{\TextColorDataB}{\ColorDataB}
    }
    \ifdefstring{\overspecial}{featex}{%
      \newcommand*{\ColorFeatex}{KITyellow}
      \newcommand*{\ColorFeatexA}{KITyellow}
      \newcommand*{\ColorFeatexB}{KITyellow}
      \newcommand*{\TextColorFeatex}{KITorange}
      \newcommand*{\TextColorFeatexA}{\TextColorFeatex}
      \newcommand*{\TextColorFeatexB}{\TextColorFeatex}
    }{%
      \newcommand*{\ColorFeatex}{nospecialcolor}
      \newcommand*{\ColorFeatexA}{nospecialcolor}
      \newcommand*{\ColorFeatexB}{nospecialcolor}
      \newcommand*{\TextColorFeatex}{nospecialcolor}
      \newcommand*{\TextColorFeatexA}{\TextColorFeatex}
      \newcommand*{\TextColorFeatexB}{\TextColorFeatex}
    }
    \ifdefstring{\overspecial}{discmod}{%
      \newcommand*{\ColorDiscmod}{KITyellow}
      \newcommand*{\TextColorDiscmod}{KITorange}
    }{%
      \newcommand*{\ColorDiscmod}{nospecialcolor}
      \newcommand*{\TextColorDiscmod}{KITblack}
    }
  }{%
    \newcommand*{\ColorDataA}{goodlayoutcolor}
    \newcommand*{\ColorDataB}{badlayoutcolor}
    \alt<4->{
      \newcommand*{\ColorDataAs}{goodlayoutcolor}
      \newcommand*{\ColorDataBs}{badlayoutcolor}
    }{
      \newcommand*{\ColorDataAs}{white}
      \newcommand*{\ColorDataBs}{white}
    }
    \newcommand*{\TextColorData}{KITblack}
    \newcommand*{\TextColorDataA}{\ColorDataA}
    \newcommand*{\TextColorDataB}{\ColorDataB}
    \newcommand*{\ColorFeatex}{layoutcolor}
    \newcommand*{\ColorFeatexA}{goodlayoutcolor}
    \newcommand*{\ColorFeatexB}{badlayoutcolor}
    \newcommand*{\TextColorFeatexA}{\ColorFeatexA}
    \newcommand*{\TextColorFeatexB}{\ColorFeatexB}
    \newcommand*{\TextColorFeatex}{KITblack}
    \newcommand*{\ColorDiscmod}{layoutcolor}
    \newcommand*{\TextColorDiscmod}{KITblack}
  }
  \begin{tikzpicture}[inner sep = 2ex]
    \coordinate (origin) at (0, 0);
    \node (ex1more) at (-1, 0) {%
      \begin{tikzpicture}[graphcolor = \ColorDataA]
        \node[vertex] (v1) at (-0.5, -0.5) {};
        \node[vertex] (v2) at (-0.5, +0.5) {};
        \node[vertex] (v3) at (+0.5, -0.5) {};
        \node[vertex] (v4) at (+0.5, +0.5) {};
        \draw[edge] (v1) -- (v2) -- (v4) -- (v3) -- (v1);
      \end{tikzpicture}
    };
    \node (ex1less) at (+1, 0) {%
      \begin{tikzpicture}[graphcolor = \ColorDataB]
        \node[vertex] (v1) at (-0.5, -0.5) {};
        \node[vertex] (v2) at (-0.5, +0.5) {};
        \node[vertex] (v3) at (+0.5, +0.5) {};
        \node[vertex] (v4) at (+0.5, -0.5) {};
        \draw[edge] (v1) -- (v2) -- (v4) -- (v3) -- (v1);
      \end{tikzpicture}
    };
    \node<-3>[text = \TextColorDataA, anchor = base, below = 0ex of ex1more] {$\Gamma_a$};
    \node<-3>[text = \TextColorDataB, anchor = base, below = 0ex of ex1less] {$\Gamma_b$};
    \node (ex2more) at (-1, -2) {%
      \begin{tikzpicture}[graphcolor = \ColorDataAs]
        \node[vertex] (v1) at (330:7mm) {};
        \node[vertex] (v2) at (210:7mm) {};
        \node[vertex] (v3) at (90:7mm) {};
        \draw[edge] (v1) -- (v2) -- (v3) -- (v1);
      \end{tikzpicture}
    };
    \node (ex2less) at (+1, -2) {%
      \begin{tikzpicture}[graphcolor = \ColorDataBs]
        \node[vertex] (v1) at (330:7mm) {};
        \node[vertex] (v2) at (210:7mm) {};
        \node[vertex] (v3) at (10:9mm) {};
        \draw[edge] (v1) -- (v2) -- (v3) -- (v1);
      \end{tikzpicture}
    };
    \node[text = \ColorDataAs] (exdotsmore) at (-1, -3.5) {$\vdots$};
    \node[text = \ColorDataBs ] (exdotsless) at (+1, -3.5) {$\vdots$};
    \node (exnmore) at (-1, -5) {%
      \begin{tikzpicture}[graphcolor = \ColorDataAs]
        \node[vertex] (v1) at (-1.0,  0.0) {};
        \node[vertex] (v2) at (-0.5,  0.0) {};
        \node[vertex] (v3) at (+0.5,  0.0) {};
        \node[vertex] (v4) at (+1.0,  0.0) {};
        \node[vertex] (v5) at ( 0.0, -0.5) {};
        \node[vertex] (v6) at ( 0.0, +0.5) {};
        \draw[edge] (v2) -- (v5) -- (v3) -- (v6) -- (v2);
        \draw[edge] (v1) -- (v2);
        \draw[edge] (v3) -- (v4);
      \end{tikzpicture}
    };
    \node (exnless) at (+1, -5) {%
      \begin{tikzpicture}[graphcolor = \ColorDataBs]
        \node[vertex] (v1) at (+1.0, +0.5) {};
        \node[vertex] (v2) at (+0.5, +0.5) {};
        \node[vertex] (v3) at (+0.5, -0.5) {};
        \node[vertex] (v4) at (+1.0, -0.5) {};
        \node[vertex] (v5) at (-0.5, +0.5) {};
        \node[vertex] (v6) at (-0.5, -0.5) {};
        \draw[edge] (v2) -- (v5) -- (v3) -- (v6) -- (v2);
        \draw[edge] (v1) -- (v2);
        \draw[edge] (v3) -- (v4);
      \end{tikzpicture}
    };
    \coordinate (auxx) at (3, 0);
    \coordinate (aux1) at (auxx|-ex1less.north);
    \coordinate (auxdots) at (auxx|-exdotsless);
    \coordinate (auxn) at (auxx|-exnless.south);
    \coordinate (auxy) at ($(ex1more)!0.5!(ex1less)$);
    \coordinate[above = 7ex of auxy] (topaux);
    \coordinate[below = 2ex of auxy] (auxy1a);
    \coordinate[above = 2ex of auxy] (auxy1b);
    \node[anchor = base west, text = white] at (topaux) {Please take this into account for the Bounding Box!};
    %\draw<2>[dashed] (ex1more.south west) rectangle (ex1less.north east);
    \node<2->[draw = \ColorDiscmod, fill = \ColorDiscmod!50, text = \TextColorDiscmod, align = center, inner sep = 2ex]
             (model) at (7, -2) {Discriminator\\Model};
    \coordinate<2-> (modelina) at ($(model.north west)!0.5!(model.north)$);
    \coordinate<2-> (modelinb) at ($(model.north east)!0.5!(model.north)$);
    \coordinate<2->[below = 4ex of model] (submod);
    \draw<2->[\ColorDiscmod, <->] (model.west|-submod) -- (model.east|-submod);
    \draw<2->[\ColorDiscmod] (submod) +(0, -3pt) -- +(0, +3pt);
    \fill<2->[\ColorDiscmod] (submod) ++(-2em, 0) -- +(-1ex, 2ex) -- +(+1ex, 2ex) -- cycle;
    \coordinate<3>[right = 4.5cm of topaux, anchor = base] (featex);
    \node<3->[text = \TextColorFeatexA, inner sep = 5pt] (featexa) at (featex|-auxy1a) {$f(\Gamma_a)$};
    \node<3->[text = \TextColorFeatexB, inner sep = 5pt] (featexb) at (featex|-auxy1b) {$f(\Gamma_b)$};
    \draw<3->[->, shorten > = 1ex, very thick, \ColorFeatexA]
             (ex1less.east|-featexa) -- (featexa) -- (modelina|-featexa) -- (modelina);
    \draw<3->[->, shorten > = 1ex, very thick, \ColorFeatexB]
             (ex1less.east|-featexb) -- (featexb) -- (modelinb|-featexb) -- (modelinb);
    \node<3->[anchor = base, text = \TextColorFeatex] at (featex|-topaux) {Feature Extraction};
    \node<4->[anchor = base, text = \TextColorData] at (origin|-topaux) {Labeled Pairs};
    \draw<5>[<->, shorten > = 1ex, draw = \ColorDiscmod, thick, preaction = {-, draw = white, line width = 10pt}] (aux1)
     to node[below, rotate = 90, text = \ColorDiscmod]{training} (auxdots);
    \draw<5>[<->, shorten > = 1ex, draw = \ColorDiscmod, thick] (auxn)
     to node[below, rotate = 90, text = \ColorDiscmod]{testing} (auxdots);
  \end{tikzpicture}
\end{frame}

\begin{frame}<1>[label = databullets]
  \frametitle{Data Acquisition {\&} Augmentation}
  \begin{enumerate}
  \item<1-> Collect a large number of graphs
    \begin{itemize}
    \item\strong{imported graphs} --- from public collections
    \item\strong{generated graphs} --- using probabilistic algorithms
    \end{itemize}
    \par\bigskip
  \item<2-> Compute various layouts for these graphs
    \begin{itemize}
    \item\strong{native layouts} --- fall out of the graph generation process
    \item\strong{proper layouts} --- using common state-of-the-art algorithms
    \item\strong{garbage layouts} --- using more or less random placements of vertices
    \end{itemize}
    \par\bigskip
  \item<3-> Use data augmentation to expand this corpus
    \begin{itemize}
    \item\strong{layout worsening} --- degrade proper layout by given rate
    \item\strong{layout interpolation} --- compute layout \enquote{between} proper and garbage layout
    \end{itemize}
  \end{enumerate}
\end{frame}

\begin{frame}
  \frametitle{Data Acquisition {\&} Augmentation}
  \framesubtitle{Imported Graphs}
  \newcommand*{\RandomSeed}{42}
  \RandomThumbnailDemo{rome, north, randdag, bcspwr, grenoble, psadmit, smtape, import}
\end{frame}

\begin{frame}
  \frametitle{Data Acquisition {\&} Augmentation}
  \framesubtitle{Generated Graphs}
  \newcommand*{\RandomSeed}{100}
  \RandomThumbnailDemo%
      {lindenmayer, mosaic1, mosaic2, grid, torus1, torus2, quasi3d, quasi4d, quasi5d, quasi6d, bottle, tree}
\end{frame}

\againframe<1,2>{databullets}

\begin{frame}
  \frametitle{Data Acquisition {\&} Augmentation}
  \framesubtitle{Layouts}
  \begin{center}
    \small\noindent%
    \begin{tabular}{l@{\quad}c@{\qquad\qquad}c@{\qquad\qquad}c}
        \rotatebox{90}{\qquad\emph{proper}}
      & \InputTikzGraph[graphcolor = goodlayoutcolor]{0.2\textwidth}{pics/native.tikz}
      & \InputTikzGraph[graphcolor = goodlayoutcolor]{0.2\textwidth}{pics/fmmm.tikz}
      & \InputTikzGraph[graphcolor = goodlayoutcolor]{0.2\textwidth}{pics/stress.tikz}\\[1ex]
      & \enum{NATIVE} & \enum{FMMM} & \enum{STRESS}\\[2ex]
        \rotatebox{90}{\qquad\emph{garbage}}
      & \InputTikzGraph[graphcolor = badlayoutcolor]{0.2\textwidth}{pics/random-uniform.tikz}
      & \InputTikzGraph[graphcolor = badlayoutcolor]{0.2\textwidth}{pics/random-normal.tikz}
      & \InputTikzGraph[graphcolor = badlayoutcolor]{0.2\textwidth}{pics/phantom.tikz}\\[1ex]
      & \enum{RANDOM\_UNIFORM} & \enum{RANDOM\_NORMAL} & \enum{PHANTOM}
    \end{tabular}
  \end{center}
\end{frame}

\againframe<2,3>{databullets}

\begin{frame}
  \frametitle{Data Acquisition {\&} Augmentation}
  \framesubtitle{Layout Worsening}
  \begin{center}
    \InputTikzGraph{0.5\textwidth}{pics/worse-parent.tikz}
  \end{center}
\end{frame}

% For some obscure reason, beamer won't tolerate these definitions inside the frame environment so we have to make them
% global.  Yuck!

\newcommand*{\DisplayWorseLayout}[2][00000]{%
  \node at (#2|-r#1) {\InputTikzGraph{0.2\textwidth}{pics/#2-#1.tikz}};
}

\newcommand*{\DisplayWorseAlgo}[2][]{
  \node[anchor = base, above = 10mm of #1] {\enum{#2}};
}

\newcommand*{\DisplayWorseRate}[2][]{
  \node[anchor = base, xshift = -15mm, rotate = 90] at (r#1) {$r=#2\percent$};
}

\begin{frame}
  \frametitle{Data Acquisition {\&} Augmentation}
  \framesubtitle{Layout Worsening}
  \begin{center}
    \begin{tikzpicture}[scale = 1.05, font = \small]

      \coordinate (perturb)    at (0.0, 0);
      \coordinate (flip-nodes) at (2.5, 0);
      \coordinate (flip-edges) at (5.0, 0);
      \coordinate (movlsq)     at (7.5, 0);

      \coordinate (r01000) at (0, -0);
      \coordinate (r02000) at (0, -2);
      \coordinate (r05000) at (0, -4);

      \DisplayWorseAlgo[perturb]{PERTURB}
      \DisplayWorseAlgo[flip-nodes]{FLIP\_NODES}
      \DisplayWorseAlgo[flip-edges]{FLIP\_EDGES}
      \DisplayWorseAlgo[movlsq]{MOVLSQ}

      \DisplayWorseRate[01000]{10}
      \DisplayWorseRate[02000]{20}
      \DisplayWorseRate[05000]{50}

      \DisplayWorseLayout[01000]{perturb}
      \DisplayWorseLayout[01000]{flip-nodes}
      \DisplayWorseLayout[01000]{flip-edges}
      \DisplayWorseLayout[01000]{movlsq}

      \DisplayWorseLayout[02000]{perturb}
      \DisplayWorseLayout[02000]{flip-nodes}
      \DisplayWorseLayout[02000]{flip-edges}
      \DisplayWorseLayout[02000]{movlsq}

      \DisplayWorseLayout[05000]{perturb}
      \DisplayWorseLayout[05000]{flip-nodes}
      \DisplayWorseLayout[05000]{flip-edges}
      \DisplayWorseLayout[05000]{movlsq}

    \end{tikzpicture}
  \end{center}
\end{frame}

\begin{frame}
  \frametitle{Data Acquisition {\&} Augmentation}
  \framesubtitle{Layout Interpolation}
  \begin{center}
    \small\noindent%
    \begin{tabular}{c@{\qquad\qquad}c@{\qquad\qquad}c@{\qquad\qquad}c}
      \InputTikzGraph[graphcolor = badlayoutcolor!0!goodlayoutcolor]{0.2\textwidth}{pics/linear-00000.tikz}&
      \InputTikzGraph[graphcolor = badlayoutcolor!20!goodlayoutcolor]{0.2\textwidth}{pics/linear-02000.tikz}&
      \InputTikzGraph[graphcolor = badlayoutcolor!40!goodlayoutcolor]{0.2\textwidth}{pics/linear-04000.tikz}\\[1ex]
      $r=0\percent$ & $r=20\percent$ & $r=40\percent$\\[2ex]
      \InputTikzGraph[graphcolor = badlayoutcolor!60!goodlayoutcolor]{0.2\textwidth}{pics/linear-06000.tikz}&
      \InputTikzGraph[graphcolor = badlayoutcolor!80!goodlayoutcolor]{0.2\textwidth}{pics/linear-08000.tikz}&
      \InputTikzGraph[graphcolor = badlayoutcolor!100!goodlayoutcolor]{0.2\textwidth}{pics/linear-10000.tikz}\\[1ex]
      $r=60\percent$ & $r=80\percent$ & $r=100\percent$
    \end{tabular}
  \end{center}
\end{frame}

\againframe<7>{outermethod}

\begin{frame}<-4>[label = innermethod]
  \frametitle{Feature Extraction Overview}
  \begin{tikzpicture}
    [
      dataflow/.style = {
        semithick,
        KITpalegreen,
      },
      transformation/.style = {
        dataflow,
        decorate,
        decoration = {snake, amplitude = 1pt, segment length = 1em, post length = 2pt},
      },
    ]

    \node<-4> (layout) {\InputTikzGraph{3cm}{pics/psadmit.tikz}};

    \alt<1>{\colorlet{hackcolor}{white}}{\colorlet{hackcolor}{black}}
    \begin{scope}[node distance = 4ex, text = hackcolor]
      \node[right = 2cm of layout.north east, anchor = base west] (princomp1)  {\enum{PRINCOMP1}};
      \node[below = of princomp1.base west,   anchor = base west] (princomp2)  {\enum{PRINCOMP2}};
      \node[below = of princomp2.base west,   anchor = base west] (angular)    {\highlight[l]{4}{\enum{ANGULAR}}};
      \node[below = of angular.base west,     anchor = base west] (edgelength) {\enum{EDGE\_LENGTH}};
      \node[below = of edgelength.base west,  anchor = base west] (tension)    {\enum{TENSION}};
      \node[below = of tension.base west,     anchor = base west] (rdfglobal)  {\highlight[l]{4}{\enum{RDF\_GLOBAL}}};
      \node[below = of rdfglobal.base west,   anchor = base west] (rdflocal)   {\(\menum{RDF\_LOCAL}(d)\)};
      \node[fit = (princomp1) (princomp2) (angular) (edgelength) (tension) (rdfglobal) (rdflocal)] (properties) {};
    \end{scope}

    \begin{scope}
      \draw<2-4>[->, transformation] (layout) -- (princomp1.west);
      \draw<2-4>[->, transformation] (layout) -- (princomp2.west);
      \draw<2-4>[->, transformation] (layout) -- (angular.west);
      \draw<2-4>[->, transformation] (layout) -- (edgelength.west);
      \draw<2-4>[->, transformation] (layout) -- (tension.west);
      \draw<2-4>[->, transformation] (layout) -- (rdfglobal.west);
      \draw<2-4>[->, transformation] (layout) -- (rdflocal.west);
    \end{scope}

    \begin{scope}
      \coordinate<3->[right = 7em of princomp1.west] (aux1);
      \node<3->[anchor = base west] (princomp1data)  at (aux1|-princomp1.base)  {$\{x_1,x_2,\ldots\ldots\}$};
      \node<3->[anchor = base west] (princomp2data)  at (aux1|-princomp2.base)  {$\{x_1,x_2,\ldots\ldots\}$};
      \node<3->[anchor = base west] (angulardata)    at (aux1|-angular.base)    {$\{x_1,x_2,\ldots\}$};
      \node<3->[anchor = base west] (tensiondata)    at (aux1|-tension.base)    {$\{x_1,x_2,\ldots\}$};
      \node<3->[anchor = base west] (edgelengthdata) at (aux1|-edgelength.base) {$\{x_1,x_2,\ldots\}$};
      \node<3->[anchor = base west] (rdfglobaldata)  at (aux1|-rdfglobal.base)  {$\{x_1,x_2,\ldots\ldots\ldots\}$};
      \node<3->[anchor = base west] (rdflocaldata)   at (aux1|-rdflocal.base)   {$\{x_1,x_2,\ldots\ldots\ldots\}(d)$};
    \end{scope}

    \begin{scope}
      \coordinate<5->[right = 12em of princomp1data.west] (aux2);
      \node<5->[anchor = base west] (princomp1feat)  at (aux2|-princomp1data.base)  {$(y_1,y_2,y_3,y_4,y_5,y_6)$};
      \node<5->[anchor = base west] (princomp2feat)  at (aux2|-princomp2data.base)  {$(y_1,y_2,y_3,y_4,y_5,y_6)$};
      \node<5->[anchor = base west] (angularfeat)    at (aux2|-angulardata.base)    {$(y_1,y_2,y_3,y_4)$};
      \node<5->[anchor = base west] (tensionfeat)    at (aux2|-tensiondata.base)    {$(y_1,y_2,y_3,y_4)$};
      \node<5->[anchor = base west] (edgelengthfeat) at (aux2|-edgelengthdata.base) {$(y_1,y_2,y_3)$};
      \node<5->[anchor = base west] (rdfglobalfeat)  at (aux2|-rdfglobaldata.base)  {$(y_1,y_2,y_3,y_4)$};
      \node<5->[anchor = base west] (rdflocalfeat)   at (aux2|-rdflocaldata.base)   {$(y_1,y_2,y_3)(d)$};
    \end{scope}

    \begin{scope}
      \draw<5-6>[->, transformation] (princomp1data)  -- (princomp1feat);
      \draw<5-6>[->, transformation] (princomp2data)  -- (princomp2feat);
      \draw<5-6>[->, transformation] (angulardata)    -- (angularfeat);
      \draw<5-6>[->, transformation] (edgelengthdata) -- (edgelengthfeat);
      \draw<5-6>[->, transformation] (tensiondata)    -- (tensionfeat);
      \draw<5-6>[->, transformation] (rdfglobaldata)  -- (rdfglobalfeat);
      \draw<5-6>[->, transformation] (rdflocaldata)   -- (rdflocalfeat);
    \end{scope}

    \begin{scope}[overlay]
      \coordinate<6->[right = 11em of princomp1feat.west] (aux3);
      \node<6->[yshift = -1cm, align = center] (featvec) at (aux3|-rdflocal) {Feature\\Vector};
      \draw<6>[dataflow, ->] (aux3) -- (featvec);
      \draw<6>[dataflow] (princomp1feat)  -- (aux3|-princomp1feat);
      \draw<6>[dataflow] (princomp2feat)  -- (aux3|-princomp2feat);
      \draw<6>[dataflow] (angularfeat)    -- (aux3|-angularfeat);
      \draw<6>[dataflow] (edgelengthfeat) -- (aux3|-edgelengthfeat);
      \draw<6>[dataflow] (tensionfeat)    -- (aux3|-tensionfeat);
      \draw<6>[dataflow] (rdfglobalfeat)  -- (aux3|-rdfglobalfeat);
      \draw<6>[dataflow] (rdflocalfeat)   -- (aux3|-rdflocalfeat);
    \end{scope}

  \end{tikzpicture}
\end{frame}

\begin{frame}<\BonusSlide>
  \frametitle{Statistical Syndromes of Graph Layouts}
  \begin{itemize}
    \item\enum{PRINVEC1} and \enum{PRINVEC2} --- first and second principal axis of vertex coordinates
    \item\enum{PRINCOMP1} and \enum{PRINCOMP2} --- projections of vertex coordinates onto principal axes
    \item\enum{ANGULAR} --- angles between incident edges
    \item\enum{EDGE\_LENGTH} --- edge lengths
    \item\enum{TENSION} -- ratios of Euclidean and graph-theoretical distances computed for all vertex pairs
    \item\enum{RDF\_GLOBAL} --- pairwise distances between vertex coordinates
    \item\(\menum{RDF\_LOCAL}(d)\) --- pairwise distances between vertex coordinates where the graph-theoretical
      distance between them is bounded by $d\in\IntsN$
  \end{itemize}
  \par\vfill
  \begin{tikzpicture}[remember picture, overlay]
    \node<2>[smartbox, anchor = south, yshift = 1cm] at (current page.south) {%
      \begin{tabular}{l@{\makebox[2em][c]{$=$}}l}
        $\menum{RDF\_LOCAL}(\makebox[1em][c]{\ensuremath{1}})$      & \enum{EDGE\_LENGTH}\\[1ex]
        $\menum{RDF\_LOCAL}(\makebox[1em][c]{\ensuremath{\infty}})$ & \enum{RDF\_GLOBAL}\\
      \end{tabular}
    };
  \end{tikzpicture}
\end{frame}

\PropertyDemo[ANGULAR]{angular}{Angles Between Incident Edges}
\PropertyDemo[RDF\_GLOBAL]{rdf-global}{Radial Distribution Function}

\againframe<5->{innermethod}

\begin{frame}
  \frametitle{Feature Extraction}
  We need to condense syndromes into a feature vector of fixed size.
  \par\bigskip
  \begin{itemize}
  \item arithmetic mean
  \item root mean squared (RMS)
  \end{itemize}
  \begin{itemize}
  \item<4-> entropy of distribution
    \begin{itemize}
    \item Problem: depends on data aggregation (histogram bin / filter width)
    \item $\to$ Compute entropy for several histograms with different bin widths
    \item $\to$ Perform linear regression
    \item $\to$ Use regression parameters instead of entropy
    \end{itemize}
  \end{itemize}
  \begin{tikzpicture}
    [
      remember picture,
      overlay,
      node distance = 1pc,
    ]
    \node<2>[thumbnail, anchor = center] at (current page.center) {%
      \begin{tabular}{c@{\qquad}c}
        \InputTikzGraph{30mm}{pics/demograph-a.tikz} & \InputTikzGraph{30mm}{pics/demograph-b.tikz}\\[3ex]
        \InputLuatikzPlot{pics/angular-a.pgf}        & \InputLuatikzPlot{pics/angular-b.pgf}
      \end{tabular}
    };
    \node<5>[thumbnail, anchor = center] at (current page.center) {\InputLuatikzPlot{pics/entropy-regression.pgf}};
  \end{tikzpicture}
\end{frame}

\againframe<8>{outermethod}

\begin{frame}
  \frametitle{Siamese Neural Network}
  \begin{center}
    \newcommand*{\ScaleFactor}{0.9}%
    \providecommand*{\SubfigLabelSM}{}%
    \providecommand*{\SubfigLabelGM}{}%
    \newcommand*{\PrimaryColor}{KITblue}%
    \newcommand*{\SecondaryColor}{KITblue}%
    % -*- coding:utf-8; mode:latex; -*-

% Copyright (C) 2018 Karlsruhe Institute of Technology
% Copyright (C) 2018 Moritz Klammler <moritz.klammler@alumni.kit.edu>
%
% Copying and distribution of this file, with or without modification, are permitted in any medium without royalty
% provided the copyright notice and this notice are preserved.  This file is offered as-is, without any warranty.

\begin{tikzpicture}
  [
    layer/.style = {
      rectangle,
      draw,
      minimum width = 15mm,
      minimum height = 4mm,
      text depth = 0.5ex,
      text height = 1.5ex,
      font = \scriptsize,
    },
    dense/.style = {
      layer,
      fill = TangoSkyBlue1!20,
    },
    operator/.style = {
      circle,
      draw,
      minimum size = 4mm,
      font = \scriptsize,
    },
    signal/.style = {
      draw = TangoSkyBlue3,
      thick,
      ->,
    },
    sigwd/.style = {
      font = \scriptsize,
      text = TangoSkyBlue3,
    },
    sigwdmark/.style = {
      draw = TangoSkyBlue3,
      very thin,
    },
    node distance = 5mm,
  ]

  \newcommand*{\NNUnknownDims}{?}
  \providecommand*{\NNSharedInputDims}{\NNUnknownDims}
  \providecommand*{\NNSharedHiddenDims}{\NNUnknownDims}
  \providecommand*{\NNSharedOutputDims}{\NNUnknownDims}
  \providecommand*{\NNTotalAuxInputDims}{\NNUnknownDims}
  \providecommand*{\NNTotalAuxHiddenDims}{\NNUnknownDims}
  \providecommand*{\NNTotalCatDims}{\NNUnknownDims}
  \providecommand*{\NNTotalOutputDims}{\NNUnknownDims}

  \coordinate (SM) at (0, 0);
  \coordinate (GM) at (7.5, -1);
  \coordinate (label) at (0, -2.25);

  \begin{scope}[shift = (SM), x = 13mm]

    \coordinate (in) at (0.25, 0);
    \coordinate (mid) at (4.75, 0);
    \node[layer, rotate = 90] (do1) at (1, 0) {dropout};
    \node[dense, rotate = 90] (l1) at (2, 0) {dense};
    \node[layer, rotate = 90] (do2) at (3, 0) {dropout};
    \node[dense, rotate = 90] (l2) at (4, 0) {dense};

    \draw[signal] (in) to node[name = a]{} (do1);
    \draw[signal] (do1) to node[name = b]{} (l1);
    \draw[signal] (l1) to node[name = c]{} (do2);
    \draw[signal] (do2) to node[name = d]{} (l2);
    \draw[signal] (l2) to node[name = e]{} (mid);

    \foreach \pos in {a, b, c, d, e} {
      \draw[sigwdmark] (\pos) +(-1pt, -2pt) -- +(+1pt, +2pt);
    }

    \node[sigwd, above = 0ex of a] {\ensuremath{\NNSharedInputDims}};
    \node[sigwd, above = 0ex of b] {\ensuremath{\NNSharedInputDims}};
    \node[sigwd, above = 0ex of c] {\ensuremath{\NNSharedHiddenDims}};
    \node[sigwd, above = 0ex of d] {\ensuremath{\NNSharedHiddenDims}};
    \node[sigwd, above = 0ex of e] {\ensuremath{\NNSharedOutputDims}};

    \node[anchor = east] at (in) {$\vec{v}$};
    \node[anchor = west] at (mid) {$\vec{\sigma}$};

    \coordinate (smlabel) at ($(in)!0.5!(mid)$);

  \end{scope}

  \begin{scope}[shift = (GM), x = 13mm]

    \coordinate (delta) at (0.75, 1.5);
    \coordinate (auxin) at (-0.25, 0);
    \coordinate (out) at (3.25, 1);

    \node[layer, rotate = 90, minimum width = 35mm] (cat) at (1.5, 1) {concatenation};
    \node[dense, rotate = 90] (aux) at (0.5, 0) {dense};
    \node[dense, rotate = 90] (l3) at (2.5, 1) {dense};

    \draw[signal] (delta) to node[name = f]{} (cat.north|-delta);
    \draw[signal] (auxin) to node[name = g]{} (aux);
    \draw[signal] (aux) to node[name = h]{} (cat.north|-aux);
    \draw[signal] (cat) to node[name = i]{} (l3);
    \draw[signal] (l3) to node[name = j]{} (out);

    \foreach \pos in {f, g, h, i, j} {
      \draw[sigwdmark] (\pos) +(-1pt, -2pt) -- +(+1pt, +2pt);
    }

    \node[sigwd, above = 0ex of f] {\ensuremath{\NNSharedOutputDims}};
    \node[sigwd, above = 0ex of g] {\ensuremath{\NNTotalAuxInputDims}};
    \node[sigwd, above = 0ex of h] {\ensuremath{\NNTotalAuxHiddenDims}};
    \node[sigwd, above = 0ex of i] {\ensuremath{\NNTotalCatDims}};
    \node[sigwd, above = 0ex of j] {\ensuremath{\NNTotalOutputDims}};

    \node[anchor = east] at (delta) {$\vec{\sigma}_a-\vec{\sigma}_b$};
    \node[anchor = east] at (auxin) {$\vec{v}_G$};
    \node[anchor = west] at (out) {$t$};

    \coordinate (gmlabel) at ($(auxin)!0.5!(out)$);

  \end{scope}

  \node at (smlabel|-label) {(a)};
  \node at (gmlabel|-label) {(b)};

\end{tikzpicture}

  \end{center}
  \slidecite{Bromley1994}
\end{frame}

\section{Evaluation}

\begin{frame}
  \frametitle{Comparison With Other Metrics}
  \InputCompetingMetrics{eval-competing-metrics.tex}
  \par\vfill
  \begin{itemize}
  \item 10-fold Cross validation via random sub-sampling
  \item \enum{STRESS} was compared for best scale
  \item \enum{COMB} weights were fitted to training data set
  \end{itemize}
  \slidecite{Welch2017,Huang2013}
\end{frame}

\begin{frame}
  \frametitle{Comparison With Other Metrics}
  \ShowDemo\DiscriminatorDemo[%
    discmod = \GetDiscriminatorResult{-}{d8c1498f}{4c326179}{234952e3}{discmod},
    comb    = \GetDiscriminatorResult{+}{d8c1498f}{4c326179}{234952e3}{comb},
    stress  = \GetDiscriminatorResult{+}{d8c1498f}{4c326179}{234952e3}{stress},
  ]{pics/d8c1498f-4c326179-more.tikz}{pics/d8c1498f-234952e3-less.tikz}
\end{frame}

\begin{frame}<\BonusSlide>
  \frametitle{Significance of Individual Syndromes}
  \begin{center}
    \InputPunctureResult[eval-cross-valid.tex]{eval-puncture.tex}
  \end{center}
  \par\vfill
  \begin{itemize}
  \item Note that \enum{RDF\_LOCAL} refers to a whole set of syndromes.
  \end{itemize}
\end{frame}

\section{Conclusion and Future Work}

\begin{frame}
  \frametitle{Conclusion and Future Work}
  \begin{itemize}
  \item Binary discrimination instead of absolute aesthetic measure
  \item Avoid a priori assumptions about influence on aesthetics
  \item Use of statistical syndromes inspired by Statistical Physics and Crystallography
  \item Data driven approach (machine learning)
  \item Accuracy usually outperforms other metrics
  \item \citeurl{GitHubRepo}
  \end{itemize}
  \par\bigskip
  \begin{itemize}
  \item Identification of necessary and sufficient syndromes
  \item Optimization of the neural network
  \item Validation against human-labeled data
  \end{itemize}
\end{frame}

\section*{\bibname}

\ResetBibliographyStyle
\begin{frame}
  \frametitle{\bibname}
  \printbibliography
\end{frame}

\end{document}
