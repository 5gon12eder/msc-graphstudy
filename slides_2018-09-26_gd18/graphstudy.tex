% -*- coding:utf-8; mode:latex; -*-

% Copyright (C) 2018 Moritz Klammler <moritz.klammler@alumni.kit.edu>
% Copyright (C) 2018 Tamara Mchedlidze <mched@iti.uka.de>
% Copyright (C) 2018 Alexey Pak <alexey.pak@iosb.fraunhofer.de>
%
% This work is licensed under a Creative Commons Attribution-NonCommercial-NoDerivatives 4.0 International License
% (https://creativecommons.org/licenses/by-nc-nd/4.0/).

\documentclass{beamer}

% -*- coding:utf-8; mode:latex; -*-

% Copyright (C) 2018 Karlsruhe Institute of Technology
% Copyright (C) 2018 Moritz Klammler <moritz.klammler@alumni.kit.edu>
%
% Copying and distribution of this file, with or without modification, are permitted in any medium without royalty
% provided the copyright notice and this notice are preserved.  This file is offered as-is, without any warranty.

\makeatletter

\usetheme{KIT}

\usepackage{fontspec}
\usepackage{polyglossia}
\usepackage{amsmath,amsfonts}
%\usepackage{slidecite}
\usepackage{csquotes}
\usepackage{ifdraft}
\usepackage{etoolbox}
\usepackage{tabularx}
\usepackage{booktabs}
\usepackage{multicol}
\usepackage{gnuplot-lua-tikz}
\usetikzlibrary{positioning}
\usepackage{libertine}
\setmainlanguage{english}

\usepackage[%
  backend = biber,
  citestyle = authoryear,
  bibstyle = chem-acs,
  alldates = iso8601,
  giveninits = true,
]{biblatex}

\usenavigationsymbols
\setcounter{tocdepth}{2}

\setbeamercolor*{bibliography entry author}{fg=KITblack}
\setbeamercolor*{bibliography entry title}{fg=KITblack}
\setbeamercolor*{bibliography entry journal}{fg=KITblack}
\setbeamercolor*{bibliography entry note}{fg=KITblack}

\setbeamercolor{structure}{fg=KITgreen}
%\setbeamercolor*{palette secondary}{fg=KITgreen}
%\setbeamercolor*{palette tertiary}{fg=KITgreen}

\AtBeginSection[]

\makeatother

%\KITtitleimage[width = 0.95\paperwidth]{pics/titlepicture.png}
%\logo{\pgfimage[height = 1.5\KITlogoht]{pics/algo-logo.pdf}}

\addbibresource{literature.bib}

\hypersetup{
  pdfauthor = {M. Klammler, T. Mchedlidze and A. Pak},
  pdftitle = {Aesthetic Discrimination of Graph Layouts},
  pdfsubject = {26th International Symposium on Graph Drawing and Network Visualization},
  pdfkeywords = {}
}

\title{Aesthetic Discrimination of Graph Layouts}
\subtitle{26\textsuperscript{th} International Symposium on Graph Drawing and Network Visualization}
\author{M.\,Klammler, T.\,Mchedlidze and A.\,Pak}
\institute[KIT {\textperiodcentered} Fraunhofer]{%
  Karlsruhe Institute of Technology in cooperation with
  Fraunhofer Institute of Optronics, System Technologies and Image Exploitation
}

\date{September 2018}

\providecommand*{\XValSuccessMean}{\mbox{\textbf{??}}}

\begin{document}

\begin{frame}
  \maketitle
\end{frame}

\begin{frame}
  \frametitle{\abstractname}
  \noindent\parbox{\textwidth}{%
    \footnotesize
    This paper addresses the following basic question: given two layouts of the same graph, which one is more
    aesthetically pleasing? We propose a neural network-based discriminator model trained on a labeled dataset that
    decides which of two layouts has a higher aesthetic quality. The feature vectors used as inputs to the model are
    based on known graph drawing quality metrics, classical statistics, information-theoretical quantities, and
    two-point statistics inspired by methods of condensed matter physics. The large corpus of layout pairs used for
    training and testing is constructed using force-directed drawing algorithms and the layouts that naturally stem from
    the process of graph generation. It is further extended using data augmentation techniques. Our model demonstrates a
    mean prediction accuracy of $\XValSuccessMean\percent$, outperforming discriminators based on stress and on the
    linear combination of popular quality metrics by a small but statistically significant margin.
  }
\end{frame}

\section{Motivation}
\section{Related Work}
\section{Our Contribution}
\section{Evaluation}
\section{Conclusion and Future Work}

\nocite{MScThesis}
\nocite{Arxiv}
\nocite{GitHubRepo}

\section{\bibname}
\begin{frame}[allowframebreaks]
  \frametitle{\bibname}
  \printbibliography
  \par\bigskip
  \emph{Please refer to the paper for a complete list of references.}
\end{frame}

\end{document}
