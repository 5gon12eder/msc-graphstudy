% -*- coding:utf-8; mode:latex; -*- %

% Copyright (C) 2018 Moritz Klammler <moritz.klammler@student.kit.edu>
%
% Copying and distribution of this file, with or without modification, are permitted in any medium without royalty
% provided the copyright notice and this notice are preserved.  This file is offered as-is, without any warranty.

\begin{tikzpicture}[scale = 0.5]

  \path[use as bounding box] (-4.5, -4.5) rectangle (4.5, 4.5);

  \node[Vertex] (v)  at (0, 0)  {$v$};
  \node[Vertex] (u1) at (  0:4) {$u_1$};
  \node[Vertex] (u2) at ( 90:4) {$u_2$};
  \node[Vertex] (u3) at (180:4) {$u_3$};
  \node[Vertex] (u4) at (270:4) {$u_4$};

  \node[vertex] (w1)  at (  0:2) {};
  \node[vertex] (w2)  at ( 30:2) {};
  \node[vertex] (w3)  at ( 60:2) {};
  \node[vertex] (w4)  at ( 90:2) {};
  \node[vertex] (w5)  at (120:2) {};
  \node[vertex] (w6)  at (150:2) {};
  \node[vertex] (w7)  at (180:2) {};
  \node[vertex] (w8)  at (210:2) {};
  \node[vertex] (w9)  at (240:2) {};
  \node[vertex] (w10) at (270:2) {};
  \node[vertex] (w11) at (300:2) {};
  \node[vertex] (w12) at (330:2) {};

  \foreach \i in {1, ..., 12} {
    \draw[edge] (v) -- (w\i);
  }

  \draw[edge] (w1)  -- (u1);
  \draw[edge] (w4)  -- (u2);
  \draw[edge] (w7)  -- (u3);
  \draw[edge] (w10) -- (u4);

\end{tikzpicture}
