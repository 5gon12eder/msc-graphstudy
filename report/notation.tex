% -*- coding:utf-8; mode:latex; -*- %

% Copyright (C) 2018 Moritz Klammler <moritz.klammler@student.kit.edu>
%
% This work is licensed under a Creative Commons Attribution-NonCommercial-NoDerivatives 4.0 International License
% (https://creativecommons.org/licenses/by-nc-nd/4.0/).

\addchap{Notation}

%% Vectors, matrices, sets and other non-scalar quantities are typeset as bold symbols like \(\vec{p}\) or \(\Vec{V}\)
%% where lower-case symbols are preferred for vectors and upper-case symbols are preferred for matrices and sets.  For
%% algorithms, we use upper-case calligraphic symbols such as \(\mathcal{A}\).

We write \(\IntsZ\) for the set of integers (positive, negative and zero alike), \(\IntsN=\{1,2,\ldots\}\) for the set
of positive (\acs{ad}~natural) integers and \(\IntsNz=\{0,1,2,\ldots\}\) for the set of non-negative integers (including
zero).  The set of real numbers is written as \(\Reals\) with an optional subscript to limit the range.  For example,
\(\RealsPos\) refers to the positive and \(\RealsNN\) to the non-negative real numbers.

For a set \(\Vec{X}\) and integer \(k\in\IntsN\) the expression \(\Vec{X}^k\) denotes the \(k\)-fold Cartesian product
\(\Vec{X}\times\cdots\times\Vec{X}\) or, in other words, all \(k\)-tuples \((x_1,\ldots,x_k)\) with \(x_i\in\Vec{X}\)
for \(i\in\{1,\ldots,k\}\).  The expression \(\Vec{X}^+\) denotes the union over all \(\Vec{X}^n\) for positive
\(n\in\IntsN\) and \(\Vec{X}^*\) the union over all \(\Vec{X}^n\) for non-negative \(n\in\IntsNz\), which includes the
empty tuple.

We use the notation \(\{\ldots\}\) for ordinary sets and \(\multiset{\ldots}\) for multisets.  (Unlike an ordinary set,
a multiset may contain duplicate elements.)  For a finite (multi-)set \(\Vec{S}\) we write \(\card{\Vec{S}}\) to denote
the number of elements in \(\Vec{S}\).  The symbol \enquote{\(\emptyset\)} refers to the empty (multi-)set.

We use the infix operators \enquote{\(\conjunction\)} and \enquote{\(\disjunction\)} for the conjunction (intersection)
and disjunction (union) of sets and likewise \enquote{\(\land\)} and \enquote{\(\lor\)} for the conjunction (AND) and
disjunction (OR) of logical values respectively.\footnote{%
  Both uses are Boolean algebra but people seem to prefer different symbols anyway.
}
The infix operator \enquote{\(\setminus\)} is used for differences of sets, therefore the expression
\(\Vec{A}\setminus\Vec{B}\) refers to all elements that are in set \(\Vec{A}\) but are not in set \(\Vec{B}\).  Logical
negation is written using the prefix operator \enquote{\(\lnot\)}.  The expression \(\lnot(x=y)\) is a convoluted way to
write \({x}\neq{y}\).  Finally, we use the symbol \enquote{\(\implies\)} to denote implication (in either direction) and
\enquote{\(\equivalent\)} to denote equivalence.  \({P}\implies{Q}\) means \enquote{\(P\) is a sufficient condition for
  \(Q\)} and is equivalent to \(\lnot{P}\lor{Q}\) whereas \({P}\equivalent{Q}\) means \enquote{\(P\) is a necessary
  \emph{and} sufficient condition for \(Q\)} and is equivalent to \(({P}\implies{Q})\land({P}\impliedby{Q})\).

The quantors \enquote{\(\forall\)} and \enquote{\(\exists\)} are to be read as \enquote{for all} and \enquote{there
  exists} respectively.  When applied to the empty set, the former is always true while the latter is always false.  We
use a colon to mean \enquote{such that}.  For example, the expression
\(\{n\in\IntsNz\suchthat\exists{}m\in\IntsNz\suchthat{}n=m^2\}\) is a somewhat overly complicated way to define the set
of square integers\footfullcite{OEISA000290} \(\{0,1,4,9,16,25,\ldots\}\).

The expression \(\argmax_{x\in\Reals}\{f(x)\}\) is to be understood as that \(x\in\Reals\) for which \(f(x)\) is a
maximum.  \(\argmin\) is defined analogously.  For example, \(\argmin_{0\leq{x}\leq2\pi}\{\cos(x)\}=\pi\) because
\(\cos(\pi)=\min_{0\leq{x}\leq2\pi}\{\cos(x)\}=-1\).

For \(x\in\Reals\) we write \(\abs{x}\) for the magnitude of \(x\).  The expressions \(\floor{x}\) (floor) and
\(\ceil{x}\) (ceiling) denote the largest / smallest integer that is not greater / less than \(x\) respectively.  The
expression \(\nint{x}\) denotes the \emph{nearest} integer to \(x\) which is commonly defined as \enquote{round to
  even}\footfullcite{MathWorldNinit}.  The function \(\sign\) (signum) is defined as \(\sign(x)=+1\) if \(x>0\) or
\(\sign(x)=-1\) if \(x<0\) or \(\sign(x)=0\) if \(x=0\).

Occasionally, we will write informal expressions like \(0<\epsilon\ll1\) which are to be read as \enquote{\(\epsilon\)
  is greater than zero but \emph{much less} than one} where it is intentionally left unspecified what \enquote{much
  less} means exactly.

We use bra/ket-notation~\cite{Dirac1939} for vector products.  For a Hilbert space \(\Vec{V}\) and vector
\(\vec{v}\in\Vec{V}\) the expression \(\bra{\vec{v}}\) denotes \(\vec{v}\) itself while \(\ket{\vec{v}}\) is the
(conjugate) transpose of \(\vec{v}\).  Therefore, the expression \(\braket{\vec{u}}{\vec{v}}\) refers to the inner
(scalar) and the expression \(\ket{\vec{u}}\bra{\vec{v}}\) to the outer (tensor) product of the vectors \(\vec{u}\) and
\(\vec{v}\).  Finally, we write \(\vecnorm{\vec{v}}\) to denote the vector norm \(\sqrt{\braket{\vec{v}}{\vec{v}}}\) of
\(\vec{v}\).

For an undirected graph \(\GraphGVE\) and \(v\in\GraphV\) the expression \(\deg(v)\) refers to the \emph{degree}
(number of incident edges) of vertex \(v\).

We use the symbol \enquote{\(\bot\)} to denote missing or undefined values.

We write \(x\gets{}f(42)\) to denote the assignment of the value of the expression \enquote{\(f(42)\)} to the variable
\(x\) in algorithm listings while trying to avoid variabe reassignment as much as possible.

To describe the asymptotic complexity of algorithms, we use the common \enquote{\(\BigO\)} notation.  For function
\(f:\IntsN\to\Reals\) the Landau symbol \(\BigO(f)\) refers to the set of all functions \(g:\IntsN\to\Reals\) for which
there exist constants \(n_0\in\IntsN\) and \(c\in\Reals\) such that \(\abs{g(n)/c}\leq{}f(n)\) for all \(n\geq{}n_0\).
That is to say, \(g\) is \emph{asymptotically dominated} by \(f\).  Like most authors, we usually cannot be bothered to
write the \enquote{argument} of the Landau symbol as a function as in \(\BigO(x\mapsto{}x^2)\) as it would be correct
and will simply write \(\BigO(x^2)\) instead.
