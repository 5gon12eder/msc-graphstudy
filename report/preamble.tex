% -*- coding:utf-8; mode:latex; -*- %

% Copyright (C) 2018 Moritz Klammler <moritz.klammler@student.kit.edu>
%
% Copying and distribution of this file, with or without modification, are permitted in any medium without royalty
% provided the copyright notice and this notice are preserved.  This file is offered as-is, without any warranty.

\makeatletter

\usepackage{booktabs}
\usepackage{tabularx}
\usepackage{gnuplot-lua-tikz}

\newcommand*{\email}[1]{\mbox{\texttt{\href{mailto:#1}{#1}}}}

\newcommand*{\Reals}[1][]{\ifblank{#1}{\mathbb{R}}{\mathbb{R}_{#1}}}
\newcommand*{\RealsPos}{\Reals[>0]}
\newcommand*{\RealsNN}{\Reals[\geq0]}
\newcommand*{\IntsZ}{\mathbb{Z}}
\newcommand*{\IntsN}{\mathbb{N}}
\newcommand*{\IntsNz}{\mathbb{N}_0}
\newcommand*{\BitZero}{\text{\texttt{0}}}
\newcommand*{\BitOne}{\text{\texttt{1}}}
\newcommand*{\Bits}{\{\BitZero,\BitOne\}}

\let\vec\relax
\let\Vec\relax
\let\implies\Rightarrow
\let\impliedby\Leftarrow
\let\equivalent\Leftrightarrow
\newcommand*{\suchthat}{:}
\newcommand*{\degree}{^\circ}
\let\conjunction\cap
\let\disjunction\cup
\newcommand*{\diff}[1]{\mathrm{d}#1\kern0.3em}

\newcommand*{\Layout}{\Gamma}
\newcommand*{\Graph}{G}
\newcommand*{\GraphV}{V}
\newcommand*{\GraphE}{E}
\newcommand*{\vecz}{0}

\newcommand*{\GraphGVE}{\Graph=(\GraphV,\GraphE)}

\newcommand*{\CMake}{\mbox{CMake}}
\newcommand*{\CXX}{\mbox{C++}}
\newcommand*{\JavaScript}{\mbox{JavaScript}}
\newcommand*{\Keras}{\mbox{Keras}}
\newcommand*{\Python}{\mbox{Python}}
\newcommand*{\SQLite}{\mbox{SQLite}}
\newcommand*{\TensorFlow}{\mbox{TensorFlow}}

\newcommand*{\Entropy}{\@ifstar\EntropyDifferential\EntropyDiscrete}
\newcommand*{\EntropyDiscrete}{S}
\newcommand*{\EntropyDifferential}{\bar{S}}

\DeclareMathOperator{\BigO}{\mathcal{O}}
\DeclareMathOperator{\dist}{dist}
\DeclareMathOperator{\edgelen}{length}
\DeclareMathOperator{\mean}{mean}
\DeclareMathOperator{\rms}{rms}
\DeclareMathOperator{\sign}{sign}
\DeclareMathOperator{\stdevp}{stdevp}
\DeclareMathOperator{\stdev}{stdev}
\DeclareMathOperator{\stress}{stress}

\DeclareMathOperator*{\Conjunction}{\bigcap}
\DeclareMathOperator*{\Disjunction}{\bigcup}
\DeclareMathOperator*{\argmax}{arg\,max}
\DeclareMathOperator*{\argmin}{arg\,min}

\newcommand*{\abs}[1]{\left\lvert#1\right\rvert}
\newcommand*{\card}[1]{\left\lvert#1\right\rvert}
\newcommand*{\vecnorm}[1]{\left\lVert#1\right\rVert}
\newcommand*{\floor}[1]{\left\lfloor#1\right\rfloor}
\newcommand*{\ceil}[1]{\left\lceil#1\right\rceil}
\newcommand*{\nint}[1]{\left\lfloor#1\right\rceil}
\newcommand*{\multiset}[1]{\left[#1\right]}

\newcommand{\bra}[2][\relax]{%
  \ifx\relax#1\relax%
    \left\langle#2\right|%
  \else%
    \left\langle#2\middle|#1\right|%
  \fi%
}

\newcommand{\ket}[2][\relax]{%
  \ifx\relax#1\relax%
    \left|#2\right\rangle%
  \else%
    \left|#1\middle|#2\right\rangle%
  \fi%
}

\newcommand{\braket}[3][\relax]{%
  \ifx\relax#1\relax%
    \left\langle#2\middle|#3\right\rangle%
  \else%
    \left\langle#2\middle|#1\middle|#3\right\rangle%
  \fi%
}

\newcommand*{\enum}[1]{\mbox{\upshape\ttfamily#1}}
\newcommand*{\menum}[1]{\mbox{\text{\upshape\ttfamily#1}}}
\newcommand*{\QuasiNd}{\enum{QUASI\(\langle{n}\rangle\)D}}
\newcommand*{\TorusN}{\enum{TORUS\(\langle{n}\rangle\)}}
\newcommand*{\code}[1]{\mbox{\texttt{#1}}}

\renewcommand*{\acroenparen}[1]{(\emph{#1})}

\newcommand*{\latintext}[1]{\emph{\textlatin{#1}}}

\newcommand*{\InputTikz}[2][]{\tikz[#1]{\input{#2}}}

\ifdraft{%
  \newcommand*{\InputTikzGraph}{\@ifstar\InputTikzGraph@Dummy\InputTikzGraph@Dummy}
}{%
  \newcommand*{\InputTikzGraph}{\@ifstar\InputTikzGraph@Big\InputTikzGraph@NotBig}
}

\newcommand*{\InputTikzGraph@NotBig}[3][]{\tikz[x=#2, y=#2, #1]{\input{#3}}}
\newcommand*{\InputTikzGraph@Big}[3][]{\tikz[x=#2, y=#2, #1]{\setlength{\vertexsize}{2pt}\input{#3}}}

\newcommand*{\InputTikzGraph@Dummy}[3][]{%
  \urldef\inputtikzgraph@url\url{#3}%
  \begin{tikzpicture}[x=#2, y=#2, #1]
    \booltrue{tikzgraphpreamble}
    \input{#3}
    \begin{scope}[yscale=\aspectratio]
      \draw[use as bounding box] (-0.5, -0.5) rectangle (0.5, 0.5);
      \node[font=\tiny] at (0, 0) {\inputtikzgraph@url};
    \end{scope}
  \end{tikzpicture}
}

\newlength{\samplelayoutwidth}
\newlength{\samplelayoutheight}
\newbool{tikzgraphpreamble}

\newcommand*{\UpdateSampleLayoutHeight}[1]{%
  \booltrue{tikzgraphpreamble}
  \input{#1}
  \setlength{\samplelayoutheight}{\aspectratio\samplelayoutwidth}
  \boolfalse{tikzgraphpreamble}
}

\newcommand*{\AtBeginSampleLayout}{%
  \path[tikzgraphbbox, use as bounding box]
       (-0.5\samplelayoutwidth, -0.5\samplelayoutheight) rectangle (+0.5\samplelayoutwidth, +0.5\samplelayoutheight);
}

\newcommand*{\WorseLayoutDemo}[2][WORSE]{%
  \begin{figure}
    \begin{center}
      \setlength{\samplelayoutwidth}{0.45\textwidth}
      \UpdateSampleLayoutHeight{pics/#2-00000.tikz}
      \begin{tabular}{c@{\qquad}c}
        \InputTikzGraph[execute at begin picture = \AtBeginSampleLayout]{\samplelayoutwidth}{pics/#2-00000.tikz}&
        \InputTikzGraph[execute at begin picture = \AtBeginSampleLayout]{\samplelayoutwidth}{pics/#2-00500.tikz}\\[1ex]
        \(r=0\,\%\) & \(r=5\,\%\)\\[3ex]
        \InputTikzGraph[execute at begin picture = \AtBeginSampleLayout]{\samplelayoutwidth}{pics/#2-01000.tikz}&
        \InputTikzGraph[execute at begin picture = \AtBeginSampleLayout]{\samplelayoutwidth}{pics/#2-02000.tikz}\\[1ex]
        \(r=10\,\%\) & \(r=20\,\%\)\\[3ex]
        \InputTikzGraph[execute at begin picture = \AtBeginSampleLayout]{\samplelayoutwidth}{pics/#2-05000.tikz}&
        \InputTikzGraph[execute at begin picture = \AtBeginSampleLayout]{\samplelayoutwidth}{pics/#2-10000.tikz}\\[1ex]
        \(r=50\,\%\) & \(r=100\,\%\)
      \end{tabular}
    \end{center}
    \caption[Example \enum{#1} application]{%
      \enum{#1} application illustrated on a regular grid for increasing rates of degradation.
    }
    \label{fig:worsening-#2}
  \end{figure}
}

\newcommand*{\InterLayoutDemoPics}[2][INTER]{%
  \begin{center}
    \setlength{\samplelayoutwidth}{0.45\textwidth}
    \begin{tabular}{c@{\qquad}c}
      \InputTikzGraph{\samplelayoutwidth}{pics/#2-00000.tikz}&
      \InputTikzGraph{\samplelayoutwidth}{pics/#2-02000.tikz}\\[1ex]
      \(r=0\,\%\) & \(r=20\,\%\)\\[3ex]
      \InputTikzGraph{\samplelayoutwidth}{pics/#2-04000.tikz}&
      \InputTikzGraph{\samplelayoutwidth}{pics/#2-06000.tikz}\\[1ex]
      \(r=40\,\%\) & \(r=60\,\%\)\\[3ex]
      \InputTikzGraph{\samplelayoutwidth}{pics/#2-08000.tikz}&
      \InputTikzGraph{\samplelayoutwidth}{pics/#2-10000.tikz}\\[1ex]
      \(r=80\,\%\) & \(r=100\,\%\)
    \end{tabular}
  \end{center}
}

\newcommand*{\PropertyDemo}[2][]{%
  \begin{center}
    \setlength{\samplelayoutwidth}{0.25\textwidth}
    \begin{tabular}{c@{\qquad}c@{\qquad}c}
      \InputTikzGraph*{\samplelayoutwidth}{pics/demograph-a.tikz}&
      \InputTikzGraph*{\samplelayoutwidth}{pics/demograph-b.tikz}&
      \InputTikzGraph*{\samplelayoutwidth}{pics/demograph-c.tikz}\\[1ex]
      \InputLuatikzPlot[width=\samplelayoutwidth, #1]{pics/#2-a.pgf}&
      \InputLuatikzPlot[width=\samplelayoutwidth, #1]{pics/#2-b.pgf}&
      \InputLuatikzPlot[width=\samplelayoutwidth, #1]{pics/#2-c.pgf}
    \end{tabular}
  \end{center}
}

\newlength{\vertexsize}
\setlength{\vertexsize}{3pt}
\tikzset{
  Vertex/.style = {
    font = \footnotesize,
  },
  vertex/.style = {
    shape = circle,
    fill,
    inner sep = 0pt,
    minimum size = \vertexsize,
  },
  edge/.style = {
    draw,
    thin,
  },
  tikzgraphbbox/.style = {
    fill = \ifdraft{yellow!10}{none},
  },
  princomp/.style = {
    draw = white,
    double = black,
    line width = 1.0pt,
    double distance = 2.0pt,
    line cap = round,
    line join = round,
  },
  princomp1st/.style = { princomp },
  princomp2nd/.style = { princomp },
}

\SetKwBlock{Routine}{Routine}{End}
\SetKwComment{comment}{}{}
\SetKwInput{KwConstant}{Constants}
\SetKwProg{SubRoutine}{SubRoutine}{}{End}
\SetKw{KwAnd}{and}
\SetKw{KwBreak}{break}
\SetKw{KwOr}{or}
\SetKw{KwYield}{yield}

% https://tex.stackexchange.com/a/251936
\DefineBibliographyStrings{english}{%
  andothers = {\em et\addabbrvspace al\adddot}
}

\newlength{\xval@hunderter}
\newlength{\xval@hunderter@pm}
\newcommand*{\InputConfusionMatrix}[1]{%
  \begingroup%
    \setlength{\parindent}{\z@}%
    \settowidth{\xval@hunderter}{\ensuremath{100.00\,\%}}%
    \settowidth{\xval@hunderter@pm}{\ensuremath{\pm100.00\,\%}}%
    \newcommand*{\FormatMeanStdev}[2]}%
      \makebox[\xval@hunderter@pm][r]{\ensuremath{\pm\kern0.2em##2\,\%}}%
    }%
    \newcommand*{\FormatCell}[1]{\FormatMeanStdev{\csuse{XVal##1Mean}}{\csuse{XVal##1Stdev}}}%
    \newcommand*{\AltLine}[2]{\multicolumn{4}{l}{\textit{##1:\hfill##2}}}%
    \input{#1}%
    \begin{tabularx}{\linewidth}{X|rr|r}%
      \toprule%
                            & \textit{Cond.~Neg.}   & \textit{Cond.~Pos.}   & \textit{$\Sigma$}\\%
      \midrule%
      \textit{Pred.~Neg.}   & \FormatCell{TrueNeg}  & \FormatCell{FalseNeg} & \FormatCell{PredNeg}\\%
      \textit{Pred.~Pos.}   & \FormatCell{FalsePos} & \FormatCell{TruePos}  & \FormatCell{PredPos}\\%
      \midrule%
      \textit{$\Sigma$}     & \FormatCell{CondNeg}  & \FormatCell{CondPos}  & \FormatMeanStdev{100.00}{0.00}\\%
      \bottomrule
      \multicolumn{4}{l}{}\\
      \AltLine{Success Rate}{\FormatCell{Success}}\\%
      \AltLine{Failure Rate}{\FormatCell{Failure}}\\[1ex]%
      \AltLine{Average Number of Tests}{\ensuremath{\approx\XValCountApprox}}\\%
      \AltLine{Number of Repetitions}{\ensuremath{\XValTestRuns}}\\%
    \end{tabularx}%
  \endgroup%
}

\newcommand*{\InputPunctureResult}[2][\relax]{%
  \begingroup%
  \newcommand*{\PunctureResult}[5][\@undefined]&\ensuremath{##3\,\%}}&
    \ifblank{##4}{\multicolumn{2}{c}{}}{\ensuremath{##4\,\%}&\ensuremath{##5\,\%}}\\
  }%
  \input{#1}%
  \begin{tabularx}{\linewidth}{X@{\qquad}r@{\makebox[1.5em][r]{$\pm$}}rr@{\makebox[1.5em][r]{$\pm$}}r}%
    \toprule%
    \textit{Property} & \multicolumn{2}{r}{\textit{Sole Exclusion}} & \multicolumn{2}{r}{\textit{Sole Inclusion}}\\%
    \midrule%
    \input{#2}%
    \midrule%
    \PunctureResult[\textit{Baseline Using All Properties}]{\XValSuccessMean}{\XValSuccessStdev}{}{}%
    \bottomrule%
  \end{tabularx}%
  \endgroup%
}

\newcommand*{\InputLuatikzPlot}[2][]{\input{#2}}

\newcommand*{\regression}[3][f]{\ensuremath{#1(x)=#2+#3x}}

\makeatother
