% -*- coding:utf-8; mode:latex; -*-

% Copyright (C) 2018 Karlsruhe Institute of Technology
% Copyright (C) 2018 Moritz Klammler <moritz.klammler@alumni.kit.edu>
%
% Copying and distribution of this file, with or without modification, are permitted in any medium without royalty
% provided the copyright notice and this notice are preserved.  This file is offered as-is, without any warranty.

\makeatletter

\usepackage{etoolbox}
\usepackage{ifluatex}
\usepackage{ifpdf}
\usepackage{ifxetex}

\ifboolexpr{ bool {luatex} or bool {xetex} }{%
  \usepackage{fontspec}
  \usepackage{polyglossia}
  \setmainlanguage{english}
}{%
  \usepackage[utf8]{inputenc}
  \usepackage[english]{babel}
}

\usepackage{amsmath,amsfonts}
\usepackage{array}
\usepackage{booktabs}
\usepackage{comment}
\usepackage{csquotes}
\usepackage{gnuplot-lua-tikz}
\usepackage{graphicx}
\usepackage[hidelinks]{hyperref}
\usepackage{ifdraft}
\usepackage{mathrsfs}
\usepackage{pifont}
\usepackage{tabularx}
\usepackage{tikz}
\usepackage{tango-colors}

\usetikzlibrary{calc}
\usetikzlibrary{positioning}
\usetikzlibrary{external}

\newbool{forgdxviii}
\newbool{forarxiv}

\ifdef{\OnlyGDXVIII}{\booltrue{forgdxviii}}{}
\ifdef{\OnlyArxiv}{\booltrue{forarxiv}}{}

\providecommand*{\OnlyGDXVIII}[1]{}
\providecommand*{\OnlyArxiv}[1]{}

\providecommand*{\TikzExternalMode}{graphics if exists}
\tikzexternalize[%
  mode = \TikzExternalMode,
  up to date check = simple,
  figure name = cache/image-,
  aux in dpth = false,
  disable dependency files,
]

\newcommand*{\BigO}{\mathcal{O}}
\newcommand*{\IntsNz}{\mathbb{N}_0}
\newcommand*{\IntsN}{\mathbb{N}}
\newcommand*{\IntsZ}{\mathbb{Z}}
\newcommand*{\Reals}{\mathbb{R}}
\newcommand*{\abs}[1]{\left\lvert#1\right\rvert}
\newcommand*{\card}[1]{\left\lvert#1\right\rvert}
\newcommand*{\ceil}[1]{\left\lceil#1\right\rceil}
\newcommand*{\diff}[1]{\mathrm{d}#1\kern0.3em}
\newcommand*{\enum}[1]{\mbox{\texttt{#1}}}
\newcommand*{\floor}[1]{\left\lfloor#1\right\rfloor}
\newcommand*{\menum}[1]{\mbox{\text{\upshape\ttfamily#1}}}
\newcommand*{\meta}[1]{\ensuremath{\langle#1\rangle}}
\newcommand*{\nint}[1]{\left\lfloor#1\right\rceil}
\newcommand*{\st}{:}
\newcommand*{\vecnorm}[1]{\left\lVert#1\right\rVert}
\newcommand*{\vv}[1]{\vec{#1}}
\newcommand*{\transposed}{^\mathrm{T}}

\newcommand*{\Entropy}{\@ifstar\EntropyDifferential\EntropyDiscrete}
\newcommand*{\EntropyDiscrete}{\mathscr{E}}
\newcommand*{\EntropyDifferential}{\mathscr{D}}

\DeclareMathOperator{\DM}{DM}
\DeclareMathOperator{\GM}{GM}
\DeclareMathOperator{\SM}{SM}
\DeclareMathOperator{\dist}{dist}
\DeclareMathOperator{\length}{length}
\DeclareMathOperator{\stress}{\mathcal{T}}
\DeclareMathOperator{\ReLU}{ReLU}

\let\epsilon\varepsilon

\newcommand*{\mathendpunct}[1]{\,#1}
\newcommand*{\percent}{\,\%}

\newcommand*{\latinphrase}[1]{\emph{#1}}

\newlength{\samplelayoutwidth}
\newlength{\samplelayoutheight}
\newbool{tikzgraphpreamble}

\newcommand*{\InputLuatikzPlot}[2][]{%
  \ifpdf%
    \tikzsetnextfilename{cache/#2}%
    \input{pics/#2.pgf}%
  \else%
    \pgfimage{cache/#2}%
  \fi%
}

\newcommand*{\InputTikzGraph}[3][]{%
  \ifdraft{%
    \urldef\inputtikzgraph@url\url{#3}%
    \begin{tikzpicture}[x=#2, y=#2, #1]%
      \booltrue{tikzgraphpreamble}%
      \input{pics/#3.tikz}%
      \begin{scope}[yscale=\aspectratio]%
        \draw[use as bounding box] (-0.5, -0.5) rectangle (0.5, 0.5);%
        \node[font=\tiny] at (0, 0) {\inputtikzgraph@url};%
      \end{scope}%
    \end{tikzpicture}%
  }{%
    \ifbool{pdf}{%
      \tikzsetnextfilename{cache/#3}%
      \tikz[x=#2, y=#2, #1]{\input{pics/#3.tikz}}%
    }{%
      \pgfimage[width=#2]{cache/#3}%
    }%
  }%
}

\colorlet{gd-node-color}{TangoSkyBlue2}%
\colorlet{gd-edge-color}{TangoSkyBlue3}%

\tikzset{
  vertex/.style = {
    shape = circle,
    fill = gd-node-color,
    inner sep = 0pt,
    minimum size = 2.25pt,
  },
  edge/.style = {
    draw = gd-edge-color,
    thin,
  },
  Vertex/.style = {
    font = \footnotesize,
  },
}

\newcommand*{\FormatPercentageWithError}[3][2.5em]{%
  \ifblank{#2}{}{\ensuremath{(\makebox[#1][r]{\ensuremath{#2}}\,\pm\makebox[2em][r]{\ensuremath{#3}}\,)\percent}}%
}

\newcommand*{\InputCompetingMetrics}[1]{%
  \begingroup%
  \newcommand*{\CompetingMetricResult}[5][\@undefined]{%
    ##1 & \FormatPercentageWithError{##2}{##3} & \FormatPercentageWithError[2em]{##4}{##5}\\
  }%
  \begin{tabularx}{\linewidth}{Xr@{\qquad}r}%
    \toprule%
    \textit{Metric} & \textit{Success Rate} & \textit{Advantage}\\%
    \midrule%
    \input{#1}%
    \bottomrule%
  \end{tabularx}%
  \endgroup%
}

\newcommand*{\InputPunctureResult}[2][\relax]{%
  \begingroup%
  \newcommand*{\PunctureResult}[5][\@undefined]{%
    ##1 & \FormatPercentageWithError{##2}{##3} & \FormatPercentageWithError{##4}{##5}\\
  }%
  \input{#1}%
  \begin{tabularx}{\linewidth}{Xr@{\qquad}r}%
    \toprule%
    \textit{Property} & \textit{Sole Exclusion} & \textit{Sole Inclusion}\\%
    \midrule%
    \input{#2}%
    \midrule%
    \PunctureResult[\textit{Baseline Using All Properties}]{\XValSuccessMean}{\XValSuccessStdev}{}{}%
    \bottomrule%
  \end{tabularx}%
  \endgroup%
}

\newcommand*{\HuangWeightResult}[3][]{%
  \csdef{HuangWeight#1}{$w_{\menum{#1}}=#2\pm#3$}%
  \ifdef{\TextualHuangWeights}{%
    \apptocmd{\TextualHuangWeights}{, \csuse{HuangWeight#1}}{}{}%
  }{%
    \newcommand*{\TextualHuangWeights}{\csuse{HuangWeight#1}}%
  }%
}

\newcommand*{\regression}[3][???]{%
  \ensuremath{%
    r = \makebox[2.1em][r]{\ensuremath{#1\percent}}%
    \quad%
    f(x) = #2 + #3 \kern2pt x%
  }%
}

\newenvironment{explanation}{%
  \newcommand*{\explain}[1]{\item[{##1}]}%
  \begin{description}%
}{%
  \end{description}%
}

\makeatother
