% -*- coding:utf-8; mode:latex; -*-

% Copyright (C) 2018 Moritz Klammler <moritz.klammler@alumni.kit.edu>
% Copyright (C) 2018 Tamara Mchedlidze <mched@iti.uka.de>
% Copyright (C) 2018 Alexey Pak <alexey.pak@iosb.fraunhofer.de>
%
% This work is licensed under a Creative Commons Attribution-NonCommercial-NoDerivatives 4.0 International License
% (https://creativecommons.org/licenses/by-nc-nd/4.0/).

\section{Introduction}
\label{sec:intro}

What makes a drawing of a graph aesthetically pleasing? This admittedly vague question is central to the field of Graph
Drawing which has over its history suggested numerous answers.  Borrowing ideas from Mathematics, Physics, Arts, etc.,
many researchers have tried to formalize the elusive concept of aesthetics.

In particular, dozens of formulas collectively known as \emph{drawing aesthetics} (or, more precisely, \emph{quality
  metrics}~\cite{EadesH0K17}) have been proposed that attempt to capture in a single number how beautiful, readable and
clear a drawing of an abstract graph is.  Of those, simple metrics such as the number of edge crossings, minimum
crossing angle, vertex distribution or angular resolution parameters, are obviously incapable \latinphrase{per se} of
providing the ultimate aesthetic statement.  Advanced metrics may represent, for example, the energy of a corresponding
system of physical bodies~\cite{eades84,FruchtermanR91}.  This approach underlies many popular graph drawing
algorithms~\cite{Tamassia2013} and often leads to pleasing results in practice.  However, it is known that low values of
energy or stress do not always correspond to the highest degree of symmetry~\cite{Welch2017} which is an important
aesthetic criterion~\cite{PurchaseCM96}.

Another direction of research aims to narrow the scope of the original question to specific application domains,
focusing on the purpose of a drawing or possible user actions it may facilitate (\emph{tasks}).  The target parameters
-- readability and the clarity of representation -- may be assessed via user performance studies.  However, even in this
case such aesthetic notions as symmetry still remain important~\cite{PurchaseCM96}.  In general, aesthetically pleasing
designs are known to positively affect the apparent and the actual usability~\cite{Norman02,TractinskyKI00} of
interfaces and induce positive mental states of users, enhancing their problem-solving abilities~\cite{Fredrickson98}.

In this work, we offer an alternative perspective on the aesthetics of graph drawings.  First, we address a slightly
modified question: \enquote{Of two given drawings of the same graph, which one is more aesthetically pleasing?}.  With
that, we implicitly admit that \enquote{the ultimate} quality metric may not exist and one can hope for at most a
(partial) ordering.  Instead of a metric, we therefore search for a binary \emph{discriminator function} of graph
drawings.  As limited as it is, it could be useful for practical applications such as picking the best answer out of
outputs of several drawing algorithms or resolving local minima in layout optimization.

Second, like Huang et al.~\cite{Huang2013}, we believe that by combining multiple metrics computed for each drawing, one
has a better chance of capturing complex aesthetic properties.  We thus also consider a \enquote{meta-algorithm} that
aggregates several \enquote{input} metrics into a single value.  However, unlike the recipe by Huang et al., we do not
specify the form of this combination \latinphrase{a priori} but let an artificial neural network \enquote{learn} it
based on a sample of labeled training data.  In the recent years, machine learning techniques have proven useful in such
aesthetics-related tasks as assessing the appeal of 3D shapes~\cite{DevLL16} or cropping photos~\cite{Nishiyama09}.  Our
network architecture is based on a so-called \emph{Siamese neural network}~\cite{Bromley1994} -- a generic model
specifically designed for binary functions of same-kind inputs.

Finally, we acknowledge that any simple or complex input metric may become crucial to the answer in some cases that are
hard to predict \latinphrase{a priori}.  We therefore implement as many input metrics as we can and relegate their
ranking to the model.  In addition to those known from the literature, we implement a few novel metrics inspired by
statistical tools used in Condensed Matter Physics and Crystallography, which we expect to be helpful in capturing the
symmetry, balance, and salient structures in large graphs.  These metrics are based on so-called \emph{syndromes} --
variable-size multi-sets of numbers computed for a graph or its drawing (e.g.~vertex coordinates or pairwise distances).
In order to reduce these heterogeneous multi-sets to a fixed-size \emph{feature vector} (input to the discriminator
model), we perform a \emph{feature extraction} process which may involve steps such as creating histograms or performing
regressions.

In our experiments, our discriminator model outperforms the known (metric-based) algorithms and achieves an average
accuracy of $\XValSuccessMean\percent$ when identifying the \enquote{better} graph drawing out of a pair.  The project
source code including the data generation procedure is available online~\cite{GitHubRepo}.

The remainder of this paper is structured as follows.  In section~\ref{sec:relwork} we briefly overview the
state-of-the-art in quantifying graph layout aesthetics.  Section~\ref{sec:syndromes} discusses the used syndromes of
aesthetic quality, section~\ref{sec:featex} feature extraction, and section~\ref{sec:model} the discriminator model.
The dataset used in our experiments is described in section~\ref{sec:data}.  The results and the comparisons with the
known metrics are presented in section~\ref{sec:eval}.  Section~\ref{sec:conclusion} finalizes the paper and provides an
outlook for future work.
