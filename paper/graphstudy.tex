% -*- coding:utf-8; mode:latex; -*-

% Copyright (C) 2018 Moritz Klammler <moritz.klammler@alumni.kit.edu>
% Copyright (C) 2018 Tamara Mchedlidze <mched@iti.uka.de>
% Copyright (C) 2018 Alexey Pak <alexey.pak@iosb.fraunhofer.de>
%
% This work is licensed under a Creative Commons Attribution-NonCommercial-NoDerivatives 4.0 International License
% (https://creativecommons.org/licenses/by-nc-nd/4.0/).

%% If you whish to build a 'draft' version of the paper, please don't edit this file (at least, don't commit the
%% change).  Instead, create (if it doesn't exist already) the file `preamble-local.tex` in the same directory and put
%% the definition `\def\GlobalClassOptions{draft}` into it.  TeX will then pick up the definition and build the 'draft'
%% version as desired without colbbering any version-controlled file.  (Never ever commit `preamble-local.tex` to Git;
%% it is for /your/ local settings only.)

\providecommand*{\GlobalClassOptions}{}
\InputIfFileExists{preamble-local.tex}{}{}

\documentclass[\GlobalClassOptions]{llncs}
% -*- coding:utf-8; mode:latex; -*- %

% Copyright (C) 2018 Moritz Klammler <moritz.klammler@student.kit.edu>
%
% Copying and distribution of this file, with or without modification, are permitted in any medium without royalty
% provided the copyright notice and this notice are preserved.  This file is offered as-is, without any warranty.

\makeatletter

\usepackage{booktabs}
\usepackage{tabularx}
\usepackage{gnuplot-lua-tikz}

\newcommand*{\email}[1]{\mbox{\texttt{\href{mailto:#1}{#1}}}}

\newcommand*{\Reals}[1][]{\ifblank{#1}{\mathbb{R}}{\mathbb{R}_{#1}}}
\newcommand*{\RealsPos}{\Reals[>0]}
\newcommand*{\RealsNN}{\Reals[\geq0]}
\newcommand*{\IntsZ}{\mathbb{Z}}
\newcommand*{\IntsN}{\mathbb{N}}
\newcommand*{\IntsNz}{\mathbb{N}_0}
\newcommand*{\BitZero}{\text{\texttt{0}}}
\newcommand*{\BitOne}{\text{\texttt{1}}}
\newcommand*{\Bits}{\{\BitZero,\BitOne\}}

\let\vec\relax
\let\Vec\relax
\let\implies\Rightarrow
\let\impliedby\Leftarrow
\let\equivalent\Leftrightarrow
\newcommand*{\suchthat}{:}
\newcommand*{\degree}{^\circ}
\let\conjunction\cap
\let\disjunction\cup
\newcommand*{\diff}[1]{\mathrm{d}#1\kern0.3em}

\newcommand*{\Layout}{\Gamma}
\newcommand*{\Graph}{G}
\newcommand*{\GraphV}{V}
\newcommand*{\GraphE}{E}
\newcommand*{\vecz}{0}

\newcommand*{\GraphGVE}{\Graph=(\GraphV,\GraphE)}

\newcommand*{\CMake}{\mbox{CMake}}
\newcommand*{\CXX}{\mbox{C++}}
\newcommand*{\JavaScript}{\mbox{JavaScript}}
\newcommand*{\Keras}{\mbox{Keras}}
\newcommand*{\Python}{\mbox{Python}}
\newcommand*{\SQLite}{\mbox{SQLite}}
\newcommand*{\TensorFlow}{\mbox{TensorFlow}}

\newcommand*{\Entropy}{\@ifstar\EntropyDifferential\EntropyDiscrete}
\newcommand*{\EntropyDiscrete}{S}
\newcommand*{\EntropyDifferential}{\bar{S}}

\DeclareMathOperator{\BigO}{\mathcal{O}}
\DeclareMathOperator{\dist}{dist}
\DeclareMathOperator{\edgelen}{length}
\DeclareMathOperator{\mean}{mean}
\DeclareMathOperator{\rms}{rms}
\DeclareMathOperator{\sign}{sign}
\DeclareMathOperator{\stdevp}{stdevp}
\DeclareMathOperator{\stdev}{stdev}
\DeclareMathOperator{\stress}{stress}

\DeclareMathOperator*{\Conjunction}{\bigcap}
\DeclareMathOperator*{\Disjunction}{\bigcup}
\DeclareMathOperator*{\argmax}{arg\,max}
\DeclareMathOperator*{\argmin}{arg\,min}

\newcommand*{\abs}[1]{\left\lvert#1\right\rvert}
\newcommand*{\card}[1]{\left\lvert#1\right\rvert}
\newcommand*{\vecnorm}[1]{\left\lVert#1\right\rVert}
\newcommand*{\floor}[1]{\left\lfloor#1\right\rfloor}
\newcommand*{\ceil}[1]{\left\lceil#1\right\rceil}
\newcommand*{\nint}[1]{\left\lfloor#1\right\rceil}
\newcommand*{\multiset}[1]{\left[#1\right]}

\newcommand{\bra}[2][\relax]{%
  \ifx\relax#1\relax%
    \left\langle#2\right|%
  \else%
    \left\langle#2\middle|#1\right|%
  \fi%
}

\newcommand{\ket}[2][\relax]{%
  \ifx\relax#1\relax%
    \left|#2\right\rangle%
  \else%
    \left|#1\middle|#2\right\rangle%
  \fi%
}

\newcommand{\braket}[3][\relax]{%
  \ifx\relax#1\relax%
    \left\langle#2\middle|#3\right\rangle%
  \else%
    \left\langle#2\middle|#1\middle|#3\right\rangle%
  \fi%
}

\newcommand*{\enum}[1]{\mbox{\upshape\ttfamily#1}}
\newcommand*{\menum}[1]{\mbox{\text{\upshape\ttfamily#1}}}
\newcommand*{\QuasiNd}{\enum{QUASI\(\langle{n}\rangle\)D}}
\newcommand*{\TorusN}{\enum{TORUS\(\langle{n}\rangle\)}}
\newcommand*{\code}[1]{\mbox{\texttt{#1}}}

\renewcommand*{\acroenparen}[1]{(\emph{#1})}

\newcommand*{\latintext}[1]{\emph{\textlatin{#1}}}

\newcommand*{\InputTikz}[2][]{\tikz[#1]{\input{#2}}}

\ifdraft{%
  \newcommand*{\InputTikzGraph}{\@ifstar\InputTikzGraph@Dummy\InputTikzGraph@Dummy}
}{%
  \newcommand*{\InputTikzGraph}{\@ifstar\InputTikzGraph@Big\InputTikzGraph@NotBig}
}

\newcommand*{\InputTikzGraph@NotBig}[3][]{\tikz[x=#2, y=#2, #1]{\input{#3}}}
\newcommand*{\InputTikzGraph@Big}[3][]{\tikz[x=#2, y=#2, #1]{\setlength{\vertexsize}{2pt}\input{#3}}}

\newcommand*{\InputTikzGraph@Dummy}[3][]{%
  \urldef\inputtikzgraph@url\url{#3}%
  \begin{tikzpicture}[x=#2, y=#2, #1]
    \booltrue{tikzgraphpreamble}
    \input{#3}
    \begin{scope}[yscale=\aspectratio]
      \draw[use as bounding box] (-0.5, -0.5) rectangle (0.5, 0.5);
      \node[font=\tiny] at (0, 0) {\inputtikzgraph@url};
    \end{scope}
  \end{tikzpicture}
}

\newlength{\samplelayoutwidth}
\newlength{\samplelayoutheight}
\newbool{tikzgraphpreamble}

\newcommand*{\UpdateSampleLayoutHeight}[1]{%
  \booltrue{tikzgraphpreamble}
  \input{#1}
  \setlength{\samplelayoutheight}{\aspectratio\samplelayoutwidth}
  \boolfalse{tikzgraphpreamble}
}

\newcommand*{\AtBeginSampleLayout}{%
  \path[tikzgraphbbox, use as bounding box]
       (-0.5\samplelayoutwidth, -0.5\samplelayoutheight) rectangle (+0.5\samplelayoutwidth, +0.5\samplelayoutheight);
}

\newcommand*{\WorseLayoutDemo}[2][WORSE]{%
  \begin{figure}
    \begin{center}
      \setlength{\samplelayoutwidth}{0.45\textwidth}
      \UpdateSampleLayoutHeight{pics/#2-00000.tikz}
      \begin{tabular}{c@{\qquad}c}
        \InputTikzGraph[execute at begin picture = \AtBeginSampleLayout]{\samplelayoutwidth}{pics/#2-00000.tikz}&
        \InputTikzGraph[execute at begin picture = \AtBeginSampleLayout]{\samplelayoutwidth}{pics/#2-00500.tikz}\\[1ex]
        \(r=0\,\%\) & \(r=5\,\%\)\\[3ex]
        \InputTikzGraph[execute at begin picture = \AtBeginSampleLayout]{\samplelayoutwidth}{pics/#2-01000.tikz}&
        \InputTikzGraph[execute at begin picture = \AtBeginSampleLayout]{\samplelayoutwidth}{pics/#2-02000.tikz}\\[1ex]
        \(r=10\,\%\) & \(r=20\,\%\)\\[3ex]
        \InputTikzGraph[execute at begin picture = \AtBeginSampleLayout]{\samplelayoutwidth}{pics/#2-05000.tikz}&
        \InputTikzGraph[execute at begin picture = \AtBeginSampleLayout]{\samplelayoutwidth}{pics/#2-10000.tikz}\\[1ex]
        \(r=50\,\%\) & \(r=100\,\%\)
      \end{tabular}
    \end{center}
    \caption[Example \enum{#1} application]{%
      \enum{#1} application illustrated on a regular grid for increasing rates of degradation.
    }
    \label{fig:worsening-#2}
  \end{figure}
}

\newcommand*{\InterLayoutDemoPics}[2][INTER]{%
  \begin{center}
    \setlength{\samplelayoutwidth}{0.45\textwidth}
    \begin{tabular}{c@{\qquad}c}
      \InputTikzGraph{\samplelayoutwidth}{pics/#2-00000.tikz}&
      \InputTikzGraph{\samplelayoutwidth}{pics/#2-02000.tikz}\\[1ex]
      \(r=0\,\%\) & \(r=20\,\%\)\\[3ex]
      \InputTikzGraph{\samplelayoutwidth}{pics/#2-04000.tikz}&
      \InputTikzGraph{\samplelayoutwidth}{pics/#2-06000.tikz}\\[1ex]
      \(r=40\,\%\) & \(r=60\,\%\)\\[3ex]
      \InputTikzGraph{\samplelayoutwidth}{pics/#2-08000.tikz}&
      \InputTikzGraph{\samplelayoutwidth}{pics/#2-10000.tikz}\\[1ex]
      \(r=80\,\%\) & \(r=100\,\%\)
    \end{tabular}
  \end{center}
}

\newcommand*{\PropertyDemo}[2][]{%
  \begin{center}
    \setlength{\samplelayoutwidth}{0.25\textwidth}
    \begin{tabular}{c@{\qquad}c@{\qquad}c}
      \InputTikzGraph*{\samplelayoutwidth}{pics/demograph-a.tikz}&
      \InputTikzGraph*{\samplelayoutwidth}{pics/demograph-b.tikz}&
      \InputTikzGraph*{\samplelayoutwidth}{pics/demograph-c.tikz}\\[1ex]
      \InputLuatikzPlot[width=\samplelayoutwidth, #1]{pics/#2-a.pgf}&
      \InputLuatikzPlot[width=\samplelayoutwidth, #1]{pics/#2-b.pgf}&
      \InputLuatikzPlot[width=\samplelayoutwidth, #1]{pics/#2-c.pgf}
    \end{tabular}
  \end{center}
}

\newlength{\vertexsize}
\setlength{\vertexsize}{3pt}
\tikzset{
  Vertex/.style = {
    font = \footnotesize,
  },
  vertex/.style = {
    shape = circle,
    fill,
    inner sep = 0pt,
    minimum size = \vertexsize,
  },
  edge/.style = {
    draw,
    thin,
  },
  tikzgraphbbox/.style = {
    fill = \ifdraft{yellow!10}{none},
  },
  princomp/.style = {
    draw = white,
    double = black,
    line width = 1.0pt,
    double distance = 2.0pt,
    line cap = round,
    line join = round,
  },
  princomp1st/.style = { princomp },
  princomp2nd/.style = { princomp },
}

\SetKwBlock{Routine}{Routine}{End}
\SetKwComment{comment}{}{}
\SetKwInput{KwConstant}{Constants}
\SetKwProg{SubRoutine}{SubRoutine}{}{End}
\SetKw{KwAnd}{and}
\SetKw{KwBreak}{break}
\SetKw{KwOr}{or}
\SetKw{KwYield}{yield}

% https://tex.stackexchange.com/a/251936
\DefineBibliographyStrings{english}{%
  andothers = {\em et\addabbrvspace al\adddot}
}

\newlength{\xval@hunderter}
\newlength{\xval@hunderter@pm}
\newcommand*{\InputConfusionMatrix}[1]{%
  \begingroup%
    \setlength{\parindent}{\z@}%
    \settowidth{\xval@hunderter}{\ensuremath{100.00\,\%}}%
    \settowidth{\xval@hunderter@pm}{\ensuremath{\pm100.00\,\%}}%
    \newcommand*{\FormatMeanStdev}[2]}%
      \makebox[\xval@hunderter@pm][r]{\ensuremath{\pm\kern0.2em##2\,\%}}%
    }%
    \newcommand*{\FormatCell}[1]{\FormatMeanStdev{\csuse{XVal##1Mean}}{\csuse{XVal##1Stdev}}}%
    \newcommand*{\AltLine}[2]{\multicolumn{4}{l}{\textit{##1:\hfill##2}}}%
    \input{#1}%
    \begin{tabularx}{\linewidth}{X|rr|r}%
      \toprule%
                            & \textit{Cond.~Neg.}   & \textit{Cond.~Pos.}   & \textit{$\Sigma$}\\%
      \midrule%
      \textit{Pred.~Neg.}   & \FormatCell{TrueNeg}  & \FormatCell{FalseNeg} & \FormatCell{PredNeg}\\%
      \textit{Pred.~Pos.}   & \FormatCell{FalsePos} & \FormatCell{TruePos}  & \FormatCell{PredPos}\\%
      \midrule%
      \textit{$\Sigma$}     & \FormatCell{CondNeg}  & \FormatCell{CondPos}  & \FormatMeanStdev{100.00}{0.00}\\%
      \bottomrule
      \multicolumn{4}{l}{}\\
      \AltLine{Success Rate}{\FormatCell{Success}}\\%
      \AltLine{Failure Rate}{\FormatCell{Failure}}\\[1ex]%
      \AltLine{Average Number of Tests}{\ensuremath{\approx\XValCountApprox}}\\%
      \AltLine{Number of Repetitions}{\ensuremath{\XValTestRuns}}\\%
    \end{tabularx}%
  \endgroup%
}

\newcommand*{\InputPunctureResult}[2][\relax]{%
  \begingroup%
  \newcommand*{\PunctureResult}[5][\@undefined]&\ensuremath{##3\,\%}}&
    \ifblank{##4}{\multicolumn{2}{c}{}}{\ensuremath{##4\,\%}&\ensuremath{##5\,\%}}\\
  }%
  \input{#1}%
  \begin{tabularx}{\linewidth}{X@{\qquad}r@{\makebox[1.5em][r]{$\pm$}}rr@{\makebox[1.5em][r]{$\pm$}}r}%
    \toprule%
    \textit{Property} & \multicolumn{2}{r}{\textit{Sole Exclusion}} & \multicolumn{2}{r}{\textit{Sole Inclusion}}\\%
    \midrule%
    \input{#2}%
    \midrule%
    \PunctureResult[\textit{Baseline Using All Properties}]{\XValSuccessMean}{\XValSuccessStdev}{}{}%
    \bottomrule%
  \end{tabularx}%
  \endgroup%
}

\newcommand*{\InputLuatikzPlot}[2][]{\input{#2}}

\newcommand*{\regression}[3][f]{\ensuremath{#1(x)=#2+#3x}}

\makeatother


\title{Aesthetic Discrimination of Graph Layouts}
\author{
  Moritz Klammler\inst{1}\and
  Tamara Mchedlidze\inst{1}\and
  Alexey Pak\inst{2}}

\institute{
  Karlsruhe Institute of Technology, 76131 Karlsruhe, Germany\and
  Fraunhofer Institute of Optronics, System Technologies and Image Exploitation,
  Fraunhoferstra{\ss}e 1, 76131 Karlsruhe, Germany\\
  \email{moritz@klammler.eu},
  \email{mched@iti.uka.de},
  \email{alexey.pak@iosb.fraunhofer.de}
}

\AtBeginDocument{
  \input{nn-info.tex}
  \input{eval-cross-valid.tex}
  \input{eval-huang-weights.tex}
}

\begin{document}

\maketitle

\begin{abstract}
  This paper addresses the following basic question: given two layouts of the same graph, which one is more
  aesthetically pleasing? We propose a neural network-based discriminator model trained on a labeled dataset that
  decides which of two layouts has a higher aesthetic quality. The feature vectors used as inputs to the model are based
  on known graph drawing quality metrics, classical statistics, information-theoretical quantities, and two-point
  statistics inspired by methods of condensed matter physics. The large corpus of layout pairs used for training and
  testing is constructed using force-directed drawing algorithms and the layouts that naturally stem from the process of
  graph generation. It is further extended using data augmentation techniques. Our model demonstrates a mean prediction
  accuracy of $\XValSuccessMean\percent$, outperforming discriminators based on stress and on the linear combination of
  popular quality metrics by a small but statistically significant margin.

  \OnlyGDXVIII{%
    The full version of the paper including the appendix with additional illustrations is available at
    \url{https://arxiv.org/abs/1809.01017}.
  }
  \OnlyArxiv{%
    This paper appears in the Proceedings of the 26\textsuperscript{th} International Symposium on Graph Drawing and
    Network Visualization (GD~2018).
  }

  \keywords{%
    graph drawing\and
    graph drawing aesthetics\and
    machine lear\-ning\and
    neural networks\and
    graph drawing syndromes
  }
\end{abstract}

% -*- coding:utf-8; mode:latex; -*-

% Copyright (C) 2018 Moritz Klammler <moritz.klammler@alumni.kit.edu>
% Copyright (C) 2018 Tamara Mchedlidze <mched@iti.uka.de>
% Copyright (C) 2018 Alexey Pak <alexey.pak@iosb.fraunhofer.de>
%
% This work is licensed under a Creative Commons Attribution-NonCommercial-NoDerivatives 4.0 International License
% (https://creativecommons.org/licenses/by-nc-nd/4.0/).

\section{Introduction}
\label{sec:intro}

What makes a drawing of a graph aesthetically pleasing? This admittedly vague question is central to the field of Graph
Drawing which has over its history suggested numerous answers.  Borrowing ideas from Mathematics, Physics, Arts, etc.,
many researchers have tried to formalize the elusive concept of aesthetics.

In particular, dozens of formulas collectively known as \emph{drawing aesthetics} (or, more precisely, \emph{quality
  metrics}~\cite{EadesH0K17}) have been proposed that attempt to capture in a single number how beautiful, readable and
clear a drawing of an abstract graph is.  Of those, simple metrics such as the number of edge crossings, minimum
crossing angle, vertex distribution or angular resolution parameters, are obviously incapable \latinphrase{per se} of
providing the ultimate aesthetic statement.  Advanced metrics may represent, for example, the energy of a corresponding
system of physical bodies~\cite{eades84,FruchtermanR91}.  This approach underlies many popular graph drawing
algorithms~\cite{Tamassia2013} and often leads to pleasing results in practice.  However, it is known that low values of
energy or stress do not always correspond to the highest degree of symmetry~\cite{Welch2017} which is an important
aesthetic criterion~\cite{PurchaseCM96}.

Another direction of research aims to narrow the scope of the original question to specific application domains,
focusing on the purpose of a drawing or possible user actions it may facilitate (\emph{tasks}).  The target parameters
-- readability and the clarity of representation -- may be assessed via user performance studies.  However, even in this
case such aesthetic notions as symmetry still remain important~\cite{PurchaseCM96}.  In general, aesthetically pleasing
designs are known to positively affect the apparent and the actual usability~\cite{Norman02,TractinskyKI00} of
interfaces and induce positive mental states of users, enhancing their problem-solving abilities~\cite{Fredrickson98}.

In this work, we offer an alternative perspective on the aesthetics of graph drawings.  First, we address a slightly
modified question: \enquote{Of two given drawings of the same graph, which one is more aesthetically pleasing?}.  With
that, we implicitly admit that \enquote{the ultimate} quality metric may not exist and one can hope for at most a
(partial) ordering.  Instead of a metric, we therefore search for a binary \emph{discriminator function} of graph
drawings.  As limited as it is, it could be useful for practical applications such as picking the best answer out of
outputs of several drawing algorithms or resolving local minima in layout optimization.

Second, like Huang et al.~\cite{Huang2013}, we believe that by combining multiple metrics computed for each drawing, one
has a better chance of capturing complex aesthetic properties.  We thus also consider a \enquote{meta-algorithm} that
aggregates several \enquote{input} metrics into a single value.  However, unlike the recipe by Huang et al., we do not
specify the form of this combination \latinphrase{a priori} but let an artificial neural network \enquote{learn} it
based on a sample of labeled training data.  In the recent years, machine learning techniques have proven useful in such
aesthetics-related tasks as assessing the appeal of 3D shapes~\cite{DevLL16} or cropping photos~\cite{Nishiyama09}.  Our
network architecture is based on a so-called \emph{Siamese neural network}~\cite{Bromley1994} -- a generic model
specifically designed for binary functions of same-kind inputs.

Finally, we acknowledge that any simple or complex input metric may become crucial to the answer in some cases that are
hard to predict \latinphrase{a priori}.  We therefore implement as many input metrics as we can and relegate their
ranking to the model.  In addition to those known from the literature, we implement a few novel metrics inspired by
statistical tools used in Condensed Matter Physics and Crystallography, which we expect to be helpful in capturing the
symmetry, balance, and salient structures in large graphs.  These metrics are based on so-called \emph{syndromes} --
variable-size multi-sets of numbers computed for a graph or its drawing (e.g.~vertex coordinates or pairwise distances).
In order to reduce these heterogeneous multi-sets to a fixed-size \emph{feature vector} (input to the discriminator
model), we perform a \emph{feature extraction} process which may involve steps such as creating histograms or performing
regressions.

In our experiments, our discriminator model outperforms the known (metric-based) algorithms and achieves an average
accuracy of $\XValSuccessMean\percent$ when identifying the \enquote{better} graph drawing out of a pair.  The project
source code including the data generation procedure is available online~\cite{GitHubRepo}.

The remainder of this paper is structured as follows.  In section~\ref{sec:relwork} we briefly overview the
state-of-the-art in quantifying graph layout aesthetics.  Section~\ref{sec:syndromes} discusses the used syndromes of
aesthetic quality, section~\ref{sec:featex} feature extraction, and section~\ref{sec:model} the discriminator model.
The dataset used in our experiments is described in section~\ref{sec:data}.  The results and the comparisons with the
known metrics are presented in section~\ref{sec:eval}.  Section~\ref{sec:conclusion} finalizes the paper and provides an
outlook for future work.

% -*- coding:utf-8; mode:latex; -*-

% Copyright (C) 2018 Moritz Klammler <moritz.klammler@alumni.kit.edu>
% Copyright (C) 2018 Tamara Mchedlidze <mched@iti.uka.de>
% Copyright (C) 2018 Alexey Pak <alexey.pak@iosb.fraunhofer.de>
%
% This work is licensed under a Creative Commons Attribution-NonCommercial-NoDerivatives 4.0 International License
% (https://creativecommons.org/licenses/by-nc-nd/4.0/).

\section{Related Work}
\label{sec:relwork}

According to empirical studies, graph drawings that maximize one or several quality metrics are more aethetically
pleasing and easier to read~\cite{HuangE05,Huang2013,Purchase97,PurchaseHNK12,WarePCM02}.  For instance, in their
seminal work, Purchase et al.~have established~\cite{PurchaseCM96} that higher numbers of edge crossings and bends as
well as lower levels of symmetry negatively influence user performance in graph reading tasks.

Many graph drawing algorithms attempt to optimize multiple quality metrics.  As one way to combine them, Huang et
al.~\cite{Huang2013} have used a weighted sum of \enquote{simple} metrics, effects of their interactions (see
Purchase~\cite{PURCHASE98} or Huang and Huang~\cite{Huang10}), and error terms to account for possible measurement
errors.

In another work, Huang et al.~\cite{HuangHL16} have empirically demonstrated that their \enquote{aggregate} metric is
sensitive to quality changes and is correlated with the human performance in graph comprehension tasks.  They have also
noticed that the dependence of aesthetic quality on input quality metrics can be non-linear (e.g.~a quadratic
relationship better describes the interplay between crossing angles and drawing quality~\cite{HuangHE08}).  Our work
extends this idea as we allow for arbitrary non-linear dependencies implemented by an artificial neural network.

In evolutionary graph drawing approaches, several techniques have been suggested to \enquote{train} a \emph{fitness
  function}\footnote{Objective function in genetic algorithms that summarizes optimization goals.}  from the user's
responses as a composition of several known quality metrics.  Masui~\cite{Masui1994} modeled the fitness function as a
linear combination in which the weights are obtained via genetic programming from the pairs of \enquote{good} and
\enquote{bad} layouts provided by users.  The so-called co-evolution was used by Barbosa and Barreto~\cite{Barbosa2001}
to evolve the weights of the fitness function in parallel with a drawing population in order to match the ranking made
by users.  Sp\"{o}nemann and others~\cite{Sponemann14} suggested two alternative techniques.  In the first one, the user
directly chooses the weights with a slider.  In the second, they select good layouts from the current population and the
weights are adjusted according to the selection.  Rosete-Suarez~\cite{Rosete-Suarez99} determined the relative
importance of individual quality metrics based on user inputs.  Several machine learning-based approaches to graph
drawing are described by dos Santos Vieira et al.~\cite{Raissa2015}.  Recently, Kwon et al.~\cite{Kwon18} presented a
novel work on topological similarity of graphs.  Their goal was to avoid expensive computations of graph layouts and
their quality measures.  The resulting system was able to sketch a graph in different layouts and estimate corresponding
quality measures.

% -*- coding:utf-8; mode:latex; -*-

% Copyright (C) 2018 Moritz Klammler <moritz.klammler@alumni.kit.edu>
% Copyright (C) 2018 Tamara Mchedlidze <mched@iti.uka.de>
% Copyright (C) 2018 Alexey Pak <alexey.pak@iosb.fraunhofer.de>
%
% This work is licensed under a Creative Commons Attribution-NonCommercial-NoDerivatives 4.0 International License
% (https://creativecommons.org/licenses/by-nc-nd/4.0/).

\section{Definitions}
\label{sec:definitions}

In this paper we consider general simple graphs $G=(V,E)$ where $V=V(G)$ and $E=E(G)$ are the vertex and edge sets of
$G$ with $\card{V}=n$ and $\card{E}=m$.  A \emph{drawing} or \emph{layout} of a graph is its graphical representation
where vertices are drawn as points or small circles, and the edges as straight line segments.  Vertex positions in a
drawing are denoted by $\vec{p^k}=(p_1^k, p_2^k)\transposed$ for $k=1,\dots,n$ and their set $P=\{\vec{p}^k\}_{k=1}^n$.
Furthermore, we use $\textrm{dist}_G(u,v)$ to denote the \emph{graph-theoretical distance} -- the length of the shortest
path between vertices $u$ and $v$ in $G$ -- and $\textrm{dist}_\Gamma(u,v)$ for the Euclidean distance between $u$ and
$v$ in the drawing $\Gamma(G)$.

\section{Quality Syndromes of Graph Layouts}
\label{sec:syndromes}

A \emph{quality syndrome} of a layout $\Gamma$ is a multi-set of numbers sharing an interpretation that are known or
suspected to correlate with the aesthetic quality (e.g.~all pairwise angles between incident edges in $\Gamma$).  In the
following we describe several syndromes (implemented in our code) inspired by popular quality metrics and common
statistical tools.  The list is by no means exhaustive, nor do we claim syndromes below as necessary or independent.
Our model accepts any combination of syndromes; better choices remain to be systematically investigated.

\begin{explanation}
\explain{\enum{PRINVEC1} and \enum{PRINVEC2}} The two principal axes of the set $P$.  If we define a covariance matrix
$C=\{c_{ij}\}$, $c_{ij}=\frac{1}{n}\sum_{k=1}^n{(p_i^k-\overline{p_i})(p_j^k-\overline{p_j})})$, $i, j \in \{1, 2\}$,
where $\overline{p_i}=\frac{1}{n}\sum_{k=1}^n{p_i^k}$ are the mean values over each dimension, then \enum{PRINVEC1} and
\enum{PRINVEC2} will be its eigenvectors.

\explain{\enum{PRINCOMP1} and \enum{PRINCOMP2}} Projections of vertex positions onto $\vec{v}_1=\menum{PRINVEC1}$ and
$\vec{v}_2=\menum{PRINVEC2}$, that is, ~$\{\langle\left(\vec{p}^j-\overline{\vec{p}}\right),\vec{v}_i\rangle\}_{j=1}^n$
for $i\in\{1,2\}$ where $\langle\cdot,\cdot\rangle$ denotes the scalar product.

\explain{\enum{ANGULAR}} Let $A(v)$ denote the sequence of edges incident to a vertex $v$, appearing in a clockwise
order around it in $\Gamma$.  Let $\alpha(e_i,e_j)$ denote the clockwise angle between edges $e_i$ and $e_j$ incident to
the same vertex.  This syndrome is then defined as $\bigcup_{v\in{}V(G)}\{\alpha(e_i,e_j)\st{}e_i,e_j\text{ are
  consecutive in }A(v)\}$.

\explain{\enum{EDGE\_LENGTH}} $\bigcup_{(u,v)\in{}E(G)}\{\textrm{dist}_\Gamma(u,v)\}$ is the set of edge lengths in
$\Gamma$.

\explain{\enum{RDF\_GLOBAL}} $\bigcup_{u\neq{}v\in{}V(G)}\{\dist_\Gamma(u,v)\}$ contains distances between all vertices
in the drawing.  The concept of a \emph{radial distribution function (RDF)}~\cite{Findenegg2015} (the distribution of
\enum{RDF\_GLOBAL}) is borrowed from Statistical Physics and Crystallography and characterizes the regularity of
molecular structures.  In large graph layouts it captures regular, periodic and symmetric patterns in the vertex
positions.
\OnlyArxiv{%
  Fig.~\ref{app:fig:rdf-global} in the Appendix shows histograms of \enum{RDF\_GLOBAL} for some graphs and layouts.
  Note that more regular drawings feature better-isolated peaks in the respective histograms.
}

\explain{$\menum{RDF\_LOCAL}(d)$} $\bigcup_{u\neq{}v\in{}V(G)}\{\dist_\Gamma(u,v)\st\dist(u,v)\leq{}d\}$ is the set of
distances between vertices such that the graph-theoretical distance between them is bounded by $d\in\IntsN$.  In our
implementation, we compute $\menum{RDF\_LOCAL}(2^i)$ for $i\in\{0,\ldots,\ceil{\log_2(D)}\}$ where $D$ is the diameter
of $G$.  $\menum{RDF\_LOCAL}(d)$ in a sense interpolates between \enum{EDGE\_LENGTH} ($d=1$) and \enum{RDF\_GLOBAL}
($d\to\infty$).

\explain{\enum{TENSION}} $\bigcup_{u\neq{}v\in{}V(G)}\{\dist_\Gamma(u,v)/\dist_G(u,v)\}$ are the ratios of Euclidean and
graph-theoretical distances computed for all vertex pairs.  \enum{TENSION} is motivated by and is related to the
well-known stress function~\cite{Kamada1989}.
\end{explanation}

\noindent
Note that before computing the quality syndromes, we \emph{normalize} all layouts so that the center of gravity of $V$
is at the origin and the mean edge length is fixed in order to remove the effects of scaling and translation (but not
rotation).

% -*- coding:utf-8; mode:latex; -*-

% Copyright (C) 2018 Moritz Klammler <moritz.klammler@alumni.kit.edu>
% Copyright (C) 2018 Tamara Mchedlidze <mched@iti.uka.de>
% Copyright (C) 2018 Alexey Pak <alexey.pak@iosb.fraunhofer.de>
%
% This work is licensed under a Creative Commons Attribution-NonCommercial-NoDerivatives 4.0 International License
% (https://creativecommons.org/licenses/by-nc-nd/4.0/).

\section{Feature Vectors}
\label{sec:featex}

The sizes of quality syndromes are in general graph- and layout-dependent.  A neural network, however, requires a
fixed-size input.  A collection of syndromes is condensed to this \emph{feature vector} via \emph{feature extraction}.
Our approach to this step relies on several auxiliary definitions.  Let $S=\{x_i\}_{i=1}^p$ be a syndrome with $p$
entries.  By $S^\mu$ we denote the arithmetic mean and by $S^\rho$ the root mean square of $S$.  We also define a
\emph{histogram sequence} $S^\beta=\frac{1}{p}(S_1,\ldots,S_\beta)$ -- normalized counts in a histogram built over $S$
with $\beta$ bins.  The \emph{entropy}~\cite{Shannon1948} of $S^\beta$ is defined as
\begin{equation}
  \label{eqn:entropy}
  \Entropy(S^\beta) = -\sum_{i=1}^{p} \log_2(S_i) S_i
  \mathendpunct{.}
\end{equation}
We expect the entropy, as a measure of disorder, to be related to the aesthetic quality of a layout and convey important
information to the discriminator.

\begin{figure}[p]
  \begin{center}
    \InputLuatikzPlot{entropy-regression}
  \end{center}
  \caption{%
    Entropy $\mathcal{E}=\mathcal{E}(S^\beta)$ computed for histogram sequences $S^\beta$ defined for different numbers
    of histogram bins $\beta$.  Different markers (colors) correspond to several layouts of a regular grid-like graph,
    progressively distorted according to the parameter $r$.
    \OnlyArxiv{See Fig.~\ref{app:fig:worsening} in the Appendix for the examples of distorted grid layout.}
    The dependence of $\mathcal{E}$ on $\log_2(\beta)$ is well approximated by a linear function.  Both intercept and
    slope show a strong correlation with the levels of distortion $r$.
  }
  \label{fig:entropy-regression}
\end{figure}

The entropy $\Entropy(S^\beta)$ is sensitive to the number of bins $\beta$ (cf.~Fig.~\ref{fig:entropy-regression}).  In
order to avoid influencing the results via arbitrary choices of $\beta$, we compute it for $\beta=8,16,\ldots,512$.
After that, we perform a linear regression of $\Entropy(S^\beta)$ as a function of $\log_2(\beta)$.  Specifically, we
find $S^\eta$ and $S^\sigma$ such that $\sum_{\beta}(S^\sigma \log_2\beta+S^\eta-\Entropy(S^\beta))^2$ is minimized.
The parameters (intercept $S^\eta$ and slope $S^\sigma$) of this regression no longer depend on the histogram size and
are used as feature vector components.  Fig.~\ref{fig:entropy-regression} illustrates that the dependence of
$\Entropy(S^\beta)$ on $\log_2(\beta)$ is indeed often close to linear and the regression provides a decent
approximation.

A discrete histogram over $S$ can be generalized to a continuous \emph{sliding average}
\begin{equation}
  S^F(x) = \frac{\sum_{i=1}^{p} F(x,x_i)}{\int_{-\infty}^{+\infty} \diff{y} \sum_{i=1}^p F(y,x_i)}
  \mathendpunct{.}
\end{equation}
A natural choice for the kernel $F(x,y)$ is the Gaussian $F_\sigma(x,y)=\exp\left(-\frac{(x - y)^2}{2\sigma^2}\right)$.
By analogy to Eq.~\ref{eqn:entropy}, we may now define the \emph{differential entropy}~\cite{Shannon1948} as
\begin{equation}
  \Entropy*(S^{F_\sigma}) = -\int_{-\infty}^{+\infty} \diff{x}\log_2(S^{F_\sigma}(x)) \: S^{F_\sigma}(x)
  \mathendpunct{.}
\end{equation}
This entropy via kernel function still depends on parameter $\sigma$ (the filter width).  Computing
$\Entropy*(S^{F_\sigma})$ for multiple $\sigma$ values as we do for $\Entropy(S^\beta)$ is too expensive.  Instead, we
have found that using Scott's Normal Reference Rule~\cite{Scott1979} as a heuristic to fix $\sigma$ yields satisfactory
results, and allows us to define $S^\epsilon = \Entropy*(S^{F_\sigma})$.

Using these definitions, for the most complex syndrome $\menum{RDF\_LOCAL}(d)$ we introduce \enum{RDF\_LOCAL} -- a
$30$-tuple containing the arithmetic mean, root mean square and the differential entropy of $\menum{RDF\_LOCAL}(2^i)$
for $i\in(0,\ldots,9)$.  With that\footnote{Values $i < 10$ are sufficient as no graph in our dataset has a diameter
  exceeding $2^9$.}, $\menum{RDF\_LOCAL} = \left(\enum{RDF\_LOCAL}(2^i)^\mu, \enum{RDF\_LOCAL}(2^i)^\rho,
\enum{RDF\_LOCAL}(2^i)^\epsilon \right)_{i = 0}^9$.

Finally, we assemble the \ensuremath{\NNSharedInputDims}-dimensional\footnote{%
  The size is one less than expected from the explanation above because we do not include the arithmetic mean for
  \enum{EDGE\_LENGTH} as it is constant (due to the layout normalization mentioned earlier) and therefore
  non-informative.
} feature vector for a layout $\Gamma$ as
\begin{equation*}
  F_\mathrm{layout}(\Gamma) =
  \menum{PRINVEC1}\cup\menum{PRINVEC2}\cup\menum{RDF\_LOCAL}
  \cup\bigcup_{S}\left(S^\mu,S^\rho,S^\eta,S^\sigma\right)
\end{equation*}
where $S$ ranges over \enum{PRINCOMP1}, \enum{PRINCOMP2}, \enum{ANGULAR}, \enum{EDGE\_LENGTH}, \enum{RDF\_GLOBAL} and
\enum{TENSION}.

In addition, the discriminator model receives the trivial properties of the underlying graph as the second
\ensuremath{\NNTotalAuxHiddenDims}-dimensional vector $F_\mathrm{graph}(G)=(\log(n),\log(m))$.

% -*- coding:utf-8; mode:latex; -*-

% Copyright (C) 2018 Moritz Klammler <moritz.klammler@alumni.kit.edu>
% Copyright (C) 2018 Tamara Mchedlidze <mched@iti.uka.de>
% Copyright (C) 2018 Alexey Pak <alexey.pak@iosb.fraunhofer.de>
%
% This work is licensed under a Creative Commons Attribution-NonCommercial-NoDerivatives 4.0 International License
% (https://creativecommons.org/licenses/by-nc-nd/4.0/).

\section{Discriminator Model}
\label{sec:model}

Feature extractors such as those introduced in the previous section reduce an arbitrary graph $G$ and its arbitrary
layout $\Gamma$ to fixed-size vectors $F_\mathrm{graph}(G)$ and $F_\mathrm{layout}(\Gamma)$.  Given a graph $G$ and a
pair of its alternative layouts $\Gamma_a$ and $\Gamma_b$, the discriminator function $\DM$ receives the feature vectors
$\vec{v}_a=F_\mathrm{layout}(\Gamma_a)$, $\vec{v}_b=F_\mathrm{layout}(\Gamma_b)$ and $\vec{v}_G=F_\mathrm{graph}(G)$ and
outputs a scalar value
\begin{equation}
  t = \DM(\vec{v}_G,\vec{v}_a,\vec{v}_b) \in [-1, 1]
  \mathendpunct{.}
\end{equation}
The interpretation is as follows: if $t<0$, then the model believes that $\Gamma_a$ is \enquote{prettier} than
$\Gamma_b$; if $t>0$, then it prefers $\Gamma_b$.  Its magnitude $\abs{t}$ encodes the confidence level of the decision
(the higher $\abs{t}$, the more solid the answer).

For the implementation of the function $\DM$ we have chosen a practically convenient and flexible model structure known
as \emph{Siamese neural networks}, originally proposed by Bromley and others~\cite{Bromley1994} that is defined as
\begin{equation}
 \DM(\vec{v}_G,\vec{v}_a,\vec{v}_b) = \GM(\vec{\sigma}_a-\vec{\sigma}_b,\vec{v}_G)
\end{equation}
where $\vec{\sigma}_a=\SM(\vec{v}_a)$ and $\vec{\sigma}_b=\SM(\vec{v}_b)$.  The \emph{shared model} $\SM$ and the
\emph{global model} $\GM$ are implemented as multi-layer neural networks with a simple structure shown in
Fig.~\ref{fig:nn-structure}.  The network was implemented using the \emph{Keras}~\cite{Keras} framework with the
\emph{TensorFlow}~\cite{TensorFlow} library as back-end.

\begin{figure}
  \begin{center}
    % -*- coding:utf-8; mode:latex; -*-

% Copyright (C) 2018 Karlsruhe Institute of Technology
% Copyright (C) 2018 Moritz Klammler <moritz.klammler@alumni.kit.edu>
%
% Copying and distribution of this file, with or without modification, are permitted in any medium without royalty
% provided the copyright notice and this notice are preserved.  This file is offered as-is, without any warranty.

\begin{tikzpicture}
  [
    layer/.style = {
      rectangle,
      draw,
      minimum width = 15mm,
      minimum height = 4mm,
      text depth = 0.5ex,
      text height = 1.5ex,
      font = \scriptsize,
    },
    dense/.style = {
      layer,
      fill = TangoSkyBlue1!20,
    },
    operator/.style = {
      circle,
      draw,
      minimum size = 4mm,
      font = \scriptsize,
    },
    signal/.style = {
      draw = TangoSkyBlue3,
      thick,
      ->,
    },
    sigwd/.style = {
      font = \scriptsize,
      text = TangoSkyBlue3,
    },
    sigwdmark/.style = {
      draw = TangoSkyBlue3,
      very thin,
    },
    node distance = 5mm,
  ]

  \newcommand*{\NNUnknownDims}{?}
  \providecommand*{\NNSharedInputDims}{\NNUnknownDims}
  \providecommand*{\NNSharedHiddenDims}{\NNUnknownDims}
  \providecommand*{\NNSharedOutputDims}{\NNUnknownDims}
  \providecommand*{\NNTotalAuxInputDims}{\NNUnknownDims}
  \providecommand*{\NNTotalAuxHiddenDims}{\NNUnknownDims}
  \providecommand*{\NNTotalCatDims}{\NNUnknownDims}
  \providecommand*{\NNTotalOutputDims}{\NNUnknownDims}

  \coordinate (SM) at (0, 0);
  \coordinate (GM) at (7.5, -1);
  \coordinate (label) at (0, -2.25);

  \begin{scope}[shift = (SM), x = 13mm]

    \coordinate (in) at (0.25, 0);
    \coordinate (mid) at (4.75, 0);
    \node[layer, rotate = 90] (do1) at (1, 0) {dropout};
    \node[dense, rotate = 90] (l1) at (2, 0) {dense};
    \node[layer, rotate = 90] (do2) at (3, 0) {dropout};
    \node[dense, rotate = 90] (l2) at (4, 0) {dense};

    \draw[signal] (in) to node[name = a]{} (do1);
    \draw[signal] (do1) to node[name = b]{} (l1);
    \draw[signal] (l1) to node[name = c]{} (do2);
    \draw[signal] (do2) to node[name = d]{} (l2);
    \draw[signal] (l2) to node[name = e]{} (mid);

    \foreach \pos in {a, b, c, d, e} {
      \draw[sigwdmark] (\pos) +(-1pt, -2pt) -- +(+1pt, +2pt);
    }

    \node[sigwd, above = 0ex of a] {\ensuremath{\NNSharedInputDims}};
    \node[sigwd, above = 0ex of b] {\ensuremath{\NNSharedInputDims}};
    \node[sigwd, above = 0ex of c] {\ensuremath{\NNSharedHiddenDims}};
    \node[sigwd, above = 0ex of d] {\ensuremath{\NNSharedHiddenDims}};
    \node[sigwd, above = 0ex of e] {\ensuremath{\NNSharedOutputDims}};

    \node[anchor = east] at (in) {$\vec{v}$};
    \node[anchor = west] at (mid) {$\vec{\sigma}$};

    \coordinate (smlabel) at ($(in)!0.5!(mid)$);

  \end{scope}

  \begin{scope}[shift = (GM), x = 13mm]

    \coordinate (delta) at (0.75, 1.5);
    \coordinate (auxin) at (-0.25, 0);
    \coordinate (out) at (3.25, 1);

    \node[layer, rotate = 90, minimum width = 35mm] (cat) at (1.5, 1) {concatenation};
    \node[dense, rotate = 90] (aux) at (0.5, 0) {dense};
    \node[dense, rotate = 90] (l3) at (2.5, 1) {dense};

    \draw[signal] (delta) to node[name = f]{} (cat.north|-delta);
    \draw[signal] (auxin) to node[name = g]{} (aux);
    \draw[signal] (aux) to node[name = h]{} (cat.north|-aux);
    \draw[signal] (cat) to node[name = i]{} (l3);
    \draw[signal] (l3) to node[name = j]{} (out);

    \foreach \pos in {f, g, h, i, j} {
      \draw[sigwdmark] (\pos) +(-1pt, -2pt) -- +(+1pt, +2pt);
    }

    \node[sigwd, above = 0ex of f] {\ensuremath{\NNSharedOutputDims}};
    \node[sigwd, above = 0ex of g] {\ensuremath{\NNTotalAuxInputDims}};
    \node[sigwd, above = 0ex of h] {\ensuremath{\NNTotalAuxHiddenDims}};
    \node[sigwd, above = 0ex of i] {\ensuremath{\NNTotalCatDims}};
    \node[sigwd, above = 0ex of j] {\ensuremath{\NNTotalOutputDims}};

    \node[anchor = east] at (delta) {$\vec{\sigma}_a-\vec{\sigma}_b$};
    \node[anchor = east] at (auxin) {$\vec{v}_G$};
    \node[anchor = west] at (out) {$t$};

    \coordinate (gmlabel) at ($(auxin)!0.5!(out)$);

  \end{scope}

  \node at (smlabel|-label) {(a)};
  \node at (gmlabel|-label) {(b)};

\end{tikzpicture}

  \end{center}
  \caption{
    Structure of the neural networks $\SM(\vec{v})$ (a) and $\GM(\vec{\sigma}_a-\vec{\sigma}_b,\vec{v}_G)$ (b).  Shaded
    blocks denote standard network layers, and the numbers on the arrows denote the dimensionality of the respective
    representations.
  }
  \label{fig:nn-structure}
\end{figure}

The $\SM$ network (Fig.~\ref{fig:nn-structure}(a)) consists of two \enquote{dense} (fully-connected) layers, each
preceded by a \enquote{dropout} layer (discarding $50\percent$ and $25\percent$ of the signals, respectively).  Dropout
is a stochastic regularization technique intended to avoid overfitting that was first proposed by Srivastava and
others~\cite{Srivastava2014}.

In the $\GM$ network (Fig.~\ref{fig:nn-structure}(b)), the graph-related feature vector $\vec{v}_G$ is passed through an
auxiliary dense layer, and concatenated with the difference signal $(\vec{\sigma}_a-\vec{\sigma}_b)$ obtained from the
output vectors of $\SM$ for the two layouts.  The final dense layer produces the scalar output value.  The first and the
auxiliary layers use linear activation functions, the hidden layer uses $\ReLU$~\cite{Hahnloser2000} and the final layer
hyperbolic tangent activation.  Following the standard practice, the inputs to the network are normalized by subtracting
the mean and dividing by the standard deviation of the feature computed over the complete dataset.

In total, the $\DM$ model has \ensuremath{\NNTotalTrainableParams} free parameters, trained via stochastic gradient
descent-based optimization of the mean squared error (MSE) loss function.

% -*- coding:utf-8; mode:latex; -*-

% Copyright (C) 2018 Moritz Klammler <moritz.klammler@alumni.kit.edu>
% Copyright (C) 2018 Tamara Mchedlidze <mched@iti.uka.de>
% Copyright (C) 2018 Alexey Pak <alexey.pak@iosb.fraunhofer.de>
%
% This work is licensed under a Creative Commons Attribution-NonCommercial-NoDerivatives 4.0 International License
% (https://creativecommons.org/licenses/by-nc-nd/4.0/).

\section{Training and Testing Data}
\label{sec:data}

For training, all machine learning methods require datasets representing the variability of possible inputs.  Our $\DM$
model needs a dataset containing graphs, their layouts, and known aesthetic orderings of layout pairs.  We have
assembled such a dataset using two types of sources.  First, we used the collections of the well-known graph archives
\enum{ROME}, \enum{NORTH} and \enum{RANDDAG} which are published on \url{graphdrawing.org} as well as the NIST's
\enquote{Matrix Market}~\cite{MatrixMarket}.  \OnlyArxiv{See Fig.~\ref{app:fig:archives} in the Appendix for examples.}

Second, we have generated random graphs using the algorithms listed below.  As a by-product, some of them produce
layouts that stem naturally from the generation logic.  We refer to these as \emph{native} layouts (see~\cite{Moritz18}
for details).  \OnlyArxiv{Sample graphs with native layouts (where available) are shown in Fig.~\ref{app:fig:generators}
  in the Appendix.}

\begin{explanation}
\explain{\enum{GRID}} Regular $n\times{}m$ grids.  Native layouts: regular rectangular grids.

\explain{\enum{TORUS1}} Same as \enum{GRID}, but the first and the last \enquote{rows} are connected to form a $1$-torus
(a cylinder).  No native layouts.

\explain{\enum{TORUS2}} Same as \enum{TORUS1}, but also the first and the last \enquote{columns} are connected to form a
$2$-torus (a doughnut).  No native layouts.

\explain{\enum{LINDENMAYER}} Uses a stochastic L-system~\cite{Lindenmayer1990} to derive increasingly complex graphs by
performing random replacements of individual vertices with more complicated substructures such as an $n$-ring or an
$n$-clique.  \OnlyArxiv{Fig.~\ref{app:fig:lindenmayer-subgens} in the Appendix shows all the implemented replacement
  rules.} Produces a planar native layout.

\explain{\enum{QUASI\meta{n}D} for $n\in\{3,\ldots,6\}$} Projection of a primitive cubic lattice in an
$n$-di\-men\-sio\-nal space onto a $2$-dimensional plane intersecting that space at a random angle.  The native layout
follows from the construction.

\explain{\enum{MOSAIC1}} Starts with a regular polygon and randomly divides faces according to a set of simple rules
until the desired graph size is reached.  The rules include adding a vertex connected to all vertices of the face;
subdividing each edge and adding a vertex that connects to each subdivision vertex; subdividing each edge and connecting
them to a cycle.  \OnlyArxiv{These operations are visualized in Fig.~\ref{app:fig:mosaic-subgens} in the Appendix.} The
native layout follows from the construction.

\explain{\enum{MOSAIC2}} Applies a randomly chosen rule of \enum{MOSAIC1} to every face, with the goal of obtaining more
symmetric graphs.

\explain{\enum{BOTTLE}} Constructs a graph as a three-dimensional mesh over a random solid of revolution.  The native
layout is an axonometric projection.
\end{explanation}

\noindent
For each graph, we have computed force-directed layouts using the FM\textsuperscript{3}~\cite{Hachul2005} and
stress-minimization~\cite{Kamada1989} algorithms.  We assume these and native layouts to be generally aesthetically
pleasing and call them all \emph{proper} layouts of a graph.

Furthermore, we have generated \latinphrase{a priori} un-pleasing (\emph{garbage}) layouts as follows.  Given a graph
$G=(V,E)$, we generate a random graph $G'=(V',E')$ with $\card{V'}=\card{V}$ and $\card{E'}=\card{E}$ and compute a
force-directed layout for $G'$.  The coordinates found for the vertices $V'$ are then assigned to $V$.  We call these
\enquote{phantom} layouts due to the use of a \enquote{phantom} graph $G'$.  We find that phantom layouts look less
artificial than purely random layouts when vertex positions are sampled from a uniform or a normal distribution.  This
might be due to the fact that $G$ and $G'$ have the same density and share some beneficial aspects of the force-directed
method (such as mutual repelling of nodes).  \OnlyArxiv{See Fig.~\ref{app:fig:layouts} in the Appendix for the examples
  of regular and garbage layouts.}

For training and testing of the discriminator model we need a corpus of labeled pairs -- triplets
$(\Gamma_a,\Gamma_b,t)$ where $\Gamma_a$ and $\Gamma_a$ are two different layouts for the same graph and $t\in [-1,1]$
is a value indicating the relative aesthetic quality of $\Gamma_a$ and $\Gamma_b$.  A negative (positive) value for $t$
expresses that the quality of $\Gamma_a$ is superior (inferior) compared to $\Gamma_b$ and the magnitude of $t$
expresses the confidence of this prediction.  We only use pairs with sufficiently large $\abs{t}$.

As manually-labelled data were unavailable, we have fixed the values of $t$ as follows.  First, we paired a proper and a
garbage layout of a graph.  The assumption is that the former is always more pleasing (i.e.~$t=\pm1$).  Second, in order
to obtain more nuanced layout pairs and to increase the amount of data, we have employed the well-known technique of
\emph{data augmentation} as follows.

\paragraph{Layout Worsening:}
Given a proper layout $\Gamma$, we apply a transformation designed to gradually reduce its aesthetic quality that is
modulated by some parameter $r\in[0,1]$, resulting in a transformed layout $\Gamma'_r$.  By varying the degree $r$ of
the distortion, we may generate a sequence of layouts ordered by their anticipated aesthetic value: a layout with less
distortion is expected to be more pleasing than a layout with more distortion when starting from a presumably decent
layout.  We have implemented the following worsening techniques.  \enum{PERTURB}: add Gaussian noise to each node's
coordinates.  \enum{FLIP\_NODES}: swap coordinates of randomly selected node pairs.  \enum{FLIP\_EDGES}: same as
\enum{FLIP\_NODES} but restricted to connected node pairs.  \enum{MOVLSQ}: apply an affine deformation based on moving
least squares suggested (although for a different purpose) by Schaefer et al.~\cite{Schaefer2006}.  In essence, all
vertices are shifted according to some smoothly varying coordinate mapping.  \OnlyArxiv{Illustrations of these worsening
  algorithms can be found in Fig.~\ref{app:fig:worsening} in the Appendix}.

\paragraph{Layout Interpolation:}
As the second data augmentation technique, we linearly interpolated the positions of corresponding vertices between the
proper and garbage layouts of the same graph.  The resulting label $t$ is then proportional to the difference in the
interpolation parameter.

\par\bigskip\noindent
In total, using all the methods described above, we have been able to collect a database of about
\ensuremath{\NNCorpusSizeApprox} labeled layout pairs.

% -*- coding:utf-8; mode:latex; -*-

% Copyright (C) 2018 Moritz Klammler <moritz.klammler@alumni.kit.edu>
% Copyright (C) 2018 Tamara Mchedlidze <mched@iti.uka.de>
% Copyright (C) 2018 Alexey Pak <alexey.pak@iosb.fraunhofer.de>
%
% This work is licensed under a Creative Commons Attribution-NonCommercial-NoDerivatives 4.0 International License
% (https://creativecommons.org/licenses/by-nc-nd/4.0/).

\section{Evaluation}
\label{sec:eval}

The performance of the discriminator model was evaluated using \emph{cross-validation} with $\XValTestRuns$-fold random
subsampling~\cite{Kohavi1995}.  In each round, $20\percent$ of graphs (with all their layouts) were chosen randomly and
were set aside for testing, and the model was trained using the remaining layout pairs.  Of $N$ labeled pairs used for
testing, in each round we computed the number $N_\mathrm{correct}$ of pairs for which the model properly predicted the
aesthetic preference, and derived the accuracy (success rate) $A=N_\mathrm{correct}/N$.  The standard deviation of $A$
over the $\XValTestRuns$ runs was taken as the uncertainty of the results.  With the average number of test samples of
$N=\XValCountApprox$, the eventual success rate was $\boldsymbol{A=(\XValSuccessMean\pm\XValSuccessStdev)\percent}$.

\subsection{Comparison With Other Metrics}

In order to assess the relative standing of the suggested method, we have implemented two known aesthetic metrics
(\emph{stress} and the \emph{combined metric} by Huang et al.~\cite{HuangHL16}) and evaluated them over the same
dataset.  The metric values were trivially converted to the respective discriminator function outputs.

Stress $\stress$ of a layout $\Gamma$ of a simple connected graph $G=(V,E)$ was defined by Kamada and
Kawai~\cite{Kamada1989} as
\begin{equation}
  \stress(\Gamma) = \sum_{i=1}^{n-1} \sum_{j=i+1}^{n}
  k_{ij} \left( \dist_\Gamma(v_i,v_j) - L \dist_G(v_i,v_j) \right)^2
  \mathendpunct{,}
  \label{eq:stress}
\end{equation}
where $L$ denotes the desirable edge length and $k_{ij}=K/\dist_G(v_i,v_j)^2$ is the strength of a \enquote{spring}
attached to $v_i$ and $v_j$.  The constant $K$ is irrelevant in the context of discriminator functions and can be set to
any value.

As observed by Welch and Kobourov~\cite{Welch2017}, the numeric value of stress depends on the layout scale via the
constant $L$ in the Eq.~\ref{eq:stress} which complicates comparisons.  Their suggested solution was for each layout to
find $L$ that minimizes $\stress$ (e.g.~using binary search).  In our implementation, we applied a similar technique
based on fitting and minimizing a quadratic function to the stress computed at three scales.  We refer to this quantity
as \enum{STRESS}.

The combined metric proposed by Huang et al.~\cite{HuangHL16} (referred to as \enum{COMB}) is a weighted average of four
simpler quality metrics: the number of edge crossings (\enum{CC}), the minimum crossing angle between any two edges in
the drawing (\enum{CR}), the minimum angle between two adjacent edges (\enum{AR}), and the standard deviation computed
over all edge lengths (\enum{EL}).

The average is computed over the so-called $z$-scores of the above metrics.  Each $z$-score is found by subtracting the
mean and dividing by the standard deviation of the metric for all layouts of a given graph to be compared with each
other.  More formally, let $G$ be a graph and $\Gamma_1,\ldots,\Gamma_k$ be its $k$ layouts to be compared pairwise.
Let $M(\Gamma_i)$ be the value of metric $M$ for $\Gamma_i$ and $\mu_M$ and $\sigma_M$ be the mean and the standard
deviation of $M(\Gamma_i)$ for $i\in\{1,\ldots,k\}$.  Then
\begin{equation}
  z_M^{(i)} = \frac{M(\Gamma_i) - \mu_M}{\sigma_M}
\end{equation}
is the $z$-score for metric $M$ and layout $\Gamma_i$.  The combined metric then is
\begin{equation}
  \menum{COMB}(\Gamma_j) = \sum_{M} w_M \: z_{M}^{(j)}
  \mathendpunct{.}
\end{equation}
The weights $w_M$ were found via Nelder-Mead maximization~\cite{Press2007} of the prediction accuracy over the training
dataset\footnote{The obtained weights are: \TextualHuangWeights}.

\begin{figure}[p]
  \csdef{Status0}{\textcolor{TangoScarletRed2}{\ding{55}}}%
  \csdef{Status1}{\textcolor{TangoChameleon2}{\ding{51}}}%
  \newcommand*{\ShowIt}[1]{\InputTikzGraph{0.25\textwidth}{#1}}
  \newcommand*{\TellIt}[3]{%
    \begin{tabular}{l@{\quad}c}
      \enum{DISC\_MODEL} & \csuse{Status#1}\\
      \enum{STRESS}      & \csuse{Status#2}\\
      \enum{COMB}        & \csuse{Status#3}\\
    \end{tabular}
  }
  \begin{center}
    \begin{tabular}{>{\centering}m{0.35\textwidth}>{\centering}m{0.35\textwidth}@{\qquad}m{0.2\textwidth}}
      \ShowIt{8314f2c1-4d644640-more} & \ShowIt{8314f2c1-bac12550-less} & \TellIt{1}{0}{1}\\[1ex]\\[1ex]
      \ShowIt{d8c1498f-8c5ad49b-more} & \ShowIt{d8c1498f-4c2beafc-less} & \TellIt{1}{0}{0}\\[1ex]\\[1ex]
      \ShowIt{5b3b66d2-49e8882a-more} & \ShowIt{5b3b66d2-f60fcc04-less} & \TellIt{1}{1}{0}\\
    \end{tabular}
  \end{center}
  \caption{%
    Examples where our discriminator model (\enum{DISC\_MODEL}) succeeds (\csuse{Status1}) and the competing metrics
    fail (\csuse{Status0}) to predict the answer correctly.  In each row, the layout on the left is expected to be
    superior compared to the one on the right.
  }
  \label{fig:model-comp}
\end{figure}

The accuracy of the stress-based and the combined model-based discriminators is shown in
Tab.~\ref{tab:competing-metrics}.  In most cases, our model outperforms these algorithms by a comfortable margin.
Fig.~\ref{fig:model-comp} provides examples of mis-predictions.  By inspecting such cases, we notice that \enum{STRESS}
often fails to guess the aesthetics of (almost) planar layouts that contain both very short and very long edges (such
behavior may also be inferred from the definition of \enum{STRESS}).  We observe that there are planar graphs, such as
nested triangulations, for which this property is unavoidable in planar drawings.  The mis-predictions of \enum{COMB}
seem to be due to the high weight of the edge length metric \enum{EL}.  Both \enum{STRESS} and \enum{COMB} are weaker
than our model in capturing the absolute symmetry and regularity of layouts.

\begin{table}[p]
  \begin{center}
    \InputCompetingMetrics{eval-competing-metrics.tex}
  \end{center}
  \caption{%
    Accuracy scores for the \enum{COMB} and \enum{STRESS} model.  The standard deviation in each column is estimated
    based on the $5$-fold cross-validation (using $20\percent$ of data for testing each time).  The \enquote{Advantage}
    column shows the improvement in the accuracy of our model with respect to the alternative metric.
  }
  \label{tab:competing-metrics}
\end{table}

\subsection{Significance of Individual Syndromes}

In order to estimate the influence of individual syndromes on the final result, we have tested several modifications of
our model.  For each syndrome, we considered the case when the feature vector contained only that syndrome.  In the
second case, that syndrome was removed from the original feature vector.  The entries for the omitted features were set
to zero.  The results are shown in Tab.~\ref{tab:eval-puncture}.

\begin{table}
  \begin{center}
    \InputPunctureResult[eval-cross-valid.tex]{eval-puncture.tex}
  \end{center}
  \caption{%
    Success rates of our discriminator when a syndrome is excluded from the feature vector, and when the feature vector
    contains only that a syndrome.  Note that \enum{RDF\_LOCAL} is a family of syndromes that are all included or
    excluded together.  The apparent paradox of higher success rates when some syndromes are excluded can be explained
    by a statistical fluctuation and is well within the listed range of uncertainty.
  }
  \label{tab:eval-puncture}
\end{table}

As can be observed, the dominant contribution to the accuracy of the model is due to the RDF-based properties
\enum{RDF\_LOCAL} and \enum{RDF\_GLOBAL}.  The exclusion of other syndromes does not significantly change the results
(they agree within the estimated uncertainty).  However, the sole inclusion of these syndromes still performs better
than random choice.  This suggests that there is a considerable overlap between the aesthetic aspects captured by
various syndromes.  Further analysis is needed to identify the nature and the magnitude of these correlations.

% -*- coding:utf-8; mode:latex; -*-

% Copyright (C) 2018 Moritz Klammler <moritz.klammler@alumni.kit.edu>
% Copyright (C) 2018 Tamara Mchedlidze <mched@iti.uka.de>
% Copyright (C) 2018 Alexey Pak <alexey.pak@iosb.fraunhofer.de>
%
% This work is licensed under a Creative Commons Attribution-NonCommercial-NoDerivatives 4.0 International License
% (https://creativecommons.org/licenses/by-nc-nd/4.0/).

\section{Conclusion}
\label{sec:conclusion}

In this paper we propose a machine learning-based discriminator model that selects the more aesthetically pleasing
drawing from a pair of graph layouts.  Our model picks the \enquote{better} layout in more than $96\percent$ cases and
outperforms known stress-based and linear combination-based models.  To the best of our knowledge, this is the first
application of machine learning methods to this question.  Previously, such techniques have proven successful in a range
of complex issues involving aesthetics, prior knowledge, and unstated rules in object recognition, industrial design,
and digital arts.  As our model uses a simple network architecture, investigating the performance of more complex
networks is warranted.

Previous efforts were focused on determining the aesthetic quality of a layout as a weighted average of individual
quality metrics.  We extend these ideas and findings in the sense that we do not assume any particular form of
dependency between the overall aesthetic quality and the individual quality metrics.

Going beyond simple quality metrics, we define quality syndromes that capture arrays of information about graphs and
layouts.  In particular, we borrow the notion of RDF from Statistical Physics and Crystallography; RDF-based features
demonstrate the strongest potential in extracting the aesthetic quality of a layout.  We expect RDFs (describing the
microscopic structure of materials) to be the most relevant for large graphs.  It is tempting to investigate whether
further tools from physics can be useful in capturing drawing aesthetics.

From multiple syndromes, we construct fixed-size feature vectors using common statistical tools.  Our feature vector
does not contain any information on crossings or crossing angles, nevertheless its performance is superior with respect
to the weighted averages-based model which accounts for both.  It would be interesting to investigate whether including
these and other features further improves the performance of the neural network-based model.

In order to train and evaluate the model, we have assembled a relatively large corpus of labeled pairs of layouts, using
available and generated graphs and exploiting the assumption that layouts produced by force-directed algorithms and
native graph layouts are aesthetically pleasing and that disturbing them reduces the aesthetic quality.  We admit that
this study should ideally be repeated with human-labeled data.  However, this requires that a dataset be collected with
a size similar to ours, which is a challenging task.  Creating such a dataset may become a critically important
accomplishment in the graph drawing field.


\bibliographystyle{splncs04}
\bibliography{literature-bibtex}

\ifforarxiv
  \clearpage
  \appendix\relax
  % -*- coding:utf-8; mode:latex; -*-

% Copyright (C) 2018 Moritz Klammler <moritz.klammler@alumni.kit.edu>
% Copyright (C) 2018 Tamara Mchedlidze <mched@iti.uka.de>
% Copyright (C) 2018 Alexey Pak <alexey.pak@iosb.fraunhofer.de>
%
% This work is licensed under a Creative Commons Attribution-NonCommercial-NoDerivatives 4.0 International License
% (https://creativecommons.org/licenses/by-nc-nd/4.0/).

\section*{Appendix -- Supplementary Figures and Tables}

\begin{figure}[h!]
  \begin{center}
    \begin{tabular}{c@{\qquad}c@{\qquad}c}
      \InputTikzGraph{30mm}{demograph-a}&
      \InputTikzGraph{30mm}{demograph-b}&
      \InputTikzGraph{30mm}{demograph-c}\\[3ex]
      \InputLuatikzPlot{rdf-global-a}&
      \InputLuatikzPlot{rdf-global-b}&
      \InputLuatikzPlot{rdf-global-c}\\[2ex]
      \InputLuatikzPlot{angular-a}&
      \InputLuatikzPlot{angular-b}&
      \InputLuatikzPlot{angular-c}
    \end{tabular}
  \end{center}
  \caption{%
    Illustration of the syndromes \enum{RDF\_GLOBAL} and \enum{ANGULAR}.  Upper row, from left to right: proper,
    distorted layouts of a regular grid, and a force-directed layout of an irregular graph (\enquote{power grid}).
    Central row: smoothed relative frequency distributions for the \enum{RDF\_GLOBAL} syndromes computed for the
    respective layouts in the upper row.  The isolated peaks in the leftmost distribution correspond to characteristic
    distances in the lattice.  In the central plot, these peaks are widened due to random distortion. In the rightmost
    plot, no regular structure can be identified.  Lower row: smoothed relative frequency distributions for the
    \enum{ANGULAR} syndromes.  The leftmost plot clearly shows the dominance of angles proportional to $\pi/2$.  In the
    rightmost plot, the distinctive peak at $\phi = 2\pi$ corresponds to the large number of degree one vertices.
  }
  \label{app:fig:rdf-global}
\end{figure}
\clearpage

\begin{figure}[bp]
  \begin{center}
    \begin{tabular}{c@{\qquad}c@{\qquad}c}
      \InputTikzGraph{0.25\textwidth}{rome}&
      \InputTikzGraph{0.25\textwidth}{north}&
      \InputTikzGraph{0.25\textwidth}{randdag}\\[2ex]
      \enum{ROME} & \enum{NORTH} & \enum{RANDDAG}
    \end{tabular}
    \par\vspace{1cm}
    \begin{tabular}{c@{\qquad\qquad}c}
      \InputTikzGraph{0.35\textwidth}{bcspwr}&
      \InputTikzGraph{0.35\textwidth}{grenoble}\\[2ex]
      \enum{BCSPWR} & \enum{GRENOBLE}
    \end{tabular}
    \par\vspace{1cm}
    \begin{tabular}{c@{\qquad\qquad}c}
      \InputTikzGraph{0.35\textwidth}{psadmit}&
      \InputTikzGraph{0.35\textwidth}{smtape}\\[2ex]
      \enum{PSADMIT} & \enum{SMTAPE}
    \end{tabular}
  \end{center}
  \caption{%
    Examples of imported graphs.  The \enum{BCSPWR}, \enum{GRENOBLE}, \enum{PSADMIT} and \enum{SMTAPE} graphs come from
    the respective datasets in the Harwell-Boeing collection in NIST's \enquote{Matrix Market}~\cite{MatrixMarket}.  All
    graphs are visualized using the FM\textsuperscript{3} algorithm.
  }
  \label{app:fig:archives}
\end{figure}

\begin{figure}[bp]
  \begin{center}
    \begin{tabular}{c@{\qquad}c@{\qquad}c@{\qquad}c}
      \InputTikzGraph{0.2\textwidth}{grid}&
      \InputTikzGraph{0.2\textwidth}{torus1}&
      \InputTikzGraph{0.2\textwidth}{torus2}&
      \InputTikzGraph{0.2\textwidth}{bottle}\\[2ex]
      \enum{GRID} & \enum{TORUS1} & \enum{TORUS2} & \enum{BOTTLE}
    \end{tabular}
    \par\vspace{1cm}
    \begin{tabular}{c@{\qquad}c@{\qquad}c@{\qquad}c}
      \InputTikzGraph{0.2\textwidth}{quasi3d}&
      \InputTikzGraph{0.2\textwidth}{quasi4d}&
      \InputTikzGraph{0.2\textwidth}{quasi5d}&
      \InputTikzGraph{0.2\textwidth}{quasi6d}\\[2ex]
      \enum{QUASI3D} & \enum{QUASI4D} & \enum{QUASI5D} & \enum{QUASI6D}
    \end{tabular}
    \par\vspace{1cm}
    \begin{tabular}{c@{\qquad}c@{\qquad}c}
      \InputTikzGraph{0.25\textwidth}{lindenmayer}&
      \InputTikzGraph{0.25\textwidth}{mosaic1}&
      \InputTikzGraph{0.25\textwidth}{mosaic2}\\[2ex]
      \enum{LINDENMAYER} & \enum{MOSAIC1} & \enum{MOSAIC2}
    \end{tabular}
  \end{center}
  \caption{%
    Examples of generated graphs labeled by the respective generators.  \enum{GRID}, \enum{LINDENMAYER},
    \enum{QUASI\meta{$n$}D}, \enum{MOSAIC1}, \enum{MOSAIC2} and \enum{BOTTLE} layouts are native.  \enum{TORUS1} and
    \enum{TORUS2} are visualized with the stress-minimization algorithm.
  }
  \label{app:fig:generators}
\end{figure}

\begin{figure}[p]
  \begin{center}
    \begin{tabular}{c@{\qquad\qquad}c}
      ../../report/pics/gen-lindenmayer-singleton.tex & ../../report/pics/gen-lindenmayer-star.tex\\
      $\menum{L/SINGLETON}$ & $\menum{L/STAR}_3(v)$\\[2ex]
      ../../report/pics/gen-lindenmayer-wheel.tex & % -*- coding:utf-8; mode:latex; -*- %

% Copyright (C) 2018 Moritz Klammler <moritz.klammler@student.kit.edu>
%
% Copying and distribution of this file, with or without modification, are permitted in any medium without royalty
% provided the copyright notice and this notice are preserved.  This file is offered as-is, without any warranty.

\begin{tikzpicture}[scale = 0.5]

  \path[use as bounding box] (-4.5, -4.5) rectangle (4.5, 4.5);

  \node[Vertex] (u1) at (  0:4) {$u_1$};
  \node[Vertex] (u2) at ( 90:4) {$u_2$};
  \node[Vertex] (u3) at (180:4) {$u_3$};
  \node[Vertex] (u4) at (270:4) {$u_4$};

  \node[vertex] (w1)  at (  0:2) {};
  \node[vertex] (w2)  at ( 30:2) {};
  \node[vertex] (w3)  at ( 60:2) {};
  \node[vertex] (w4)  at ( 90:2) {};
  \node[vertex] (w5)  at (120:2) {};
  \node[vertex] (w6)  at (150:2) {};
  \node[vertex] (w7)  at (180:2) {};
  \node[vertex] (w8)  at (210:2) {};
  \node[vertex] (w9)  at (240:2) {};
  \node[vertex] (w10) at (270:2) {};
  \node[vertex] (w11) at (300:2) {};
  \node[vertex] (w12) at (330:2) {};

  \draw[edge] (w1) -- (w2) -- (w3) -- (w4) -- (w5) -- (w6) -- (w7) -- (w8) -- (w9) -- (w10) -- (w11) -- (w12) -- (w1);

  \draw[edge] (w1)  -- (u1);
  \draw[edge] (w4)  -- (u2);
  \draw[edge] (w7)  -- (u3);
  \draw[edge] (w10) -- (u4);

\end{tikzpicture}
\\
      $\menum{L/WHEEL}_3(v)$ & $\menum{L/RING}_3(v)$\\[2ex]
      ../../report/pics/gen-lindenmayer-clique.tex & % -*- coding:utf-8; mode:latex; -*- %

% Copyright (C) 2018 Moritz Klammler <moritz.klammler@student.kit.edu>
%
% Copying and distribution of this file, with or without modification, are permitted in any medium without royalty
% provided the copyright notice and this notice are preserved.  This file is offered as-is, without any warranty.

\begin{tikzpicture}[scale = 0.5]

  \path[use as bounding box] (-4.5, -4.5) rectangle (4.5, 4.5);

  \begin{scope}[scale = 1.5]
    \draw[edge] (-2, -1) grid[step = 1] (+2, +1);
    \foreach \i in {-1, 0, +1} {
      \foreach \j in {-2, -1, 0, +1, +2} {
        \node[vertex] at (\j, \i) {};
      }
    }
    \node[Vertex, fill = white] (v) at (0, 0) {$v$};
  \end{scope}

\end{tikzpicture}
\\
      $\menum{L/CLIQUE}_2(v)$ & $\menum{L/GRID}_{3,5}(v)$
    \end{tabular}
  \end{center}
  \caption{%
    Illustration of the \enum{LINDENMAYER} generator operations.  A degree four vertex may be replaced by any of the
    above subgraphs, except for the bottom right subgraph which replaces a degree zero vertex.
  }
  \label{app:fig:lindenmayer-subgens}
\end{figure}

\begin{figure}[p]
  \begin{center}
    \newcommand*{\GenMosaicScale}{0.5}
    \begin{tabular}{c@{\quad}c@{\quad}c}
      ../../report/pics/gen-mosaic-star.tex&
      % -*- coding:utf-8; mode:latex; -*- %

% Copyright (C) 2018 Moritz Klammler <moritz.klammler@student.kit.edu>
%
% Copying and distribution of this file, with or without modification, are permitted in any medium without royalty
% provided the copyright notice and this notice are preserved.  This file is offered as-is, without any warranty.

\providecommand*{\GenMosaicScale}{0.6}
\begin{tikzpicture}[scale = \GenMosaicScale, rotate = 18]

  \node[Vertex] (u1) at (  0:3) {$u_1$};
  \node[Vertex] (u2) at ( 72:3) {$u_2$};
  \node[Vertex] (u3) at (144:3) {$u_3$};
  \node[Vertex] (u4) at (216:3) {$u_4$};
  \node[Vertex] (u5) at (288:3) {$u_5$};

  \node[Vertex] (w1) at ($(u1)!0.5!(u2)$) {$w_1$};
  \node[Vertex] (w2) at ($(u2)!0.5!(u3)$) {$w_2$};
  \node[Vertex] (w3) at ($(u3)!0.5!(u4)$) {$w_3$};
  \node[Vertex] (w4) at ($(u4)!0.5!(u5)$) {$w_4$};
  \node[Vertex] (w5) at ($(u5)!0.5!(u1)$) {$w_5$};

  \node[Vertex] (v) at (0, 0) {$v$};

  \draw[edge] (u1) -- (w1) -- (u2) -- (w2) -- (u3) -- (w3) -- (u4) -- (w4) -- (u5) -- (w5) -- (u1);

  \draw[edge] (v) -- (w1);
  \draw[edge] (v) -- (w2);
  \draw[edge] (v) -- (w3);
  \draw[edge] (v) -- (w4);
  \draw[edge] (v) -- (w5);

\end{tikzpicture}
&
      ../../report/pics/gen-mosaic-shape.tex\\[1ex]
      \enum{M/STAR} & \enum{M/FLOWER} & \enum{M/SHAPE}
    \end{tabular}
  \end{center}
  \caption{%
    Operations of the \enum{MOSAIC} generator on a pentagonal facet $\{u_1,\ldots,u_5\}$.
  }
  \label{app:fig:mosaic-subgens}
\end{figure}

\begin{figure}[p]
  \begin{center}
    \begin{tabular}{c@{\qquad\qquad}c@{\qquad\qquad}c}
      \InputTikzGraph{0.2\textwidth}{native}&
      \InputTikzGraph{0.2\textwidth}{fmmm}&
      \InputTikzGraph{0.2\textwidth}{stress}\\[2ex]
      \enum{NATIVE} & \enum{FMMM} & \enum{STRESS}
    \end{tabular}
    \par\vspace{1cm}
    \begin{tabular}{c@{\qquad\qquad}c@{\qquad\qquad}c}
      \InputTikzGraph{0.2\textwidth}{random-uniform}&
      \InputTikzGraph{0.2\textwidth}{random-normal}&
      \InputTikzGraph{0.2\textwidth}{phantom}\\[2ex]
      \enum{RANDOM\_UNIFORM} & \enum{RANDOM\_NORMAL} & \enum{PHANTOM}
    \end{tabular}
  \end{center}
  \caption{%
    Examples of different layouts for the same graph.  \enum{RANDOM\_UNIFORM}, \enum{RANDOM\_NORMAL} are random layouts
    where vertex positions are sampled from the uniform and the normal distributions, respectively.
  }
  \label{app:fig:layouts}
\end{figure}

\begin{figure}[p]
  \begin{center}
    \begin{tabular}{lc@{\quad}c@{\quad}c@{\quad}c}
      \rotatebox{90}{\enum{PERTURB}}&
      \InputTikzGraph{0.2\textwidth}{perturb-00000}&
      \InputTikzGraph{0.2\textwidth}{perturb-01500}&
      \InputTikzGraph{0.2\textwidth}{perturb-05000}&
      \InputTikzGraph{0.2\textwidth}{perturb-10000}\\[2ex]
      \rotatebox{90}{\enum{FLIP\_NODES}}&
      \InputTikzGraph{0.2\textwidth}{flip-nodes-00000}&
      \InputTikzGraph{0.2\textwidth}{flip-nodes-01500}&
      \InputTikzGraph{0.2\textwidth}{flip-nodes-05000}&
      \InputTikzGraph{0.2\textwidth}{flip-nodes-10000}\\[2ex]
      \rotatebox{90}{\enum{FLIP\_EDGES}}&
      \InputTikzGraph{0.2\textwidth}{flip-edges-00000}&
      \InputTikzGraph{0.2\textwidth}{flip-edges-01500}&
      \InputTikzGraph{0.2\textwidth}{flip-edges-05000}&
      \InputTikzGraph{0.2\textwidth}{flip-edges-10000}\\[2ex]
      \rotatebox{90}{\enum{MOVLSQ}}&
      \InputTikzGraph{0.2\textwidth}{movlsq-00000}&
      \InputTikzGraph{0.2\textwidth}{movlsq-01500}&
      \InputTikzGraph{0.2\textwidth}{movlsq-05000}&
      \InputTikzGraph{0.2\textwidth}{movlsq-10000}\\[2ex]
      & $r=0\percent$ & $r=15\percent$ & $r=50\percent$ & $r=100\percent$
    \end{tabular}
  \end{center}
  \caption{%
    Examples of applying  different layout worsening techniques at different rates.
  }
  \label{app:fig:worsening}
\end{figure}

\begin{figure}[p]
  \begin{center}
    \begin{tabular}{c@{\qquad}c@{\qquad}c@{\qquad}c}
      \InputTikzGraph{0.2\textwidth}{linear-00000}&
      \InputTikzGraph{0.2\textwidth}{linear-02500}&
      \InputTikzGraph{0.2\textwidth}{linear-07500}&
      \InputTikzGraph{0.2\textwidth}{linear-10000}\\[2ex]
      $r=0\percent$ & $r=25\percent$ & $r=75\percent$ & $r=100\percent$
    \end{tabular}
  \end{center}
  \caption{%
    Example of linear interpolation between a proper and a garbage layout.
  }
  \label{app:fig:interpolating}
\end{figure}

\fi

\end{document}
