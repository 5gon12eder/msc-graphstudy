% -*- coding:utf-8; mode:latex; -*-

% Copyright (C) 2018 Moritz Klammler <moritz.klammler@alumni.kit.edu>
% Copyright (C) 2018 Tamara Mchedlidze <mched@iti.uka.de>
% Copyright (C) 2018 Alexey Pak <alexey.pak@iosb.fraunhofer.de>
%
% This work is licensed under a Creative Commons Attribution-NonCommercial-NoDerivatives 4.0 International License
% (https://creativecommons.org/licenses/by-nc-nd/4.0/).

\section{Discriminator Model}
\label{sec:model}

Feature extractors such as those introduced in the previous section reduce an arbitrary graph $G$ and its arbitrary
layout $\Gamma$ to fixed-size vectors $F_\mathrm{graph}(G)$ and $F_\mathrm{layout}(\Gamma)$.  Given a graph $G$ and a
pair of its alternative layouts $\Gamma_a$ and $\Gamma_b$, the discriminator function $\DM$ receives the feature vectors
$\vec{v}_a=F_\mathrm{layout}(\Gamma_a)$, $\vec{v}_b=F_\mathrm{layout}(\Gamma_b)$ and $\vec{v}_G=F_\mathrm{graph}(G)$ and
outputs a scalar value
\begin{equation}
  t = \DM(\vec{v}_G,\vec{v}_a,\vec{v}_b) \in [-1, 1]
  \mathendpunct{.}
\end{equation}
The interpretation is as follows: if $t<0$, then the model believes that $\Gamma_a$ is \enquote{prettier} than
$\Gamma_b$; if $t>0$, then it prefers $\Gamma_b$.  Its magnitude $\abs{t}$ encodes the confidence level of the decision
(the higher $\abs{t}$, the more solid the answer).

For the implementation of the function $\DM$ we have chosen a practically convenient and flexible model structure known
as \emph{Siamese neural networks}, originally proposed by Bromley and others~\cite{Bromley1994} that is defined as
\begin{equation}
 \DM(\vec{v}_G,\vec{v}_a,\vec{v}_b) = \GM(\vec{\sigma}_a-\vec{\sigma}_b,\vec{v}_G)
\end{equation}
where $\vec{\sigma}_a=\SM(\vec{v}_a)$ and $\vec{\sigma}_b=\SM(\vec{v}_b)$.  The \emph{shared model} $\SM$ and the
\emph{global model} $\GM$ are implemented as multi-layer neural networks with a simple structure shown in
Fig.~\ref{fig:nn-structure}.  The network was implemented using the \emph{Keras}~\cite{Keras} framework with the
\emph{TensorFlow}~\cite{TensorFlow} library as back-end.

\begin{figure}
  \begin{center}
    ../../paper/pics/discriminator.tex
  \end{center}
  \caption{
    Structure of the neural networks $\SM(\vec{v})$ (a) and $\GM(\vec{\sigma}_a-\vec{\sigma}_b,\vec{v}_G)$ (b).  Shaded
    blocks denote standard network layers, and the numbers on the arrows denote the dimensionality of the respective
    representations.
  }
  \label{fig:nn-structure}
\end{figure}

The $\SM$ network (Fig.~\ref{fig:nn-structure}(a)) consists of two \enquote{dense} (fully-connected) layers, each
preceded by a \enquote{dropout} layer (discarding $50\percent$ and $25\percent$ of the signals, respectively).  Dropout
is a stochastic regularization technique intended to avoid overfitting that was first proposed by Srivastava and
others~\cite{Srivastava2014}.

In the $\GM$ network (Fig.~\ref{fig:nn-structure}(b)), the graph-related feature vector $\vec{v}_G$ is passed through an
auxiliary dense layer, and concatenated with the difference signal $(\vec{\sigma}_a-\vec{\sigma}_b)$ obtained from the
output vectors of $\SM$ for the two layouts.  The final dense layer produces the scalar output value.  The first and the
auxiliary layers use linear activation functions, the hidden layer uses $\ReLU$~\cite{Hahnloser2000} and the final layer
hyperbolic tangent activation.  Following the standard practice, the inputs to the network are normalized by subtracting
the mean and dividing by the standard deviation of the feature computed over the complete dataset.

In total, the $\DM$ model has \ensuremath{\NNTotalTrainableParams} free parameters, trained via stochastic gradient
descent-based optimization of the mean squared error (MSE) loss function.
