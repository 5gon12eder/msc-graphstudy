% -*- coding:utf-8; mode:latex; -*-

% Copyright (C) 2018 Moritz Klammler <moritz.klammler@alumni.kit.edu>
% Copyright (C) 2018 Tamara Mchedlidze <mched@iti.uka.de>
% Copyright (C) 2018 Alexey Pak <alexey.pak@iosb.fraunhofer.de>
%
% This work is licensed under a Creative Commons Attribution-NonCommercial-NoDerivatives 4.0 International License
% (https://creativecommons.org/licenses/by-nc-nd/4.0/).

\section*{Appendix -- Supplementary Figures and Tables}

\begin{figure}[h!]
  \begin{center}
    \begin{tabular}{c@{\qquad}c@{\qquad}c}
      \InputTikzGraph{30mm}{demograph-a}&
      \InputTikzGraph{30mm}{demograph-b}&
      \InputTikzGraph{30mm}{demograph-c}\\[3ex]
      \InputLuatikzPlot{rdf-global-a}&
      \InputLuatikzPlot{rdf-global-b}&
      \InputLuatikzPlot{rdf-global-c}\\[2ex]
      \InputLuatikzPlot{angular-a}&
      \InputLuatikzPlot{angular-b}&
      \InputLuatikzPlot{angular-c}
    \end{tabular}
  \end{center}
  \caption{%
    Illustration of the syndromes \enum{RDF\_GLOBAL} and \enum{ANGULAR}.  Upper row, from left to right: proper,
    distorted layouts of a regular grid, and a force-directed layout of an irregular graph (\enquote{power grid}).
    Central row: smoothed relative frequency distributions for the \enum{RDF\_GLOBAL} syndromes computed for the
    respective layouts in the upper row.  The isolated peaks in the leftmost distribution correspond to characteristic
    distances in the lattice.  In the central plot, these peaks are widened due to random distortion. In the rightmost
    plot, no regular structure can be identified.  Lower row: smoothed relative frequency distributions for the
    \enum{ANGULAR} syndromes.  The leftmost plot clearly shows the dominance of angles proportional to $\pi/2$.  In the
    rightmost plot, the distinctive peak at $\phi = 2\pi$ corresponds to the large number of degree one vertices.
  }
  \label{app:fig:rdf-global}
\end{figure}
\clearpage

\begin{figure}[bp]
  \begin{center}
    \begin{tabular}{c@{\qquad}c@{\qquad}c}
      \InputTikzGraph{0.25\textwidth}{rome}&
      \InputTikzGraph{0.25\textwidth}{north}&
      \InputTikzGraph{0.25\textwidth}{randdag}\\[2ex]
      \enum{ROME} & \enum{NORTH} & \enum{RANDDAG}
    \end{tabular}
    \par\vspace{1cm}
    \begin{tabular}{c@{\qquad\qquad}c}
      \InputTikzGraph{0.35\textwidth}{bcspwr}&
      \InputTikzGraph{0.35\textwidth}{grenoble}\\[2ex]
      \enum{BCSPWR} & \enum{GRENOBLE}
    \end{tabular}
    \par\vspace{1cm}
    \begin{tabular}{c@{\qquad\qquad}c}
      \InputTikzGraph{0.35\textwidth}{psadmit}&
      \InputTikzGraph{0.35\textwidth}{smtape}\\[2ex]
      \enum{PSADMIT} & \enum{SMTAPE}
    \end{tabular}
  \end{center}
  \caption{%
    Examples of imported graphs.  The \enum{BCSPWR}, \enum{GRENOBLE}, \enum{PSADMIT} and \enum{SMTAPE} graphs come from
    the respective datasets in the Harwell-Boeing collection in NIST's \enquote{Matrix Market}~\cite{MatrixMarket}.  All
    graphs are visualized using the FM\textsuperscript{3} algorithm.
  }
  \label{app:fig:archives}
\end{figure}

\begin{figure}[bp]
  \begin{center}
    \begin{tabular}{c@{\qquad}c@{\qquad}c@{\qquad}c}
      \InputTikzGraph{0.2\textwidth}{grid}&
      \InputTikzGraph{0.2\textwidth}{torus1}&
      \InputTikzGraph{0.2\textwidth}{torus2}&
      \InputTikzGraph{0.2\textwidth}{bottle}\\[2ex]
      \enum{GRID} & \enum{TORUS1} & \enum{TORUS2} & \enum{BOTTLE}
    \end{tabular}
    \par\vspace{1cm}
    \begin{tabular}{c@{\qquad}c@{\qquad}c@{\qquad}c}
      \InputTikzGraph{0.2\textwidth}{quasi3d}&
      \InputTikzGraph{0.2\textwidth}{quasi4d}&
      \InputTikzGraph{0.2\textwidth}{quasi5d}&
      \InputTikzGraph{0.2\textwidth}{quasi6d}\\[2ex]
      \enum{QUASI3D} & \enum{QUASI4D} & \enum{QUASI5D} & \enum{QUASI6D}
    \end{tabular}
    \par\vspace{1cm}
    \begin{tabular}{c@{\qquad}c@{\qquad}c}
      \InputTikzGraph{0.25\textwidth}{lindenmayer}&
      \InputTikzGraph{0.25\textwidth}{mosaic1}&
      \InputTikzGraph{0.25\textwidth}{mosaic2}\\[2ex]
      \enum{LINDENMAYER} & \enum{MOSAIC1} & \enum{MOSAIC2}
    \end{tabular}
  \end{center}
  \caption{%
    Examples of generated graphs labeled by the respective generators.  \enum{GRID}, \enum{LINDENMAYER},
    \enum{QUASI\meta{$n$}D}, \enum{MOSAIC1}, \enum{MOSAIC2} and \enum{BOTTLE} layouts are native.  \enum{TORUS1} and
    \enum{TORUS2} are visualized with the stress-minimization algorithm.
  }
  \label{app:fig:generators}
\end{figure}

\begin{figure}[p]
  \begin{center}
    \begin{tabular}{c@{\qquad\qquad}c}
      % -*- coding:utf-8; mode:latex; -*- %

% Copyright (C) 2018 Moritz Klammler <moritz.klammler@student.kit.edu>
%
% Copying and distribution of this file, with or without modification, are permitted in any medium without royalty
% provided the copyright notice and this notice are preserved.  This file is offered as-is, without any warranty.

\begin{tikzpicture}[scale = 0.5]

  \path[use as bounding box] (-4.5, -4.5) rectangle (4.5, 4.5);

  \node[Vertex] (v) at (0, 0) {$v$};
  \node[Vertex] (u1) at (  0:4) {$u_1$};
  \node[Vertex] (u2) at ( 90:4) {$u_2$};
  \node[Vertex] (u3) at (180:4) {$u_3$};
  \node[Vertex] (u4) at (270:4) {$u_4$};

  \draw[edge] (v)  -- (u1);
  \draw[edge] (v)  -- (u2);
  \draw[edge] (v)  -- (u3);
  \draw[edge] (v) -- (u4);

\end{tikzpicture}
 & ../../report/pics/gen-lindenmayer-star.tex\\
      $\menum{L/SINGLETON}$ & $\menum{L/STAR}_3(v)$\\[2ex]
      % -*- coding:utf-8; mode:latex; -*- %

% Copyright (C) 2018 Moritz Klammler <moritz.klammler@student.kit.edu>
%
% Copying and distribution of this file, with or without modification, are permitted in any medium without royalty
% provided the copyright notice and this notice are preserved.  This file is offered as-is, without any warranty.

\begin{tikzpicture}[scale = 0.5]

  \path[use as bounding box] (-4.5, -4.5) rectangle (4.5, 4.5);

  \node[Vertex] (v)  at (0, 0)  {$v$};
  \node[Vertex] (u1) at (  0:4) {$u_1$};
  \node[Vertex] (u2) at ( 90:4) {$u_2$};
  \node[Vertex] (u3) at (180:4) {$u_3$};
  \node[Vertex] (u4) at (270:4) {$u_4$};

  \node[vertex] (w1)  at (  0:2) {};
  \node[vertex] (w2)  at ( 30:2) {};
  \node[vertex] (w3)  at ( 60:2) {};
  \node[vertex] (w4)  at ( 90:2) {};
  \node[vertex] (w5)  at (120:2) {};
  \node[vertex] (w6)  at (150:2) {};
  \node[vertex] (w7)  at (180:2) {};
  \node[vertex] (w8)  at (210:2) {};
  \node[vertex] (w9)  at (240:2) {};
  \node[vertex] (w10) at (270:2) {};
  \node[vertex] (w11) at (300:2) {};
  \node[vertex] (w12) at (330:2) {};

  \draw[edge] (w1) -- (w2) -- (w3) -- (w4) -- (w5) -- (w6) -- (w7) -- (w8) -- (w9) -- (w10) -- (w11) -- (w12) -- (w1);

  \foreach \i in {1, ..., 12} {
    \draw[edge] (v) -- (w\i);
  }

  \draw[edge] (w1)  -- (u1);
  \draw[edge] (w4)  -- (u2);
  \draw[edge] (w7)  -- (u3);
  \draw[edge] (w10) -- (u4);

\end{tikzpicture}
 & ../../report/pics/gen-lindenmayer-ring.tex\\
      $\menum{L/WHEEL}_3(v)$ & $\menum{L/RING}_3(v)$\\[2ex]
      ../../report/pics/gen-lindenmayer-clique.tex & ../../report/pics/gen-lindenmayer-grid.tex\\
      $\menum{L/CLIQUE}_2(v)$ & $\menum{L/GRID}_{3,5}(v)$
    \end{tabular}
  \end{center}
  \caption{%
    Illustration of the \enum{LINDENMAYER} generator operations.  A degree four vertex may be replaced by any of the
    above subgraphs, except for the bottom right subgraph which replaces a degree zero vertex.
  }
  \label{app:fig:lindenmayer-subgens}
\end{figure}

\begin{figure}[p]
  \begin{center}
    \newcommand*{\GenMosaicScale}{0.5}
    \begin{tabular}{c@{\quad}c@{\quad}c}
      ../../report/pics/gen-mosaic-star.tex&
      ../../report/pics/gen-mosaic-flower.tex&
      % -*- coding:utf-8; mode:latex; -*- %

% Copyright (C) 2018 Moritz Klammler <moritz.klammler@student.kit.edu>
%
% Copying and distribution of this file, with or without modification, are permitted in any medium without royalty
% provided the copyright notice and this notice are preserved.  This file is offered as-is, without any warranty.

\providecommand*{\GenMosaicScale}{0.6}
\begin{tikzpicture}[scale = \GenMosaicScale, rotate = 18]

  \node[Vertex] (u1) at (  0:3) {$u_1$};
  \node[Vertex] (u2) at ( 72:3) {$u_2$};
  \node[Vertex] (u3) at (144:3) {$u_3$};
  \node[Vertex] (u4) at (216:3) {$u_4$};
  \node[Vertex] (u5) at (288:3) {$u_5$};

  \node[Vertex] (w1) at ($(u1)!0.5!(u2)$) {$w_1$};
  \node[Vertex] (w2) at ($(u2)!0.5!(u3)$) {$w_2$};
  \node[Vertex] (w3) at ($(u3)!0.5!(u4)$) {$w_3$};
  \node[Vertex] (w4) at ($(u4)!0.5!(u5)$) {$w_4$};
  \node[Vertex] (w5) at ($(u5)!0.5!(u1)$) {$w_5$};

  \draw[edge] (u1) -- (w1) -- (u2) -- (w2) -- (u3) -- (w3) -- (u4) -- (w4) -- (u5) -- (w5) -- (u1);
  \draw[edge] (w1) -- (w2) -- (w3) -- (w4) -- (w5) -- (w1);

\end{tikzpicture}
\\[1ex]
      \enum{M/STAR} & \enum{M/FLOWER} & \enum{M/SHAPE}
    \end{tabular}
  \end{center}
  \caption{%
    Operations of the \enum{MOSAIC} generator on a pentagonal facet $\{u_1,\ldots,u_5\}$.
  }
  \label{app:fig:mosaic-subgens}
\end{figure}

\begin{figure}[p]
  \begin{center}
    \begin{tabular}{c@{\qquad\qquad}c@{\qquad\qquad}c}
      \InputTikzGraph{0.2\textwidth}{native}&
      \InputTikzGraph{0.2\textwidth}{fmmm}&
      \InputTikzGraph{0.2\textwidth}{stress}\\[2ex]
      \enum{NATIVE} & \enum{FMMM} & \enum{STRESS}
    \end{tabular}
    \par\vspace{1cm}
    \begin{tabular}{c@{\qquad\qquad}c@{\qquad\qquad}c}
      \InputTikzGraph{0.2\textwidth}{random-uniform}&
      \InputTikzGraph{0.2\textwidth}{random-normal}&
      \InputTikzGraph{0.2\textwidth}{phantom}\\[2ex]
      \enum{RANDOM\_UNIFORM} & \enum{RANDOM\_NORMAL} & \enum{PHANTOM}
    \end{tabular}
  \end{center}
  \caption{%
    Examples of different layouts for the same graph.  \enum{RANDOM\_UNIFORM}, \enum{RANDOM\_NORMAL} are random layouts
    where vertex positions are sampled from the uniform and the normal distributions, respectively.
  }
  \label{app:fig:layouts}
\end{figure}

\begin{figure}[p]
  \begin{center}
    \begin{tabular}{lc@{\quad}c@{\quad}c@{\quad}c}
      \rotatebox{90}{\enum{PERTURB}}&
      \InputTikzGraph{0.2\textwidth}{perturb-00000}&
      \InputTikzGraph{0.2\textwidth}{perturb-01500}&
      \InputTikzGraph{0.2\textwidth}{perturb-05000}&
      \InputTikzGraph{0.2\textwidth}{perturb-10000}\\[2ex]
      \rotatebox{90}{\enum{FLIP\_NODES}}&
      \InputTikzGraph{0.2\textwidth}{flip-nodes-00000}&
      \InputTikzGraph{0.2\textwidth}{flip-nodes-01500}&
      \InputTikzGraph{0.2\textwidth}{flip-nodes-05000}&
      \InputTikzGraph{0.2\textwidth}{flip-nodes-10000}\\[2ex]
      \rotatebox{90}{\enum{FLIP\_EDGES}}&
      \InputTikzGraph{0.2\textwidth}{flip-edges-00000}&
      \InputTikzGraph{0.2\textwidth}{flip-edges-01500}&
      \InputTikzGraph{0.2\textwidth}{flip-edges-05000}&
      \InputTikzGraph{0.2\textwidth}{flip-edges-10000}\\[2ex]
      \rotatebox{90}{\enum{MOVLSQ}}&
      \InputTikzGraph{0.2\textwidth}{movlsq-00000}&
      \InputTikzGraph{0.2\textwidth}{movlsq-01500}&
      \InputTikzGraph{0.2\textwidth}{movlsq-05000}&
      \InputTikzGraph{0.2\textwidth}{movlsq-10000}\\[2ex]
      & $r=0\percent$ & $r=15\percent$ & $r=50\percent$ & $r=100\percent$
    \end{tabular}
  \end{center}
  \caption{%
    Examples of applying  different layout worsening techniques at different rates.
  }
  \label{app:fig:worsening}
\end{figure}

\begin{figure}[p]
  \begin{center}
    \begin{tabular}{c@{\qquad}c@{\qquad}c@{\qquad}c}
      \InputTikzGraph{0.2\textwidth}{linear-00000}&
      \InputTikzGraph{0.2\textwidth}{linear-02500}&
      \InputTikzGraph{0.2\textwidth}{linear-07500}&
      \InputTikzGraph{0.2\textwidth}{linear-10000}\\[2ex]
      $r=0\percent$ & $r=25\percent$ & $r=75\percent$ & $r=100\percent$
    \end{tabular}
  \end{center}
  \caption{%
    Example of linear interpolation between a proper and a garbage layout.
  }
  \label{app:fig:interpolating}
\end{figure}
