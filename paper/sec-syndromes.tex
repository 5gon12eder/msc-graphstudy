% -*- coding:utf-8; mode:latex; -*-

% Copyright (C) 2018 Moritz Klammler <moritz.klammler@alumni.kit.edu>
% Copyright (C) 2018 Tamara Mchedlidze <mched@iti.uka.de>
% Copyright (C) 2018 Alexey Pak <alexey.pak@iosb.fraunhofer.de>
%
% This work is licensed under a Creative Commons Attribution-NonCommercial-NoDerivatives 4.0 International License
% (https://creativecommons.org/licenses/by-nc-nd/4.0/).

\section{Definitions}
\label{sec:definitions}

In this paper we consider general simple graphs $G=(V,E)$ where $V=V(G)$ and $E=E(G)$ are the vertex and edge sets of
$G$ with $\card{V}=n$ and $\card{E}=m$.  A \emph{drawing} or \emph{layout} of a graph is its graphical representation
where vertices are drawn as points or small circles, and the edges as straight line segments.  Vertex positions in a
drawing are denoted by $\vec{p^k}=(p_1^k, p_2^k)\transposed$ for $k=1,\dots,n$ and their set $P=\{\vec{p}^k\}_{k=1}^n$.
Furthermore, we use $\textrm{dist}_G(u,v)$ to denote the \emph{graph-theoretical distance} -- the length of the shortest
path between vertices $u$ and $v$ in $G$ -- and $\textrm{dist}_\Gamma(u,v)$ for the Euclidean distance between $u$ and
$v$ in the drawing $\Gamma(G)$.

\section{Quality Syndromes of Graph Layouts}
\label{sec:syndromes}

A \emph{quality syndrome} of a layout $\Gamma$ is a multi-set of numbers sharing an interpretation that are known or
suspected to correlate with the aesthetic quality (e.g.~all pairwise angles between incident edges in $\Gamma$).  In the
following we describe several syndromes (implemented in our code) inspired by popular quality metrics and common
statistical tools.  The list is by no means exhaustive, nor do we claim syndromes below as necessary or independent.
Our model accepts any combination of syndromes; better choices remain to be systematically investigated.

\begin{explanation}
\explain{\enum{PRINVEC1} and \enum{PRINVEC2}} The two principal axes of the set $P$.  If we define a covariance matrix
$C=\{c_{ij}\}$, $c_{ij}=\frac{1}{n}\sum_{k=1}^n{(p_i^k-\overline{p_i})(p_j^k-\overline{p_j})})$, $i, j \in \{1, 2\}$,
where $\overline{p_i}=\frac{1}{n}\sum_{k=1}^n{p_i^k}$ are the mean values over each dimension, then \enum{PRINVEC1} and
\enum{PRINVEC2} will be its eigenvectors.

\explain{\enum{PRINCOMP1} and \enum{PRINCOMP2}} Projections of vertex positions onto $\vec{v}_1=\menum{PRINVEC1}$ and
$\vec{v}_2=\menum{PRINVEC2}$, that is, ~$\{\langle\left(\vec{p}^j-\overline{\vec{p}}\right),\vec{v}_i\rangle\}_{j=1}^n$
for $i\in\{1,2\}$ where $\langle\cdot,\cdot\rangle$ denotes the scalar product.

\explain{\enum{ANGULAR}} Let $A(v)$ denote the sequence of edges incident to a vertex $v$, appearing in a clockwise
order around it in $\Gamma$.  Let $\alpha(e_i,e_j)$ denote the clockwise angle between edges $e_i$ and $e_j$ incident to
the same vertex.  This syndrome is then defined as $\bigcup_{v\in{}V(G)}\{\alpha(e_i,e_j)\st{}e_i,e_j\text{ are
  consecutive in }A(v)\}$.

\explain{\enum{EDGE\_LENGTH}} $\bigcup_{(u,v)\in{}E(G)}\{\textrm{dist}_\Gamma(u,v)\}$ is the set of edge lengths in
$\Gamma$.

\explain{\enum{RDF\_GLOBAL}} $\bigcup_{u\neq{}v\in{}V(G)}\{\dist_\Gamma(u,v)\}$ contains distances between all vertices
in the drawing.  The concept of a \emph{radial distribution function (RDF)}~\cite{Findenegg2015} (the distribution of
\enum{RDF\_GLOBAL}) is borrowed from Statistical Physics and Crystallography and characterizes the regularity of
molecular structures.  In large graph layouts it captures regular, periodic and symmetric patterns in the vertex
positions.
\OnlyArxiv{%
  Fig.~\ref{app:fig:rdf-global} in the Appendix shows histograms of \enum{RDF\_GLOBAL} for some graphs and layouts.
  Note that more regular drawings feature better-isolated peaks in the respective histograms.
}

\explain{$\menum{RDF\_LOCAL}(d)$} $\bigcup_{u\neq{}v\in{}V(G)}\{\dist_\Gamma(u,v)\st\dist(u,v)\leq{}d\}$ is the set of
distances between vertices such that the graph-theoretical distance between them is bounded by $d\in\IntsN$.  In our
implementation, we compute $\menum{RDF\_LOCAL}(2^i)$ for $i\in\{0,\ldots,\ceil{\log_2(D)}\}$ where $D$ is the diameter
of $G$.  $\menum{RDF\_LOCAL}(d)$ in a sense interpolates between \enum{EDGE\_LENGTH} ($d=1$) and \enum{RDF\_GLOBAL}
($d\to\infty$).

\explain{\enum{TENSION}} $\bigcup_{u\neq{}v\in{}V(G)}\{\dist_\Gamma(u,v)/\dist_G(u,v)\}$ are the ratios of Euclidean and
graph-theoretical distances computed for all vertex pairs.  \enum{TENSION} is motivated by and is related to the
well-known stress function~\cite{Kamada1989}.
\end{explanation}

\noindent
Note that before computing the quality syndromes, we \emph{normalize} all layouts so that the center of gravity of $V$
is at the origin and the mean edge length is fixed in order to remove the effects of scaling and translation (but not
rotation).
