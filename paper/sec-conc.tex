% -*- coding:utf-8; mode:latex; -*-

% Copyright (C) 2018 Moritz Klammler <moritz.klammler@alumni.kit.edu>
% Copyright (C) 2018 Tamara Mchedlidze <mched@iti.uka.de>
% Copyright (C) 2018 Alexey Pak <alexey.pak@iosb.fraunhofer.de>
%
% This work is licensed under a Creative Commons Attribution-NonCommercial-NoDerivatives 4.0 International License
% (https://creativecommons.org/licenses/by-nc-nd/4.0/).

\section{Conclusion}
\label{sec:conclusion}

In this paper we propose a machine learning-based discriminator model that selects the more aesthetically pleasing
drawing from a pair of graph layouts.  Our model picks the \enquote{better} layout in more than $96\percent$ cases and
outperforms known stress-based and linear combination-based models.  To the best of our knowledge, this is the first
application of machine learning methods to this question.  Previously, such techniques have proven successful in a range
of complex issues involving aesthetics, prior knowledge, and unstated rules in object recognition, industrial design,
and digital arts.  As our model uses a simple network architecture, investigating the performance of more complex
networks is warranted.

Previous efforts were focused on determining the aesthetic quality of a layout as a weighted average of individual
quality metrics.  We extend these ideas and findings in the sense that we do not assume any particular form of
dependency between the overall aesthetic quality and the individual quality metrics.

Going beyond simple quality metrics, we define quality syndromes that capture arrays of information about graphs and
layouts.  In particular, we borrow the notion of RDF from Statistical Physics and Crystallography; RDF-based features
demonstrate the strongest potential in extracting the aesthetic quality of a layout.  We expect RDFs (describing the
microscopic structure of materials) to be the most relevant for large graphs.  It is tempting to investigate whether
further tools from physics can be useful in capturing drawing aesthetics.

From multiple syndromes, we construct fixed-size feature vectors using common statistical tools.  Our feature vector
does not contain any information on crossings or crossing angles, nevertheless its performance is superior with respect
to the weighted averages-based model which accounts for both.  It would be interesting to investigate whether including
these and other features further improves the performance of the neural network-based model.

In order to train and evaluate the model, we have assembled a relatively large corpus of labeled pairs of layouts, using
available and generated graphs and exploiting the assumption that layouts produced by force-directed algorithms and
native graph layouts are aesthetically pleasing and that disturbing them reduces the aesthetic quality.  We admit that
this study should ideally be repeated with human-labeled data.  However, this requires that a dataset be collected with
a size similar to ours, which is a challenging task.  Creating such a dataset may become a critically important
accomplishment in the graph drawing field.
