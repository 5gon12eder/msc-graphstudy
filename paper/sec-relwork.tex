% -*- coding:utf-8; mode:latex; -*-

% Copyright (C) 2018 Moritz Klammler <moritz.klammler@alumni.kit.edu>
% Copyright (C) 2018 Tamara Mchedlidze <mched@iti.uka.de>
% Copyright (C) 2018 Alexey Pak <alexey.pak@iosb.fraunhofer.de>
%
% This work is licensed under a Creative Commons Attribution-NonCommercial-NoDerivatives 4.0 International License
% (https://creativecommons.org/licenses/by-nc-nd/4.0/).

\section{Related Work}
\label{sec:relwork}

According to empirical studies, graph drawings that maximize one or several quality metrics are more aethetically
pleasing and easier to read~\cite{HuangE05,Huang2013,Purchase97,PurchaseHNK12,WarePCM02}.  For instance, in their
seminal work, Purchase et al.~have established~\cite{PurchaseCM96} that higher numbers of edge crossings and bends as
well as lower levels of symmetry negatively influence user performance in graph reading tasks.

Many graph drawing algorithms attempt to optimize multiple quality metrics.  As one way to combine them, Huang et
al.~\cite{Huang2013} have used a weighted sum of \enquote{simple} metrics, effects of their interactions (see
Purchase~\cite{PURCHASE98} or Huang and Huang~\cite{Huang10}), and error terms to account for possible measurement
errors.

In another work, Huang et al.~\cite{HuangHL16} have empirically demonstrated that their \enquote{aggregate} metric is
sensitive to quality changes and is correlated with the human performance in graph comprehension tasks.  They have also
noticed that the dependence of aesthetic quality on input quality metrics can be non-linear (e.g.~a quadratic
relationship better describes the interplay between crossing angles and drawing quality~\cite{HuangHE08}).  Our work
extends this idea as we allow for arbitrary non-linear dependencies implemented by an artificial neural network.

In evolutionary graph drawing approaches, several techniques have been suggested to \enquote{train} a \emph{fitness
  function}\footnote{Objective function in genetic algorithms that summarizes optimization goals.}  from the user's
responses as a composition of several known quality metrics.  Masui~\cite{Masui1994} modeled the fitness function as a
linear combination in which the weights are obtained via genetic programming from the pairs of \enquote{good} and
\enquote{bad} layouts provided by users.  The so-called co-evolution was used by Barbosa and Barreto~\cite{Barbosa2001}
to evolve the weights of the fitness function in parallel with a drawing population in order to match the ranking made
by users.  Sp\"{o}nemann and others~\cite{Sponemann14} suggested two alternative techniques.  In the first one, the user
directly chooses the weights with a slider.  In the second, they select good layouts from the current population and the
weights are adjusted according to the selection.  Rosete-Suarez~\cite{Rosete-Suarez99} determined the relative
importance of individual quality metrics based on user inputs.  Several machine learning-based approaches to graph
drawing are described by dos Santos Vieira et al.~\cite{Raissa2015}.  Recently, Kwon et al.~\cite{Kwon18} presented a
novel work on topological similarity of graphs.  Their goal was to avoid expensive computations of graph layouts and
their quality measures.  The resulting system was able to sketch a graph in different layouts and estimate corresponding
quality measures.
