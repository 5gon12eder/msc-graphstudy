% -*- coding:utf-8; mode:latex; -*-

% Copyright (C) 2018 Moritz Klammler <moritz.klammler@alumni.kit.edu>
% Copyright (C) 2018 Tamara Mchedlidze <mched@iti.uka.de>
% Copyright (C) 2018 Alexey Pak <alexey.pak@iosb.fraunhofer.de>
%
% This work is licensed under a Creative Commons Attribution-NonCommercial-NoDerivatives 4.0 International License
% (https://creativecommons.org/licenses/by-nc-nd/4.0/).

\section{Evaluation}
\label{sec:eval}

The performance of the discriminator model was evaluated using \emph{cross-validation} with $\XValTestRuns$-fold random
subsampling~\cite{Kohavi1995}.  In each round, $20\percent$ of graphs (with all their layouts) were chosen randomly and
were set aside for testing, and the model was trained using the remaining layout pairs.  Of $N$ labeled pairs used for
testing, in each round we computed the number $N_\mathrm{correct}$ of pairs for which the model properly predicted the
aesthetic preference, and derived the accuracy (success rate) $A=N_\mathrm{correct}/N$.  The standard deviation of $A$
over the $\XValTestRuns$ runs was taken as the uncertainty of the results.  With the average number of test samples of
$N=\XValCountApprox$, the eventual success rate was $\boldsymbol{A=(\XValSuccessMean\pm\XValSuccessStdev)\percent}$.

\subsection{Comparison With Other Metrics}

In order to assess the relative standing of the suggested method, we have implemented two known aesthetic metrics
(\emph{stress} and the \emph{combined metric} by Huang et al.~\cite{HuangHL16}) and evaluated them over the same
dataset.  The metric values were trivially converted to the respective discriminator function outputs.

Stress $\stress$ of a layout $\Gamma$ of a simple connected graph $G=(V,E)$ was defined by Kamada and
Kawai~\cite{Kamada1989} as
\begin{equation}
  \stress(\Gamma) = \sum_{i=1}^{n-1} \sum_{j=i+1}^{n}
  k_{ij} \left( \dist_\Gamma(v_i,v_j) - L \dist_G(v_i,v_j) \right)^2
  \mathendpunct{,}
  \label{eq:stress}
\end{equation}
where $L$ denotes the desirable edge length and $k_{ij}=K/\dist_G(v_i,v_j)^2$ is the strength of a \enquote{spring}
attached to $v_i$ and $v_j$.  The constant $K$ is irrelevant in the context of discriminator functions and can be set to
any value.

As observed by Welch and Kobourov~\cite{Welch2017}, the numeric value of stress depends on the layout scale via the
constant $L$ in the Eq.~\ref{eq:stress} which complicates comparisons.  Their suggested solution was for each layout to
find $L$ that minimizes $\stress$ (e.g.~using binary search).  In our implementation, we applied a similar technique
based on fitting and minimizing a quadratic function to the stress computed at three scales.  We refer to this quantity
as \enum{STRESS}.

The combined metric proposed by Huang et al.~\cite{HuangHL16} (referred to as \enum{COMB}) is a weighted average of four
simpler quality metrics: the number of edge crossings (\enum{CC}), the minimum crossing angle between any two edges in
the drawing (\enum{CR}), the minimum angle between two adjacent edges (\enum{AR}), and the standard deviation computed
over all edge lengths (\enum{EL}).

The average is computed over the so-called $z$-scores of the above metrics.  Each $z$-score is found by subtracting the
mean and dividing by the standard deviation of the metric for all layouts of a given graph to be compared with each
other.  More formally, let $G$ be a graph and $\Gamma_1,\ldots,\Gamma_k$ be its $k$ layouts to be compared pairwise.
Let $M(\Gamma_i)$ be the value of metric $M$ for $\Gamma_i$ and $\mu_M$ and $\sigma_M$ be the mean and the standard
deviation of $M(\Gamma_i)$ for $i\in\{1,\ldots,k\}$.  Then
\begin{equation}
  z_M^{(i)} = \frac{M(\Gamma_i) - \mu_M}{\sigma_M}
\end{equation}
is the $z$-score for metric $M$ and layout $\Gamma_i$.  The combined metric then is
\begin{equation}
  \menum{COMB}(\Gamma_j) = \sum_{M} w_M \: z_{M}^{(j)}
  \mathendpunct{.}
\end{equation}
The weights $w_M$ were found via Nelder-Mead maximization~\cite{Press2007} of the prediction accuracy over the training
dataset\footnote{The obtained weights are: \TextualHuangWeights}.

\begin{figure}[p]
  \csdef{Status0}{\textcolor{TangoScarletRed2}{\ding{55}}}%
  \csdef{Status1}{\textcolor{TangoChameleon2}{\ding{51}}}%
  \newcommand*{\ShowIt}[1]{\InputTikzGraph{0.25\textwidth}{#1}}
  \newcommand*{\TellIt}[3]{%
    \begin{tabular}{l@{\quad}c}
      \enum{DISC\_MODEL} & \csuse{Status#1}\\
      \enum{STRESS}      & \csuse{Status#2}\\
      \enum{COMB}        & \csuse{Status#3}\\
    \end{tabular}
  }
  \begin{center}
    \begin{tabular}{>{\centering}m{0.35\textwidth}>{\centering}m{0.35\textwidth}@{\qquad}m{0.2\textwidth}}
      \ShowIt{8314f2c1-4d644640-more} & \ShowIt{8314f2c1-bac12550-less} & \TellIt{1}{0}{1}\\[1ex]\\[1ex]
      \ShowIt{d8c1498f-8c5ad49b-more} & \ShowIt{d8c1498f-4c2beafc-less} & \TellIt{1}{0}{0}\\[1ex]\\[1ex]
      \ShowIt{5b3b66d2-49e8882a-more} & \ShowIt{5b3b66d2-f60fcc04-less} & \TellIt{1}{1}{0}\\
    \end{tabular}
  \end{center}
  \caption{%
    Examples where our discriminator model (\enum{DISC\_MODEL}) succeeds (\csuse{Status1}) and the competing metrics
    fail (\csuse{Status0}) to predict the answer correctly.  In each row, the layout on the left is expected to be
    superior compared to the one on the right.
  }
  \label{fig:model-comp}
\end{figure}

The accuracy of the stress-based and the combined model-based discriminators is shown in
Tab.~\ref{tab:competing-metrics}.  In most cases, our model outperforms these algorithms by a comfortable margin.
Fig.~\ref{fig:model-comp} provides examples of mis-predictions.  By inspecting such cases, we notice that \enum{STRESS}
often fails to guess the aesthetics of (almost) planar layouts that contain both very short and very long edges (such
behavior may also be inferred from the definition of \enum{STRESS}).  We observe that there are planar graphs, such as
nested triangulations, for which this property is unavoidable in planar drawings.  The mis-predictions of \enum{COMB}
seem to be due to the high weight of the edge length metric \enum{EL}.  Both \enum{STRESS} and \enum{COMB} are weaker
than our model in capturing the absolute symmetry and regularity of layouts.

\begin{table}[p]
  \begin{center}
    \InputCompetingMetrics{eval-competing-metrics.tex}
  \end{center}
  \caption{%
    Accuracy scores for the \enum{COMB} and \enum{STRESS} model.  The standard deviation in each column is estimated
    based on the $5$-fold cross-validation (using $20\percent$ of data for testing each time).  The \enquote{Advantage}
    column shows the improvement in the accuracy of our model with respect to the alternative metric.
  }
  \label{tab:competing-metrics}
\end{table}

\subsection{Significance of Individual Syndromes}

In order to estimate the influence of individual syndromes on the final result, we have tested several modifications of
our model.  For each syndrome, we considered the case when the feature vector contained only that syndrome.  In the
second case, that syndrome was removed from the original feature vector.  The entries for the omitted features were set
to zero.  The results are shown in Tab.~\ref{tab:eval-puncture}.

\begin{table}
  \begin{center}
    \InputPunctureResult[eval-cross-valid.tex]{eval-puncture.tex}
  \end{center}
  \caption{%
    Success rates of our discriminator when a syndrome is excluded from the feature vector, and when the feature vector
    contains only that a syndrome.  Note that \enum{RDF\_LOCAL} is a family of syndromes that are all included or
    excluded together.  The apparent paradox of higher success rates when some syndromes are excluded can be explained
    by a statistical fluctuation and is well within the listed range of uncertainty.
  }
  \label{tab:eval-puncture}
\end{table}

As can be observed, the dominant contribution to the accuracy of the model is due to the RDF-based properties
\enum{RDF\_LOCAL} and \enum{RDF\_GLOBAL}.  The exclusion of other syndromes does not significantly change the results
(they agree within the estimated uncertainty).  However, the sole inclusion of these syndromes still performs better
than random choice.  This suggests that there is a considerable overlap between the aesthetic aspects captured by
various syndromes.  Further analysis is needed to identify the nature and the magnitude of these correlations.
