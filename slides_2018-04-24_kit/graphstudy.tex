% -*- coding:utf-8; mode:latex; -*-

% Copyright (C) 2018 Moritz Klammler <moritz.klammler@student.kit.edu>
%
% This work is licensed under a Creative Commons Attribution-NonCommercial-NoDerivatives 4.0 International License
% (https://creativecommons.org/licenses/by-nc-nd/4.0/).

\documentclass{beamer}

% -*- coding:utf-8; mode:latex; -*-

% Copyright (C) 2018 Karlsruhe Institute of Technology
% Copyright (C) 2018 Moritz Klammler <moritz.klammler@alumni.kit.edu>
%
% Copying and distribution of this file, with or without modification, are permitted in any medium without royalty
% provided the copyright notice and this notice are preserved.  This file is offered as-is, without any warranty.

\makeatletter

\usetheme{KIT}

\usepackage{fontspec}
\usepackage{polyglossia}
\usepackage{amsmath,amsfonts}
%\usepackage{slidecite}
\usepackage{csquotes}
\usepackage{ifdraft}
\usepackage{etoolbox}
\usepackage{tabularx}
\usepackage{booktabs}
\usepackage{multicol}
\usepackage{gnuplot-lua-tikz}
\usetikzlibrary{positioning}
\usepackage{libertine}
\setmainlanguage{english}

\usepackage[%
  backend = biber,
  citestyle = authoryear,
  bibstyle = chem-acs,
  alldates = iso8601,
  giveninits = true,
]{biblatex}

\usenavigationsymbols
\setcounter{tocdepth}{2}

\setbeamercolor*{bibliography entry author}{fg=KITblack}
\setbeamercolor*{bibliography entry title}{fg=KITblack}
\setbeamercolor*{bibliography entry journal}{fg=KITblack}
\setbeamercolor*{bibliography entry note}{fg=KITblack}

\setbeamercolor{structure}{fg=KITgreen}
%\setbeamercolor*{palette secondary}{fg=KITgreen}
%\setbeamercolor*{palette tertiary}{fg=KITgreen}

\AtBeginSection[]

\makeatother

\KITtitleimage[width = 0.95\paperwidth]{pics/titlepicture.png}
\logo{\pgfimage[height = 1.5\KITlogoht]{pics/algo-logo.pdf}}

\addbibresource{literature.bib}

\hypersetup{
  pdfauthor = {Moritz Klammler},
  pdftitle = {Aesthetic Value of Graph Layouts: Investigation of Statistical Syndromes for Automatic Quantification},
  pdfsubject = {Master's Thesis},
  pdfkeywords = {}
}

\title{Aesthetic Value of Graph Layouts}
\subtitle{Investigation of Statistical Syndromes for Automatic Quantification}
\author{Moritz Klammler}
\institute[Institute of Theoretical Computer Science]%
          {Master's Thesis of Moritz Klammler at the Institute of Theoretical Computer Science}

\date{April 2018}

\begin{document}

\begin{frame}
  \maketitle
\end{frame}

\begin{frame}
  \frametitle{\abstractname}
  \noindent\parbox{\textwidth}{%
    \footnotesize
    Visualizing relational data as drawing of graphs is a technique in very wide-spread use across many fields and
    professions.  While many graph drawing algorithms have been proposed to automatically generate a supposedly
    high-quality picture from an abstract mathematical data structure, the graph drawing community is still searching
    for a way to quantify the aesthetic value of any given solution in a way that allows one to compare graph layouts
    created by different algorithms for the same graph (presumably to automatically choose the better one).  We believe
    that one promising path towards this goal could be enabled by combining data analysis techniques that have proven
    useful in other scientific disciplines that are dealing with large structures such as astronomy, crystallography or
    thermodynamics.  In this work we present an initial investigation of some statistical properties of graph layouts
    that we believe could provide viable syndromes for the aesthetic value.  As a proof of concept, we used machine
    learning techniques to train a neural network with the results of our data analysis and thereby built a model that
    is able to discriminate between better and worse layouts with an accuracy of \(95\,\%\).  A rudimentary evaluation
    of the model was performed and is presented.  This work primarily provides an infrastructure to enable further
    experimentation on the topic and will be made available to the public as Free Software.
  }
\end{frame}

\begin{frame}
  \frametitle{\contentsname}
  \begin{multicols}{2}
    \tableofcontents
  \end{multicols}
\end{frame}

\section{Introduction}

\subsection{Motivation}
\begin{frame}
  \frametitle{Motivation}
  \begin{center}
    \begin{tabular}{c@{\qquad\qquad}c}
      \InputTikzGraph{0.35\textwidth}{pics/example-a.tikz} & \InputTikzGraph{0.35\textwidth}{pics/example-b.tikz}
    \end{tabular}
  \end{center}
  \par\vfill
  Possible applications:
  \par\smallskip
  \begin{itemize}
  \item Run \(N\) layout algorithms in parallel, choose the best result
  \item Select layout algorithm for a given application
  \item Aid the development of domain-specific layout algorithms
  \end{itemize}
\end{frame}

\subsection{Problem Statement}
\begin{frame}
  \frametitle{Problem Statement}
  \begin{itemize}
  \item \emph{Simple graphs} only\footnote{undirected, no loops, no multiple edges}
  \item \emph{Vertex layouts} only\footnote{vertices are 2D points, edges are straight lines}
  \item Two layouts for the \emph{same} graph given
  \item Decide which one is aesthetically more pleasing\footnote{ideally, we aim for a \emph{partial order}}
  \end{itemize}
\end{frame}

\subsection{Previous Work}
\begin{frame}
  \frametitle{Previous Work}
  Existing Measures:
  \par\smallskip
  \begin{itemize}
  \item Energy (from force-directed methods)
  \item\textcite{Purchase2002}\newline
    \begingroup\footnotesize
    edge crossings, edge bends, symmetry, minimal angles between edges, edge orthogonality, node orthogonality,
    consistent flow direction
    \endgroup
  \item Binary Stress \parencite{Kamada1989,Koren2008}
  \item\textcite{Klapaukh2014}
  \item Combined Metric \parencite{Huang2016}
  \end{itemize}
  \par\bigskip
  Problems with these:
  \par\smallskip
  \begin{itemize}
  \item Too many a priory assumptions
  \item Too localized / too little context
  \item Might loose valuable information due to oversimplification
  \item Some are unstable with respect to the simplest transformations (scaling)
  \end{itemize}
\end{frame}

\subsection{Our Contribution}
\begin{frame}
  \frametitle{Our Contribution}
  \begin{itemize}
  \item Idea: Use data analysis techniques that were successful in other disciplines (crystallography, astronomy,
    thermodynamics, \dots)
  \item Strategy: Condense this information into a fixed-size feature vector via statistic analysis
  \item Question: Can we find syndromes that allow for successful automatic quantification?
  \item Guideline: Try to use as few a priori assumptions as possible and detect features from first principles
  \end{itemize}
\end{frame}

\begin{frame}
  \begin{tikzpicture}[remember picture, overlay]
    \node at (current page.center) {\pgfimage[height = 0.8\paperheight]{pics/webui-compare.png}};
  \end{tikzpicture}
\end{frame}

\section{Methodology}
\begin{frame}
  \frametitle{Methodology}
  \begin{itemize}
  \item Gather as many graphs as possible
  \item Obtain as many (known) good and bad layouts as possible
  \item Compute properties for all of them
  \item Use \emph{a priori} knowledge to label pairs of layouts
  \item Use data to build a discriminator
  \end{itemize}
\end{frame}

\subsection{Implementation}
\begin{frame}
  \frametitle{Implementation}
  \begin{itemize}
  \item Plug-in infrastructure for unattended experimentation
  \item Graph generators, layouts, layout transformations and properties implemented as small programs
  \item Siamese neural network
  \item Feature-rich web front-end for data inspection
  \end{itemize}
  \par\bigskip
  Technologies:
  \par\smallskip
  \begin{itemize}
  \item Open Graph Drawing Framework (OGDF)
  \item {\Keras} + {\TensorFlow}
  \item {\CXX}, {\Python}, {\SQLite}, {\XSLT}, {\CMake}, \dots
  \end{itemize}
\end{frame}

\section{Statistical Syndromes}
\begin{frame}
  \frametitle{Statistical Syndromes of Graph Layouts}
  We investigated several \emph{properties} (multisets of scalar events) for a given layout.
  \par\medskip
  \begin{itemize}
  \item Principal Components (\enum{PRINCOMP1ST} and \enum{PRINCOMP2ND})
  \item Angles Between Incident Edges (\enum{ANGULAR})
  \item Edge Lengths (\enum{EDGE\_LENGTH})
  \item Pairwise Distances (\enum{RDF\_GLOBAL} and \enum{RDF\_LOCAL})
  \item Tension (\enum{TENSION})
  \end{itemize}
\end{frame}

\begin{frame}
  \frametitle{Principal Components}
  %\framesubtitle{\enum{PRINCOMP1ST} and \enum{PRINCOMP2ND}}
  \begin{itemize}
  \item Find linear independent axes among which the moment of inertia is maximized
  \item Consider the distribution of vertex coordinates along those axes
  \item Can be computed with \(\BigO(n)\) effort
  \end{itemize}
  \par\vspace{4ex}
  \begin{align*}
    &\menum{PRINCOMP1ST} = \multiset{\braket{\toolmark[2]{TT1}{\vec{p}^{(1)}}}{\Layout(v)}\suchthat{v}\in\GraphV}\\
    &\menum{PRINCOMP2ND} = \multiset{\braket{\toolmark[2]{TT2}{\vec{p}^{(2)}}}{\Layout(v)}\suchthat{v}\in\GraphV}
  \end{align*}
  \begin{tikzpicture}[remember picture, overlay]
    \node<2>[tooltip, anchor = south east, above left = 2ex of TT1] (TT1text) {first principal axis};
    \draw<2>[toolarrow] (TT1text.east) .. controls (TT1|-TT1text.east) .. (TT1);
    \node<2>[tooltip, anchor = north east, below left = 2ex of TT2] (TT2text) {second principal axis};
    \draw<2>[toolarrow] (TT2text.east) .. controls (TT2|-TT2text.east) .. (TT2);
  \end{tikzpicture}
\end{frame}

\PropertyDemo[PRINCOMP1ST]{princomp1st}{1\textsuperscript{st} Principal Component}
\PropertyDemo[PRINCOMP2ND]{princomp2nd}{2\textsuperscript{nd} Principal Component}

\begin{frame}
  \frametitle{Angles Between Incident Edges (\enum{ANGULAR})}
  \begin{itemize}
  \item Enumerate polar angles of incident edges in clockwise order
  \item Compute adjacent differences (of polar angles)
  \item Special cases for \(\deg(v)=1\) and degenerate cases
  \item Consider distribution of all those angles
  \item Can be computed with \(\BigO(n+m)\) effort
  \end{itemize}

  \begin{equation*}
    \menum{ANGULAR} = \Disjunction_{v\in\GraphV}\phi_\Layout(v)
  \end{equation*}
\end{frame}

\PropertyDemo[ANGULAR]{angular}{Angles Between Incident Edges}

\begin{frame}
  \frametitle{Edge Lengths (\enum{EDGE\_LENGTH})}
  \begin{itemize}
  \item Consider distribution of edge lengths
  \item Can be computed with \(\BigO(m)\) effort
  \end{itemize}
  \par
  \begin{equation*}
    \menum{EDGE\_LENGTH}=\multiset{\toolmark[2]{TT}{\edgelen_\Layout(e)}\suchthat{e}\in\GraphE}
  \end{equation*}
  \begin{tikzpicture}[remember picture, overlay]
    \node<2>[tooltip, anchor = north east, below left = 2ex of TT] (TTtext)
            {length of edge \(e\) in layout \(\Layout\)};
    \draw<2>[toolarrow] (TTtext.east) .. controls (TT|-TTtext.east) .. (TT);
  \end{tikzpicture}
\end{frame}

\PropertyDemo[EDGE\_LENGTH]{edge-length}{Edge Lengths}

\begin{frame}
  \frametitle{Pairwise Distances (\enum{RDF\_GLOBAL})}
  \begin{itemize}
  \item Compute pairwise distances between all pairs of vertices
  \item Can be computed with \(\BigO(n^2)\) effort
  \end{itemize}
  \par
  \begin{equation*}
    \menum{RDF\_GLOBAL} = \multiset{\toolmark[2]{TT}{\dist_\Layout(v_1,v_2)}\suchthat{}v_1,v_2\in\GraphV}
  \end{equation*}
  \begin{tikzpicture}[remember picture, overlay]
    \node<2>[tooltip, anchor = north east, below left = 2ex of TT] (TTtext)
            {distance between \(\Layout(v_1)\) and \(\Layout(v_2)\)};
    \draw<2>[toolarrow] (TTtext.east) .. controls (TT|-TTtext.east) .. (TT);
  \end{tikzpicture}
\end{frame}

\PropertyDemo[RDF\_GLOBAL]{rdf-global}{Pairwise Distances}

\begin{frame}
  \frametitle{Pairwise Distances (\(\menum{RDF\_LOCAL}(d)\))}
  \begin{itemize}
  \item Restrict global RDF to those pairs of vertices that have a graph-theoretical distance of at most
    \(d\in\mathbb{R}\)
  \item Repeat for different values of \(d=2^i\) for \(i\in\mathbb{N}_0\) up to the longest finite shortest path in the
    graph
  \item Interpolates between \enum{EDGE\_LENGTH} and \enum{RDF\_GLOBAL}
  \item Can be computed with \(\BigO(n^3)\) effort
  \end{itemize}
  \par
  \begin{equation*}
    \menum{RDF\_LOCAL}(d) = \multiset{%
      \toolmark[2]{TT1}{\dist_\Layout(v_1,v_2)}
      \suchthat
      \toolmark[2]{TT2}{\dist(v_1,v_2)} \leq d
      \suchthat
      v_1, v_2 \in \GraphV
    }
  \end{equation*}
  \begin{tikzpicture}[remember picture, overlay]
    \node<2>[tooltip, below left = 2ex of TT1, anchor = north east] (TT1text)
            {distance between \(\Layout(v_1)\) and \(\Layout(v_2)\)};
    \node<2>[tooltip, below left = 6ex of TT2, anchor = north east] (TT2text)
            {length of shortest path from \(v_1\) to \(v_2\)};
    \draw<2>[toolarrow] (TT1text.east) .. controls (TT1|-TT1text.east) .. (TT1);
    \draw<2>[toolarrow] (TT2text.east) .. controls (TT2|-TT2text.east) .. (TT2);
  \end{tikzpicture}
\end{frame}

\PropertyDemo[RDF\_LOCAL(1)]{rdf-local-0}{Pairwise Distances}
\PropertyDemo[RDF\_LOCAL(2)]{rdf-local-1}{Pairwise Distances}
\PropertyDemo[RDF\_LOCAL(4)]{rdf-local-2}{Pairwise Distances}
\PropertyDemo[RDF\_LOCAL(8)]{rdf-local-3}{Pairwise Distances}
\PropertyDemo[RDF\_LOCAL(16)]{rdf-local-4}{Pairwise Distances}
\PropertyDemo[RDF\_LOCAL(32)]{rdf-local-5}{Pairwise Distances}

\begin{frame}
  \frametitle{Tension (\enum{TENSION})}
  \begin{itemize}
  \item Consider distribution of quotients of node and graph distances
  \item Inspired by stress but has well-behaved response to scaling
  \item Can be computed with \(\BigO(n^3)\) effort
  \end{itemize}
  \par\vfill
  \begin{equation*}
    \menum{TENSION} =
    \frac{\card{\GraphE}}{\sum\limits_{e\in\GraphE}\toolmark[2]{TT1}{\edgelen_\Layout(e)}}
    \multiset{%
      \frac{\toolmark[2]{TT2}{\dist_\Layout(v_1,v_2)}}{\toolmark[2]{TT3}{\dist(v_1,v_2)}}
      \suchthat
      v_1,v_2\in\GraphV
    }
  \end{equation*}
  \begin{tikzpicture}[remember picture, overlay]
    \node<2>[tooltip, anchor = north east, below left = 3ex of TT1] (TT1text)
            {length of edge \(e\) in layout \(\Layout\)};
    \node<2>[tooltip, anchor = south east, above left = 3ex of TT2] (TT2text)
            {distance between \(\Layout(v_1)\) and \(\Layout(v_2)\)};
    \node<2>[tooltip, anchor = north east, below left = 7ex of TT3] (TT3text)
            {length of shortest path from \(v_1\) to \(v_2\)};
    \draw<2>[toolarrow] (TT1text.east) .. controls (TT1|-TT1text.east) .. (TT1);
    \draw<2>[toolarrow] (TT2text.east) .. controls (TT2|-TT2text.east) .. (TT2);
    \draw<2>[toolarrow] (TT3text.east) .. controls (TT3|-TT3text.east) .. (TT3);
  \end{tikzpicture}
\end{frame}

\PropertyDemo[TENSION]{tension}{Tension}

\section{Data Generation}
\subsection{Graphs}

\subsubsection{Graph Generators}
\begin{frame}
  \frametitle{Graph Generators}
  \begin{center}
    \small\noindent%
    \begin{tabular}{cccc}
      \InputTikzGraph{0.2\textwidth}{pics/lindenmayer.tikz}
      & \InputTikzGraph{0.2\textwidth}{pics/quasi4d.tikz}
      & \InputTikzGraph{0.2\textwidth}{pics/bottle.tikz}
      & \InputTikzGraph{0.2\textwidth}{pics/mosaic1.tikz}\\[1ex]
      \enum{LINDENMAYER} & \enum{QUASI4D} & \enum{BOTTLE} & \enum{MOSAIC1}\\[2ex]
      \InputTikzGraph{0.2\textwidth}{pics/grid.tikz}
      & \InputTikzGraph{0.2\textwidth}{pics/torus1.tikz}
      & \InputTikzGraph{0.2\textwidth}{pics/torus2.tikz}
      & \InputTikzGraph{0.2\textwidth}{pics/mosaic2.tikz}\\[1ex]
      \enum{GRID} & \enum{TORUS1} & \enum{TORUS2} & \enum{MOSAIC2}
    \end{tabular}
  \end{center}
\end{frame}

\subsubsection{Graph Import Sources}
\begin{frame}
  \frametitle{Graph Import Sources}
  \begin{center}
    \small\noindent%
    \begin{tabular}{cccc}
      \InputTikzGraph{0.2\textwidth}{pics/rome.tikz}
      & \InputTikzGraph{0.2\textwidth}{pics/north.tikz}
      & \InputTikzGraph{0.2\textwidth}{pics/randdag.tikz}
      & \InputTikzGraph{0.2\textwidth}{pics/import.tikz}\\[1ex]
      \enum{ROME} & \enum{NORTH} & \enum{RANDDAG} & \enum{IMPORT}\\[2ex]
      \InputTikzGraph{0.2\textwidth}{pics/bcspwr.tikz}
      & \InputTikzGraph{0.2\textwidth}{pics/grenoble.tikz}
      & \InputTikzGraph{0.2\textwidth}{pics/psadmit.tikz}
      & \InputTikzGraph{0.2\textwidth}{pics/smtape.tikz}\\[1ex]
      \enum{BCSPWR} & \enum{GRENOBLE} & \enum{PSADMIT} & \enum{SMTAPE}
    \end{tabular}
  \end{center}
\end{frame}

\subsection{Layouts}
\begin{frame}
  \frametitle{Layouts}
  \begin{center}
    \small\noindent%
    \begin{tabular}{ccc}
      \InputTikzGraph{0.2\textwidth}{pics/native.tikz}
      & \InputTikzGraph{0.2\textwidth}{pics/fmmm.tikz}
      & \InputTikzGraph{0.2\textwidth}{pics/stress.tikz}\\[1ex]
      \enum{NATIVE} & \enum{FMMM} & \enum{STRESS}\\[2ex]
      \InputTikzGraph{0.2\textwidth}{pics/random-uniform.tikz}
      & \InputTikzGraph{0.2\textwidth}{pics/random-normal.tikz}
      & \InputTikzGraph{0.2\textwidth}{pics/phantom.tikz}\\[1ex]
      \enum{RANDOM\_UNIFORM} & \enum{RANDOM\_NORMAL} & \enum{PHANTOM}
    \end{tabular}
  \end{center}
\end{frame}

\subsubsection{Proper Layouts}% dummy
\subsubsection{Garbage Layouts}% dummy

\section{Data Augmentation}

\begin{frame}
  \frametitle{Data Augmentation}
  \begin{itemize}
  \item Layout worsening (unary layout transformation)
    \begin{itemize}
    \item Input: Parent layout \(\Layout\) and parameter \(0\leq{r}\leq1\)
    \item Output: Worsened layout \(\Layout_r'\)
    \end{itemize}
  \item Layout interpolation (binary layout transformation)
    \begin{itemize}
    \item Input: Parent layouts \(\Layout_\mathrm{A}\) and \(\Layout_\mathrm{B}\) and parameter \(0\leq{r}\leq1\)
    \item Output: Interpolated layout \(\Layout_r'\)
    \end{itemize}
  \end{itemize}
\end{frame}

\subsection{Layout Worsening}

\LayWorseDemo[PERTURB]{perturb}
\LayWorseDemo[FLIP\_NODES]{flip-nodes}
\LayWorseDemo[FLIP\_EDGES]{flip-edges}
\LayWorseDemo[MOVLSQ]{movlsq}
\nocite{Schaefer2006}

\subsection{Layout Interpolation}

\LayInterDemo[LINEAR]{linear}
\LayInterDemo[XLINEAR]{xlinear}

\section{Feature Extraction}
\begin{frame}
  \frametitle{Feature Extraction}
  \begin{itemize}
  \item Properties are multisets of unbounded size
  \item We need to condense them into a fixed-size feature vector
  \item Use histograms to present data
  \item Use sliding averages (\alert<2>{Gaussian kernel}) for \enum{RDF\_LOCAL} instead
  \item For each Property:
    \begin{itemize}
    \item Arithmetic mean and root mean squared (2 values)
    \item Entropy regression of histograms (2 values)
    \item Differential entropy in case of \enum{RDF\_LOCAL} (1 value)
    \end{itemize}
  \item Principal components (4 values)
  \item (Logarithm of) number of vertices and edges (2 values)
  \end{itemize}
  \par\bigskip
  All features are normalized by subtracting the mean and dividing by the standard deviation.
  \begin{tikzpicture}[remember picture, overlay]
    \node<2>[smartbox] at (current page.center) {%
      \begin{minipage}{0.4\paperwidth}
        Sliding average with kernel \(f\):
        \[ F_f(x) = \frac{\sum_{i=1}^{n} f(x,x_i)}{\int_{-\infty}^{+\infty}\diff{y}\sum_{i=1}^{n} f(y,x_i)} \]
        Gaussian kernel:
        \[ g_\sigma(\mu,x) = \frac{1}{\sigma\sqrt{2\pi}} \: e^{-\frac{1}{2}\left(\frac{x-\mu}{\sigma}\right)^2} \]
      \end{minipage}
    };
    \node<3>{};% dummy
  \end{tikzpicture}
\end{frame}

\subsection{Entropy of Histograms}
\begin{frame}
  \frametitle{Entropy of Histograms}
  \begin{definition}[Discrete Entropy]
    Let \(\Vec{H}\) be a histogram with \(n\in\IntsN\) bins that have the values (relative frequency counts)
    \(H_1,\ldots,H_2\in\RealsNN\) such that \(\sum_{i=1}^{n}H_i=1\).  Then the entropy of \(\Vec{H}\) is
    \begin{equation*}
      \Entropy(\Vec{H}) = -\sum_{i=1}^{n} H_i \log_2(H_i)
    \end{equation*}
    where we use the convention that bins with \(H_i=0\) shall contribute a zero term to the sum.
  \end{definition}
  \par\bigskip
  \begin{itemize}
  \item Depends strongly on the bin width / count
  \item Entropy grows exponentially with bin count
  \end{itemize}
\end{frame}

\begin{frame}
  \frametitle{Entropy of Histogram}
  \begin{center}
    \InputTikzGraph{0.5\textwidth}{pics/thingy.tikz}
  \end{center}
\end{frame}

\begin{frame}
  \frametitle{Entropy of Histogram}
  \begin{center}
    \InputLuatikzPlot{pics/entropy-regression.pgf}
  \end{center}
\end{frame}

\subsection{Entropy of Sliding Averages}
\begin{frame}
  \frametitle{Entropy of Sliding Averages}
  \begin{definition}[Differential Entropy]
    Let \(f:\Reals\to\RealsNN\) be a non-negative steady function normalized such that
    \(\int_{-\infty}^{+\infty}\diff{x}f(x)=1\).  The \emph{differential entropy} of \(f\) is defined as
    \begin{equation*}
      \Entropy*(f) = -\int_{-\infty}^{+\infty} \diff{x} x\log_2(x)
    \end{equation*}
    where we use the convention that the integrand shall be zero for those \(x\in\Reals\) where \(f(x)=0\).
  \end{definition}
  \par\bigskip
  \begin{itemize}
  \item Originally proposed by \textcite{Shannon1948}
  \item Not a measure of information \parencite{Jaynes1963}
  \item Can actually be negative
  \item Remotely useful transcendental properties \parencite{Cover1991}
  \end{itemize}
\end{frame}

\section{Discriminator Model}
\begin{frame}
  \frametitle{Discriminator Model}
  \begin{itemize}
  \item Receives two feature vectors of layouts \(\Layout_\mathrm{A}\) and \(\Layout_\mathrm{B}\) as inputs
  \item Outputs a number between \(-1\) and \(+1\)
  \end{itemize}
  \begin{equation*}
    \mathcal{D}(\Layout_\mathrm{A},\Layout_\mathrm{B}) =
    \begin{cases}
      <0, & \Layout_\mathrm{A} \text{ is considered better}\\
      >0, & \Layout_\mathrm{B} \text{ is considered better}\\
      =0, & \text{neither layout is considered better}
    \end{cases}
  \end{equation*}
  \begin{itemize}
  \item Siamese neural network \parencite{Bromley1994}
  \end{itemize}
\end{frame}

\begin{frame}
  \frametitle{Discriminator Model}
  \begin{center}
    \pgfimage[height=2.5in]{pics/nn-structure-total.pdf}
  \end{center}
\end{frame}

\begin{frame}
  \frametitle{Discriminator Model}
  \begin{center}
    \pgfimage[height=2.5in]{pics/nn-structure-shared.pdf}
  \end{center}
\end{frame}

\section{Evaluation}
\begin{frame}
  \frametitle{Evaluation}
  \begin{itemize}
  \item Corpora with tens of thousands of labeled pairs \((\Layout_\mathrm{A},\Layout_\mathrm{B},t)\)
  \item We ensure that \(t\not\approx0\) (\emph{i.e.}~we don't use borderline cases)
  \item 20\,\% (chosen randomly) of this data is set aside for testing
  \item \(\mathcal{D}(\Layout_\mathrm{A},\Layout_\mathrm{B})=p\) is considered a success if and only if
    \(\sign(p)=\sign(t)\)
  \item Training and testing are repeated with different partitions (cross validation via random subsampling)
  \item Reproducible success rates in excess of \(95\,\%\) achieved
  \end{itemize}
\end{frame}

\subsection{Accuracy}
\begin{frame}
  \frametitle{Accuracy}
  \InputConfusionMatrix{eval-cross-valid.tex}
\end{frame}

\subsection{Contribution of Individual Properties}
\begin{frame}
  \frametitle{Contribution of Individual Properties}
  \InputPunctureResult[eval-cross-valid.tex]{eval-puncture.tex}
\end{frame}

\section{Conclusions {\&} Future Work}
\subsection{Summary}
\begin{frame}
  \frametitle{Summary}
  \begin{itemize}
  \item Built framework for flexible experimentation
  \item Developed several graph generators
  \item Explored ways of data augmentation
  \item Investigated several properties (some of them promising)
  \item \enum{RDF\_LOCAL} is most valuable but also most expensive
  \item Demonstrated feasibility by training neural network
  \item All experiments run fully automatic and can be repeated by anybody who wishes to do so
  \item Download source code at \url{http://klammler.eu/msc/}
  \end{itemize}
\end{frame}

\subsection{Open Questions and Future Plans}
\begin{frame}
  \frametitle{Open Questions and Future Plans}
  \begin{itemize}
  \item Additional Properties
    \begin{itemize}
    \item Properties involving \emph{edges}
    \item Shapelet analysis
    \item More correlations between graph-theoretical and layout properties
    \end{itemize}
  \item More elaborate data analysis
  \item Comparison with existing measures
  \item User study
  \item Extension to more general graph drawings
  \item Application as a meta-heuristic for a genetic layout algorithm
  \end{itemize}
\end{frame}

\section{\bibname}
\begin{frame}[allowframebreaks]
  \frametitle{\bibname}
  \printbibliography
  \par\bigskip
  \emph{Please refer to the printed thesis for a complete list of references.}
\end{frame}

\end{document}
