% -*- coding:utf-8; mode:latex; -*-

% Copyright (C) 2018 Moritz Klammler <moritz.klammler@student.kit.edu>
%
% Copying and distribution of this file, with or without modification, are permitted in any medium without royalty
% provided the copyright notice and this notice are preserved.  This file is offered as-is, without any warranty.

\makeatletter

\usetheme{KIT}

\usepackage{fontspec}
\usepackage{polyglossia}
\usepackage{amsmath,amsfonts}
%\usepackage{slidecite}
\usepackage{csquotes}
\usepackage{ifdraft}
\usepackage{etoolbox}
\usepackage{tabularx}
\usepackage{booktabs}
\usepackage{multicol}
\usepackage{gnuplot-lua-tikz}
\usetikzlibrary{positioning}
\usepackage{libertine}
\setmainlanguage{english}

\usepackage[%
  backend = biber,
  citestyle = authoryear,
  bibstyle = chem-acs,
  alldates = iso8601,
  giveninits = true,
]{biblatex}

\usenavigationsymbols
\setcounter{tocdepth}{2}

\setbeamercolor*{bibliography entry author}{fg=KITblack}
\setbeamercolor*{bibliography entry title}{fg=KITblack}
\setbeamercolor*{bibliography entry journal}{fg=KITblack}
\setbeamercolor*{bibliography entry note}{fg=KITblack}

\setbeamercolor{structure}{fg=KITgreen}
%\setbeamercolor*{palette secondary}{fg=KITgreen}
%\setbeamercolor*{palette tertiary}{fg=KITgreen}

\AtBeginSection[]{%
  \begin{frame}<beamer>
    \frametitle{\contentsname}
    \begin{multicols}{2}
      \tableofcontents[currentsection,hideothersubsections]
    \end{multicols}
  \end{frame}
}

\newcommand*{\Reals}[1][]{\ifblank{#1}{\mathbb{R}}{\mathbb{R}_{#1}}}
\newcommand*{\RealsPos}{\Reals[>0]}
\newcommand*{\RealsNN}{\Reals[\geq0]}
\newcommand*{\IntsZ}{\mathbb{Z}}
\newcommand*{\IntsN}{\mathbb{N}}
\newcommand*{\IntsNz}{\mathbb{N}_0}
\newcommand*{\BitZero}{\text{\texttt{0}}}
\newcommand*{\BitOne}{\text{\texttt{1}}}
\newcommand*{\Bits}{\{\BitZero,\BitOne\}}

\let\vec\relax
\let\Vec\relax
\let\implies\Rightarrow
\let\impliedby\Leftarrow
\let\equivalent\Leftrightarrow
\newcommand*{\suchthat}{:}
\newcommand*{\degree}{^\circ}
\let\conjunction\cap
\let\disjunction\cup
\newcommand*{\diff}[1]{\mathrm{d}#1\kern0.3em}

\newcommand*{\Layout}{\Gamma}
\newcommand*{\Graph}{G}
\newcommand*{\GraphV}{V}
\newcommand*{\GraphE}{E}
\newcommand*{\vecz}{0}

\newcommand*{\GraphGVE}{\Graph=(\GraphV,\GraphE)}

\newcommand*{\CMake}{\mbox{CMake}}
\newcommand*{\CXX}{\mbox{C++}}
\newcommand*{\JavaScript}{\mbox{JavaScript}}
\newcommand*{\Keras}{\mbox{Keras}}
\newcommand*{\Python}{\mbox{Python}}
\newcommand*{\SQLite}{\mbox{SQLite}}
\newcommand*{\TensorFlow}{\mbox{TensorFlow}}
\newcommand*{\XSLT}{\mbox{XSLT}}

\newcommand*{\Entropy}{\@ifstar\EntropyDifferential\EntropyDiscrete}
\newcommand*{\EntropyDiscrete}{S}
\newcommand*{\EntropyDifferential}{\bar{S}}

\DeclareMathOperator{\BigO}{\mathcal{O}}
\DeclareMathOperator{\dist}{dist}
\DeclareMathOperator{\edgelen}{length}
\DeclareMathOperator{\mean}{mean}
\DeclareMathOperator{\rms}{rms}
\DeclareMathOperator{\sign}{sign}
\DeclareMathOperator{\stdevp}{stdevp}
\DeclareMathOperator{\stdev}{stdev}
\DeclareMathOperator{\stress}{stress}

\DeclareMathOperator*{\Conjunction}{\bigcap}
\DeclareMathOperator*{\Disjunction}{\bigcup}
\DeclareMathOperator*{\argmax}{arg\,max}
\DeclareMathOperator*{\argmin}{arg\,min}

\newcommand*{\abs}[1]{\left\lvert#1\right\rvert}
\newcommand*{\card}[1]{\left\lvert#1\right\rvert}
\newcommand*{\vecnorm}[1]{\left\lVert#1\right\rVert}
\newcommand*{\floor}[1]{\left\lfloor#1\right\rfloor}
\newcommand*{\ceil}[1]{\left\lceil#1\right\rceil}
\newcommand*{\nint}[1]{\left\lfloor#1\right\rceil}
\newcommand*{\multiset}[1]{\left[#1\right]}

\newcommand{\bra}[2][\relax]{%
  \ifx\relax#1\relax%
    \left\langle#2\right|%
  \else%
    \left\langle#2\middle|#1\right|%
  \fi%
}

\newcommand{\ket}[2][\relax]{%
  \ifx\relax#1\relax%
    \left|#2\right\rangle%
  \else%
    \left|#1\middle|#2\right\rangle%
  \fi%
}

\newcommand{\braket}[3][\relax]{%
  \ifx\relax#1\relax%
    \left\langle#2\middle|#3\right\rangle%
  \else%
    \left\langle#2\middle|#1\middle|#3\right\rangle%
  \fi%
}

\newcommand*{\enum}[1]{\mbox{\upshape\ttfamily#1}}
\newcommand*{\menum}[1]{\mbox{\text{\upshape\ttfamily#1}}}
\newcommand*{\QuasiNd}{\enum{QUASI\(\langle{n}\rangle\)D}}
\newcommand*{\TorusN}{\enum{TORUS\(\langle{n}\rangle\)}}
\newcommand*{\code}[1]{\mbox{\texttt{#1}}}

\newbool{tikzgraphpreamble}
\newlength{\samplelayoutwidth}
\newlength{\samplelayoutheight}

\newlength{\vertexsize}
\setlength{\vertexsize}{2pt}

\tikzset{
  Vertex/.style = {
    font = \footnotesize,
  },
  vertex/.style = {
    shape = circle,
    fill = KITgreen,
    inner sep = 0pt,
    minimum size = \vertexsize,
  },
  edge/.style = {
    draw = KITgreen,
    thin,
  },
  tikzgraphbbox/.style = {
    fill = \ifdraft{yellow!10}{none},
  },
  princomp/.style = {
    draw = red,
    line width = 1.0pt,
    double distance = 2.0pt,
    line cap = round,
    line join = round,
  },
  princomp1st/.style = { princomp },
  princomp2nd/.style = { princomp },
  smartbox/.style = {
    fill = KITyellow!10,
    draw = KITyellow,
    inner sep = 1ex,
    preaction = {draw, line width = 5pt, white},
  },
  tooltip/.style = {
    text = KITlilac,
    font = \footnotesize,
  },
  toolarrow/.style = {
    draw = KITlilac,
    thick,
    ->,
    shorten > = 0.5ex,
  },
}

\newcommand*{\toolmark}[3][]{%
  \tikz[remember picture, baseline = (#2.base)]{\node[inner sep = \z@](#2){\only<#1>{\color{KITlilac}}\ensuremath{#3}};}%
}

\ifdraft{%
  \newcommand*{\InputTikzGraph}{\@ifstar\InputTikzGraph@Dummy\InputTikzGraph@Dummy}
}{%
  \newcommand*{\InputTikzGraph}{\@ifstar\InputTikzGraph@Big\InputTikzGraph@NotBig}
}

\newcommand*{\InputTikzGraph@NotBig}[3][]{\tikz[x=#2, y=-#2, #1]{\input{#3}}}
\newcommand*{\InputTikzGraph@Big}[3][]{\tikz[x=#2, y=-#2, #1]{\setlength{\vertexsize}{1pt}\input{#3}}}

\newcommand*{\InputTikzGraph@Dummy}[3][]{%
  \urldef\inputtikzgraph@url\url{#3}%
  \begin{tikzpicture}[x=#2, y=#2, #1]
    \booltrue{tikzgraphpreamble}
    \input{#3}
    \begin{scope}[yscale=\aspectratio]
      \draw[use as bounding box] (-0.5, -0.5) rectangle (0.5, 0.5);
      \node[font=\tiny] at (0, 0) {\inputtikzgraph@url};
    \end{scope}
  \end{tikzpicture}
}

\newcommand*{\UpdateSampleLayoutHeight}[1]{%
  \booltrue{tikzgraphpreamble}
  \input{#1}
  \setlength{\samplelayoutheight}{\aspectratio\samplelayoutwidth}
  \boolfalse{tikzgraphpreamble}
}

\newcommand*{\AtBeginSampleLayout}{%
  \path[tikzgraphbbox, use as bounding box]
       (-0.5\samplelayoutwidth, -0.5\samplelayoutheight) rectangle (+0.5\samplelayoutwidth, +0.5\samplelayoutheight);
}

\newcommand*{\LayWorseDemo}[2][WORSE]{%
  \begin{frame}
    \frametitle{Layout Worsening (\enum{#1})}
    \setlength{\samplelayoutwidth}{0.6\textwidth}
    \UpdateSampleLayoutHeight{pics/#2-00000.tikz}
    \begin{center}
      \only< 1>{\InputTikzGraph[execute at begin picture = \AtBeginSampleLayout]{\samplelayoutwidth}{pics/#2-00000.tikz}}%
      \only< 2>{\InputTikzGraph[execute at begin picture = \AtBeginSampleLayout]{\samplelayoutwidth}{pics/#2-01000.tikz}}%
      \only< 3>{\InputTikzGraph[execute at begin picture = \AtBeginSampleLayout]{\samplelayoutwidth}{pics/#2-02000.tikz}}%
      \only< 4>{\InputTikzGraph[execute at begin picture = \AtBeginSampleLayout]{\samplelayoutwidth}{pics/#2-03000.tikz}}%
      \only< 5>{\InputTikzGraph[execute at begin picture = \AtBeginSampleLayout]{\samplelayoutwidth}{pics/#2-04000.tikz}}%
      \only< 6>{\InputTikzGraph[execute at begin picture = \AtBeginSampleLayout]{\samplelayoutwidth}{pics/#2-05000.tikz}}%
      \only< 7>{\InputTikzGraph[execute at begin picture = \AtBeginSampleLayout]{\samplelayoutwidth}{pics/#2-06000.tikz}}%
      \only< 8>{\InputTikzGraph[execute at begin picture = \AtBeginSampleLayout]{\samplelayoutwidth}{pics/#2-07000.tikz}}%
      \only< 9>{\InputTikzGraph[execute at begin picture = \AtBeginSampleLayout]{\samplelayoutwidth}{pics/#2-08000.tikz}}%
      \only<10>{\InputTikzGraph[execute at begin picture = \AtBeginSampleLayout]{\samplelayoutwidth}{pics/#2-09000.tikz}}%
      \only<11>{\InputTikzGraph[execute at begin picture = \AtBeginSampleLayout]{\samplelayoutwidth}{pics/#2-10000.tikz}}%
    \end{center}
  \end{frame}
}

\newcommand*{\LayInterDemo}[2][INTER]{%
  \begin{frame}
    \frametitle{Layout Interpolation (\enum{#1})}
    \setlength{\samplelayoutwidth}{0.6\textwidth}
    \begin{center}
      \only< 1>{\InputTikzGraph{\samplelayoutwidth}{pics/#2-00000.tikz}}%
      \only< 2>{\InputTikzGraph{\samplelayoutwidth}{pics/#2-01000.tikz}}%
      \only< 3>{\InputTikzGraph{\samplelayoutwidth}{pics/#2-02000.tikz}}%
      \only< 4>{\InputTikzGraph{\samplelayoutwidth}{pics/#2-03000.tikz}}%
      \only< 5>{\InputTikzGraph{\samplelayoutwidth}{pics/#2-04000.tikz}}%
      \only< 6>{\InputTikzGraph{\samplelayoutwidth}{pics/#2-05000.tikz}}%
      \only< 7>{\InputTikzGraph{\samplelayoutwidth}{pics/#2-06000.tikz}}%
      \only< 8>{\InputTikzGraph{\samplelayoutwidth}{pics/#2-07000.tikz}}%
      \only< 9>{\InputTikzGraph{\samplelayoutwidth}{pics/#2-08000.tikz}}%
      \only<10>{\InputTikzGraph{\samplelayoutwidth}{pics/#2-09000.tikz}}%
      \only<11>{\InputTikzGraph{\samplelayoutwidth}{pics/#2-10000.tikz}}%
    \end{center}
  \end{frame}
}

\newcommand*{\InputLuatikzPlot}[2][]{\input{#2}}

\newcommand*{\regression}[3][f]{\ensuremath{#1(x)=#2+#3x}}

\newcommand*{\PropertyDemo}[3][PROPERTY]{%
  \begin{frame}
    \frametitle{#3 (\enum{#1})}
    \begin{tabular}{ccc}
      \InputTikzGraph{30mm}{pics/demograph-a.tikz}&
      \InputTikzGraph{30mm}{pics/demograph-b.tikz}&
      \InputTikzGraph{30mm}{pics/demograph-c.tikz}\\[2ex]
      \InputLuatikzPlot{pics/#2-a.pgf}&
      \InputLuatikzPlot{pics/#2-b.pgf}&
      \InputLuatikzPlot{pics/#2-c.pgf}
    \end{tabular}
  \end{frame}
}

\newlength{\xval@hunderter}
\newlength{\xval@hunderter@pm}
\newcommand*{\InputConfusionMatrix}[1]{%
  \begingroup%
    \setlength{\parindent}{\z@}%
    \settowidth{\xval@hunderter}{\ensuremath{100.00\,\%}}%
    \settowidth{\xval@hunderter@pm}{\ensuremath{\pm100.00\,\%}}%
    \newcommand*{\FormatMeanStdev}[2]}%
      \makebox[\xval@hunderter@pm][r]{\ensuremath{\pm\kern0.2em##2\,\%}}%
    }%
    \newcommand*{\FormatCell}[1]{\FormatMeanStdev{\csuse{XVal##1Mean}}{\csuse{XVal##1Stdev}}}%
    \newcommand*{\AltLine}[2]{\multicolumn{4}{l}{\textit{##1:\hfill##2}}}%
    \input{#1}%
    \begin{tabularx}{\linewidth}{X|rr|r}%
      \toprule%
                            & \textit{Cond.~Neg.}   & \textit{Cond.~Pos.}   & \textit{$\Sigma$}\\%
      \midrule%
      \textit{Pred.~Neg.}   & \FormatCell{TrueNeg}  & \FormatCell{FalseNeg} & \FormatCell{PredNeg}\\%
      \textit{Pred.~Pos.}   & \FormatCell{FalsePos} & \FormatCell{TruePos}  & \FormatCell{PredPos}\\%
      \midrule%
      \textit{$\Sigma$}     & \FormatCell{CondNeg}  & \FormatCell{CondPos}  & \FormatMeanStdev{100.00}{0.00}\\%
      \bottomrule
      \multicolumn{4}{l}{}\\
      \AltLine{Success Rate}{\FormatCell{Success}}\\%
      \AltLine{Failure Rate}{\FormatCell{Failure}}\\[1ex]%
      \AltLine{Average Number of Tests}{\ensuremath{\approx\XValCountApprox}}\\%
      \AltLine{Number of Repetitions}{\ensuremath{\XValTestRuns}}\\%
    \end{tabularx}%
  \endgroup%
}

\newcommand*{\InputPunctureResult}[2][\relax]{%
  \begingroup%
  \newcommand*{\PunctureResult}[5][\@undefined]&\ensuremath{##3\,\%}}&
    \ifblank{##4}{\multicolumn{2}{c}{}}{\ensuremath{##4\,\%}&\ensuremath{##5\,\%}}\\
  }%
  \input{#1}%
  \begin{tabularx}{\linewidth}{X@{\qquad}r@{\makebox[1.5em][r]{$\pm$}}rr@{\makebox[1.5em][r]{$\pm$}}r}%
    \toprule%
    \textit{Property} & \multicolumn{2}{r}{\textit{Sole Exclusion}} & \multicolumn{2}{r}{\textit{Sole Inclusion}}\\%
    \midrule%
    \input{#2}%
    \midrule%
    \PunctureResult[\textit{Baseline Using All Properties}]{\XValSuccessMean}{\XValSuccessStdev}{}{}%
    \bottomrule%
  \end{tabularx}%
  \endgroup%
}

\makeatother
